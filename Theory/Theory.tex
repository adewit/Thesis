\chapter{Theory and motivation}
\label{chap:theory}
To understand and motivate the searches presented in chapters 
\ref{chap:hhh} and \ref{chap:mssm} it is important
to understand the theory and phenomenology behind them. 
This chapter describes the \acl{SM} of particle physics, the
motivations for \acl{SUSY} and the phenomenology of Higgs
sectors beyond the \acl{SM}.

\section{The \acl{SM} of particle physics}
\label{sec:theory_sm}
The \ac{SM} of particle physics is a theory that describes the \ac{EM}, weak nuclear and strong
nuclear forces, and the interaction of those forces
with particles. The \ac{SM} is a \ac{QFT} in which matter particles
are represented by spin-$\frac{1}{2}$ fermions, with the forces represented
by spin-1 bosons.

\subsection{Fundamental particles and forces}
\label{sec:theory_sm_particles}
In the \ac{SM} there are twelve fundamental fermions, six that 
interact via the strong nuclear force (quarks) and six that do not (leptons).
Each of the fermions has an antiparticle
with opposite quantum numbers. The quarks and leptons
are divided into three generations, with corresponding particles from different
generations sharing the same properties apart from their mass.
The quarks and leptons are summarised in 
table \ref{tab:theory_fermions}.
%Symmetries of the Lagrangian describing a physical system lead
%to conserved quantities, as shown by Noether's theorem \cite{noethersthm}.
%This is and important concept for the description of particle interactions
%under different forces: invariance (symmetry) under a certain 
%transformation group is imposed on 
%the Lagrangians describing the fundamental forces. This is brought
%about as local gauge invariance, where the Lagrangians 
%are required to be invariant under a local phase transformation
%plus a gauge transformation. The local phase transformation
%might introduce additional derivatives of potentials, taken care of by gauge transf.

The symmetries governing the forces in the \ac{SM} imply the presence
of spin-1 bosons which carry the forces. The gluon (\Pgluon) mediates 
the strong nuclear force, while the \ac{EM} force is mediated
by the photon (\Pphoton), and the weak force by the $\PW^{\pm}$ and \PZ bosons.

\begin{table}[htp]
\begin{center}
\caption{The fundamental fermions.}
\begin{tabular}{@{}llll@{}}
\toprule
 & \textbf{$1^{\text{st}}$ generation} & \textbf{$2^{\text{nd}}$ generation} & \textbf{$3^{\text{rd}}$ generation}\\
\midrule
\multicolumn{4}{c}{\textbf{Quarks}}\\
\midrule
Charge: +$\frac{2}{3}$& up (\Pup)  & charm (\Pcharm) & top (\Ptop) \\
Charge: -$\frac{1}{3}$& down (\Pdown) & strange (\Pstrange) & bottom (\Pbottom) \\
\midrule
\multicolumn{4}{c}{\textbf{Leptons}} \\
\midrule
Charge: -1 & electron (\Pe) & muon (\Pgm) & tau (\Pgt) \\
Charge: 0  & electron neutrino ($\Pgn_{\Pe}$) & muon neutrino ($\Pgn_{\Pgm}$) & tau neutrino ($\Pgn_{\Pgt}$)\\
\bottomrule
\end{tabular}
\label{tab:theory_fermions}
\end{center}
\end{table}

The quantum number associated with the strong force is colour, which
both the quarks and the gluons carry.
Within the framework of \ac{QFT}, the
strong force is described by \ac{QCD}, governed by an SU(3) (colour) symmetry.
%There are 8 gluons due to the SU(3) symmetry, colour octet and a colour singlet, but
%if there was a colour singlet it should occur as a free particle while it does not.
%See chapter 285 of griffiths
Characteristics of \ac{QCD} include confinement, meaning quarks are always observed in bound
states called hadrons, and asymptotic
freedom \cite{asympt-I,asympt-II}. Due to asymptotic freedom the 
strong coupling strength decreases at 
high energy, or short distances, making the force weaker. 
A more detailed introduction to \ac{QCD} can 
be found in reference \cite{griffiths}.

The unification of the weak nuclear and \ac{EM} forces, introduced 
in the 1960s by Glashow \cite{glashow-ewk}, Weinberg \cite{weinberg-ewk} and Salam \cite{salam-ewk}, 
is a key component of the \ac{SM}. The electroweak force 
is governed by an SU(2)$_{\text{L}}\times$U(1)$_{\text{Y}}$ symmetry and implies the presence of weak
neutral currents. 

%Particles that carry electric charge interact via the
%electromagnetic force, while the weak force works
%between particles carrying weak isospin.
%The weak force, which interacts with all fermions,
%only couples to the left-handed part of fermion states. In 
%A general fermion state $\chi$ can be decomposed into its left- and right-handed parts
%\begin{equation}\label{eqn:leftright}
%\chi=\psi_R+\psi_L.
%\end{equation}
%The weak force only couples to $\psi_L$, and is thus violates parity maximally.
%where only $\psi_L$ will interact via the weak force. %For massless particles
%the general state $\chi$ is not a mixture of left-handed and right-handed parts,
%but is either completely left-handed or right-handed. As such neutrinos, which are
%massless in the \ac{SM}, must all be left-handed. %CAREFUL: handedness =/= helicity
%The observation of neutrino flavour oscillations since the formulation of
%the \ac{SM} implies that neutrinos do have a small but finite mass
%and so there must be right-handed neutrinos in the \ac{SM}. This will
%not be discussed here, for a more detailed account see reference \cite{pdg-2014}.
%The weak isospin acts on doublets of left-handed lepton pairs, while
%the right-handed leptons transform as singlet states. This works similarly
%for the quarks, with weak isospin acting three left-handed doublets and
%Another feature of the weak force is that the weak eigenstates of the quarks
%differ from the physical states, which mix via the \ac{CKM}-matrix \cite{ckm-matrix-cabibbo,ckm-matrix-KM}
%in charged current interactions as
%\begin{equation}\label{eqn:ckm_matrix}
%\begin{pmatrix} d' \\
%s' \\
%b'  \end{pmatrix} = \begin{pmatrix} V_{ud} & V_{us} & V_{ub} \\
%V_{cd} & V_{cs} & V_{cb} \\
%V_{td} & V_{ts} & V_{tb} \end{pmatrix} \begin{pmatrix} d \\
%s\\
%b \end{pmatrix}.
%\end{equation}
The SU(2) part of the symmetry is built from the weak
isospin generators \mbox{$I_{1,2,3} = \frac{1}{2}\sigma_{1,2,3}$},
where $\sigma_i$ are the Pauli spin matrices. The U(1)
symmetry comes from the weak hypercharge generator $Y$, which
is related to weak isospin and electric charge as
\begin{equation}\label{eqn:hypercharge}
Q = I_3 + \frac{1}{2}Y.
\end{equation}
The associated gauge fields are the three weak isospin fields $\PW_{\mu}^{i}$ and the
weak hypercharge field $\text{B}_{\mu}$. These four fields mix to form the physical \Pphoton,
\PZ and $\PW^{\pm}$ states as
\begin{equation}\label{eqn:ewk_propagators_W}
\PW_{\mu}^{\pm} &= \frac{1}{\sqrt{2}}(\PW_{\mu}^1 \mp i \PW_{\mu}^2),
\end{equation}
\begin{equation}\label{eqn:ewk_propagators_phot}
\text{A}_{\mu} &= \PW_{\mu}^3\sin{\theta_W} + \text{B}_{\mu}\cos{\theta_W},
\end{equation}
\begin{equation}\label{eqn:ewk_propagators_Z}
\PZ_{\mu} &= \PW_{\mu}^3\cos{\theta_W} - \text{B}_{\mu}\sin{\theta_W}.
\end{equation}
In these equations $\text{A}_{\mu}$ is the photon field and $\PZ_{\mu}$ the \PZ field.
The weak mixing angle $\theta_W$ is related to the weak ($g$) and electromagnetic ($g'$)
coupling constants as
\begin{equation}\label{eqn:thetaw}
\begin{split}
g\sin{\theta_W} &= g'\cos{\theta_W}\\
\Rightarrow \sin{\theta_W} &= \frac{g'}{\sqrt{g^2+g'^2}} \text{ and } \cos{\theta_W} = \frac{g}{\sqrt{g^2+g'^2}}.
\end{split}
\end{equation}
Neutral weak-current interactions were discovered in 1973 at the Gargamelle
bubble chamber experiment at \acs{CERN} \cite{gargamelle}. This discovery was 
followed ten years later by the discovery of the $\PW^{\pm}$ \cite{UA1-1,UA2-1} and $\PZ$ \cite{UA1-2,UA2-2} bosons by the UA1
and UA2 Collaborations at \acs{CERN}, confirming the predictions from electroweak unification.
It should be noted that the weak force only couples to the left-handed parts
of fermion states, and so is maximally parity violating.

An issue with electroweak theory is that the gauge invariance of the Lagrangian would be
broken by the addition of mass terms of the form $-\frac{1}{2}m^2\PZ_{\mu}\PZ^{\mu}$ 
for the gauge bosons. However, the \PW and \PZ bosons are known to be massive. 
A similar problem exists for fermions, as they are known to have mass but again the introduction
of mass terms would break the gauge invariance of the Lagrangian. 
To be able to introduce mass terms electroweak symmetry must be broken. In the 
\ac{SM} this happens via the Higgs mechanism.

\subsection{The Higgs mechanism}
\label{sec:theory_sm_higgsmech}
The mechanism of electroweak symmetry breaking, the
Higgs mechanism, was proposed in the 1960s by Englert and Brout \cite{englertbrout},
Higgs \cite{higgs-I,higgs-II,higgs-III} and Guralnik, Hagen and Kibble \cite{GHK}.
The idea is to add a field that is symmetric under gauge transformations, but
has a non-zero vacuum expectation value and thus breaks the symmetry spontaneously.
According to Goldstone's theorem \cite{goldstone-theorem,goldstone-theorem-2} a 
consequence of spontaneous symmetry breaking is the appearance of a 
massless scalar (Goldstone boson) for each broken generator.

The simplest field that needs to be added to break electroweak symmetry is a complex doublet,
\begin{equation}\label{eqn:phi_hfield}
\Phi = \begin{pmatrix} \phi^{+} \\
\phi^{0} \end{pmatrix}.
\end{equation}
The corresponding Lagrangian is of the form
\begin{equation}\label{eqn:higgs_gen_lag}
\begin{split}
&\mathcal{L}_{\text{Higgs}} = -(D_{\mu}\Phi)^{\dagger}(D^{\mu}\Phi) - V(\Phi^{\dagger}\Phi) \text{, with }\\
&V = \lambda(\Phi^{\dagger}\Phi - \frac{\mu_{\text{SM}}^2}{2\lambda})^2,
\end{split}
\end{equation}
where $\lambda$ and $\mu_{\text{SM}}^2$ are two real parameters. The parameter $\lambda$ 
is a quartic coupling strength modifier, and it will later be shown 
that $\mu_{\text{SM}}$ is a mass parameter.
For the vacuum to be stable $\lambda$ must be positive,
and for spontaneous symmetry breaking $\mu_{\text{SM}}^2$ is also required to be positive.

Under a local SU(2)$\times$U(1) transformation, the field transforms as,
\begin{equation}\label{eqn:hfield_trsf}
\begin{pmatrix} \phi^+ \\
\phi^0 \end{pmatrix} = \exp{\{\frac{i}{2}(\vec{\theta}\cdot\vec{\sigma} + \rho)\}}
\begin{pmatrix} \phi^+ \\
\phi^0 \end{pmatrix}, 
\end{equation}
where $\vec{\theta}$ and $\rho$
are phase transformations which are functions of the space-time coordinates. The covariant
derivative associated with such a transformation is
\begin{equation}\label{eqn:H_cov_deriv}
D_{\mu} = \partial_{\mu} - \frac{i}{2}g\vec{\PW_{\mu}}\cdot\vec{\sigma} - \frac{i}{2}g'\text{B}_{\mu}.
\end{equation}
In the vacuum state, $V(\Phi^{\dagger}\Phi) =0$, $\Phi^{\dagger}\Phi=\frac{\mu_{\text{SM}}^2}{2\lambda}$ and so
$\Phi$ has to be non-zero.
Now choosing $\phi^+$ = 0 and taking $\phi^0$ to be real, this means:
%\begin{equation}\label{eqn:vacuum_expt}
%\begin{split}
%&\Phi^{\dagger}\Phi - \frac{\mu^2}{2\lambda} = 0 \Rightarrow \\
%&\Phi^{\dagger}\Phi = \frac{\mu^2}{2\lambda} \Rightarrow\\
%&\phi^{+*}\phi^+ + \phi^{0*}\phi^0 = \frac{\mu_^2}{2\lambda} \text{ and choosing } \phi^+ \text{ = 0 and } \phi^0 \text{ to be real,}\\
%&\phi^0 = \frac{v}{\sqrt{2}} \text{ where } v \text{ is real and} v^2=\frac{\mu^2}{\lambda}.
%\end{split}
%\end{equation}
\begin{equation}\label{eqn:field_vev}
\langle 0 | \Phi | 0 \rangle = \begin{pmatrix} 0 \\
\frac{v}{\sqrt{2}} \end{pmatrix},
\end{equation}
where $v$ is a real parameter, $v^2 \equiv \frac{\mu_{\text{SM}}^2}{\lambda}$.

Considering infinitesimal fluctuations around the vacuum state, it can be shown that the
only remaining generator that leaves the field invariant is $I_3+\frac{Y}{2}$, the generator
of the U(1) group.
%\begin{equation}\label{eqn:fluctvac}
%\begin{split}
%&\delta\Phi = \frac{i}{2}(\vec{\theta}\cdot\vec{\sigma} + \rho)\Phi\\
%&=\frac{i}{2}\begin{pmatrix} \theta_3 + \rho & \theta_1 - i\theta_2 \\
%\theta_1+i\theta_2 & \rho - \theta_3 \end{pmatrix} \begin{pmatrix} 0\\
%\frac{v}{\sqrt{2}} \end{pmatrix}.
%\end{split}
%\end{equation}
%Requiring $\delta\Phi=0$, we have $\theta_1=\theta_2=0$ and $\rho=\theta_3$, 
%and the only generator that leaves the field invariant is $I_3+Y$, the generator of the U(1) group.
Denoting the 
remaining, broken, generators as $\vec{b}$ and parameterising the field as an expansion around the
vacuum state gives
\begin{equation}\label{eqn:vev_expansion}
\Phi = \exp{\{\frac{i}{\sqrt{2}v}\vec{\theta}\cdot\vec{b}\}}\begin{pmatrix} 0 \\
\frac{1}{\sqrt{2}} (v+\PHiggs) \end{pmatrix}.
\end{equation}
By choosing an appropriate gauge the Goldstone bosons $\vec{\theta}$
can be eliminated. They become the longitudinal degrees of freedom of the $\PW^{\pm}$ and \PZ 
%longitudinal polarisation: ms=0
bosons. This then leaves
\begin{equation}\label{eqn:vev_expansion_satisfied}
\Phi = \begin{pmatrix} 0 \\
\frac{1}{\sqrt{2}}(v+\PHiggs) \end{pmatrix}.
\end{equation}
Substituting this into the Lagrangian of equation \ref{eqn:higgs_gen_lag}, 
and using the definitions of $\PW_{\mu}^{\pm}$ and $\PZ_{\mu}$ from 
equations \ref{eqn:ewk_propagators_W}--\ref{eqn:ewk_propagators_Z}, the Lagrangian becomes
\begin{equation}\label{eqn:completed_lagrangian}
\begin{split}
&\mathcal{L}_{\text{Higgs}} = -\frac{1}{2}\partial_{\mu}\PHiggs\partial^{\mu}\PHiggs -\frac{g^2v^2}{4}\PW_{\mu}^- \PW^{\mu +} - \frac{g^2+g'^2}{8}v^2\PZ_{\mu}\PZ^{\mu}  - \mu_{\text{SM}}^2\PHiggs^2  + \\ 
&\text{terms of at least }\mathcal{O}(3)\text{ in the fields}.
\end{split}
\end{equation}
This Lagrangian contains mass terms for the $\PW^{\pm}$ and \PZ bosons, with 
$m_{\PW} = \frac{gv}{2}$ and $m_{\PZ} = \frac{1}{2}v\sqrt{g^2+g'^2}$. Using equation \ref{eqn:thetaw} 
it can be seen that these two masses are
related as $\frac{m_{\PW}}{m_{\PZ}} = \cos{\theta_W}$. The remaining mass term is for the Higgs field \PHiggs, 
$M_{\PHiggs} = \sqrt{2\mu_{\text{SM}}^2}$. It is denoted $M_{\PHiggs}$ to avoid confusion with the mass of the heavy scalar
\PHiggs boson in the \ac{MSSM}, $m_{\PHiggs}$, which will be introduced in section \ref{sec:theory_MSSM_H}.
Note that the photon does not acquire a mass.
%Mass term : -1/2m_X^2. However for m_W note W_mu+Wmu- = 1/2(W_mu^1+iW_mu^2)(W_mu^1-iW_mu^2) = 
%1/2*(W_mu^1W_mu^1+W_mu^2W_mu^2). But also W+^2+W^-2 = W1^2+W2^2 so masses end up the same

Using these developments it is also possible to add mass terms for the fermions
through Yukawa couplings between the fermion and Higgs fields, which are of the form
\begin{equation}\label{eqn:yukawa_coupl}
\lambda_f(\bar{\psi_L}\Phi\psi_R + \bar{\psi_R}\Phi\psi_L).
\end{equation}
The parameter $\lambda_f$ is a coupling constant specific to each fermion, $\lambda_f \propto m_f$.
This means that heavier fermions couple more strongly to the Higgs field.

\section{Standard model Higgs boson measurements}
\label{sec:theory_smH}
The mass of the Higgs boson is a free parameter in the \ac{SM}, and so
a large range of possible masses needed to be covered in Higgs boson searches. 
Searches for the Higgs boson at the \ac{LEP} collider did not lead
to its observation. However, masses below $114.4\,\GeV$ were excluded at the 95\% \ac{CL} \cite{LEP-Higgs},
providing a lower search limit. Subsequent searches at the 
Tevatron also did not lead to the discovery of the Higgs boson, but did exclude 
an additional mass range of $140$--$186\,\GeV$ at the \mbox{95\% \ac{CL} \cite{TEV-Higgs}}.

On the fourth of July 2012 the ATLAS and CMS Collaborations
announced the discovery of a boson with a mass of around $125\,\GeV$ \cite{HDiscoveryATLAS,HDiscoveryCMS}.
The discovery was based on the observation of excesses in the 
search for $\PHiggs\rightarrow\gamma\gamma$ and $\PHiggs\rightarrow \PZ\PZ\rightarrow 4\ell$, with a significance 
above 5$\sigma$.
More data were collected and analysed during the remainder of 2012, and subsequent studies of its properties,
such as spin, parity \cite{ATLASspin,CMSspin} and couplings to other particles, 
increased the confidence in the compatibility of this new state with the \ac{SM} Higgs boson. 
The most precise measurement of the mass of the Higgs boson, \mbox{$M_{\PHiggs} = 125.09 \pm 0.21 \text{ (stat)} \pm 0.11 \text{ (syst)}\,\GeV$}, 
comes from the ATLAS and CMS combined
mass measurement using the full dataset collected up to the end of 2012 \cite{MassComb}.
The combined ATLAS and CMS production and decay rate measurements, using the same dataset,
show very good agreement with the \ac{SM} predictions \cite{CouplComb}. In addition, 
the $\PHiggs \rightarrow \Pgt\Pgt$ decay was observed with a significance of 5.5$\sigma$ through this combination. 
The individual searches by the ATLAS and CMS Collaborations resulted in observed 
significances of $4.5\sigma$ \cite{ATLAS-tautau} and $3.2\sigma$ \cite{SMHtautauCMS} for this process, respectively. 
%Despite the good agreement with the \ac{SM}, some discrepancies do remain, 
%and so the measurements of \ac{SM} Higgs boson

\begin{figure}[h!]
\begin{center}
\subfloat[Gluon fusion]{\includegraphics[width=0.5\textwidth]{./Theory/Figures/feynman_ggH.pdf}}
\subfloat[Vector boson fusion]{\includegraphics[width=0.5\textwidth]{./Theory/Figures/feynman_qqH.pdf}}~\\
\subfloat[W/Z associated production]{\includegraphics[width=0.5\textwidth]{./Theory/Figures/feynman_VH.pdf}}
\subfloat[\ttbar associated production]{\includegraphics[width=0.5\textwidth]{./Theory/Figures/feynman_ttH.pdf}}
\end{center}
\caption[Tree-level Feynman diagrams for the dominant Higgs boson production modes at the LHC]{Tree-level Feynman diagrams for the dominant Higgs boson production modes at
the \ac{LHC}.}
\label{fig:theory_smhprod}
\end{figure}

Figure \ref{fig:theory_smhprod} shows Feynman diagrams of the dominant Higgs boson production modes
at the \ac{LHC}. The production cross sections for each mode, shown in figure
\ref{fig:theory_smhxsbr}a, show that gluon fusion production is by far dominant. The other modes
shown have cross sections at least an order of magnitude smaller, but are of importance due to their topologies.
Tagging the additional leptons or jets in the final states of
these production modes can reduce the size
of the \ac{SM} background.

\begin{figure}[h!]
\begin{center}
\subfloat[\ac{SM} Higgs boson production cross sections]{\includegraphics[width=0.5\textwidth]{./Theory/Figures/plot_13tev_H_sqrt.pdf}}
\subfloat[\ac{SM} Higgs boson branching ratios]{\includegraphics[width=0.5\textwidth]{./Theory/Figures/SMHiggsBRYR4-square.pdf}}
\end{center}
\caption[SM Higgs boson production cross sections at $\sqrt{s}=13\,\TeV$ and SM Higgs boson branching ratios, for Higgs boson masses between $120$ and $130\,\GeV$]{(a) \ac{SM} Higgs boson production cross sections at $\sqrt{s} = 13\,\TeV$ and (b) \ac{SM}
Higgs boson branching ratios, for Higgs boson masses between $120$ and $130\,\GeV$ \cite{YR4}.}
\label{fig:theory_smhxsbr}
\end{figure}

Figure \ref{fig:theory_smhxsbr}b shows that
the branching ratios of
$\PHiggs \rightarrow \gamma\gamma$ and $\PHiggs \rightarrow \PZ\PZ$, the two discovery
channels, are smaller than the branching ratios into many other final states.
These two channels are thus not sensitive as a result of the size of their branching ratios, but
rather due to the small backgrounds in these analyses as well as the excellent photon and lepton energy resolution the detectors provide.

Since the restart of the LHC in 2015, \ac{SM} Higgs boson measurements, and searches for the 
\ac{SM} Higgs boson in decay channels not previously established, have 
already been performed. The decay into $\gamma\gamma$ has been re-established \cite{CMSHgamgam2016,ATLASHgamgam2016},
as has the decay into $\PZ\PZ\rightarrow 4\ell$ \cite{CMSHZZ2016,ATLASHZZ2016}, and measurements
of $\PHiggs\rightarrow \PW\PW$ have been performed \cite{CMSHWW2016,ATLASHWW2016}. Establishing evidence for the decay of 
$\PHiggs\rightarrow \Pbottom\Pbottom$ \cite{CMSVBFHbb2016,ATLASVHbb2016} and of $\PHiggs\rightarrow \mu\mu$ \cite{ATLASHmm2016} 
is still to be achieved.
Finally, searches for $\Ptop\Ptop\PHiggs$ production, which are important to directly probe the top-Higgs coupling, have also been carried 
out \cite{CMSttH2016,CMSttHmultilep2016,ATLASttHbb2016,ATLASttHmultilep2016}. As more data
are collected studies at the \acs{LHC} will continue to test the compatibility of the Higgs boson with the \ac{SM}.

\section{Beyond the standard model}
\label{sec:theory_BSM}
The \ac{SM} has been successfully tested to high accuracy \cite{pdg-2014}. However,
some issues remain that cannot be addressed by the \ac{SM} alone \cite{griffiths}.
For example, it does not provide a candidate for dark matter, which is estimated
to make up 25\% of the matter-energy density in the universe.
In addition, the running coupling constants of the electroweak and
strong forces do not intersect at a common energy scale. Another problem is that the force
of gravity does not appear in the \ac{SM}.

These points aside, the most important issue concerning
the Higgs sector of the \ac{SM} is the hierarchy problem.
The \ac{SM}
is accepted to be an effective field theory that describes
the elementary particles and their interactions well, up to an energy scale $\Lambda$ beyond which it breaks down.
This is the energy scale at which new physics must enter.
It is known that $\Lambda \leq \mathcal{O}(10^{19})$, the Planck scale, as quantum 
gravitational effects start to become important in that region. The energy scale $\Lambda$ enters
the corrections due to fermion and boson loops to the Higgs boson mass quadratically \cite{MSSM-carena-haber}:
\begin{equation}\label{eqn:mh_hierarchy}
\begin{split}
&M_{\PHiggs_{\text{SM}}}^2  = (M_{\PHiggs})_0^2 + \Delta M_{\PHiggs}^2,\\
&\Delta M_{\PHiggs}^2 \sim \mathcal{O}(\Lambda^2).
\end{split}
\end{equation}
Thus assuming that there is no new physics all the way up to
the Planck scale the observation $M_{\PHiggs}=125\,\GeV$ would only be compatible with $\Lambda$ of order $10^{19}\,\GeV$
if there was extreme fine-tuning of the bare Higgs boson mass $(M_{\PHiggs})_0^2$.

Many \acf{BSM} theories that can address the hierarchy problem, and some of
the other issues that were mentioned, have been developed.
One of the most
popular of these theories is \acf{SUSY} \cite{SUSY-primer}, which postulates
that there is a symmetry between bosons and fermions. This means every fermionic
\ac{SM} particle has a bosonic superpartner (sfermion, for scalar fermion)
and every bosonic \ac{SM} particle
has a fermionic superpartner (boson name + `ino', e.g. higgsino, Wino).

If this symmetry is unbroken the \ac{SUSY} particles must
have the same mass as their \ac{SM} partners. However, if that had 
been the case such particles would have been detected a long time ago. The 
only possibility is then that \ac{SUSY} is a broken symmetry and the superpartners
are heavier than their \ac{SM} equivalents. It is important to note
that the fermion and boson loops contribute to the Higgs boson
mass corrections with opposite sign, and so the symmetry between bosons and fermions 
allows for the cancellation of $\Delta M_{\PHiggs}^2$ terms. The cancellation
would be exact if \ac{SUSY} were unbroken, but because this is not the case the
cancellation is not complete. Therefore if the \ac{SUSY} breaking scale, where
new particles should be found, is much larger than a few TeV further fine-tuning of
the bare Higgs boson mass would be required \cite{MSSM-carena-haber,SUSY-primer}.
Apart from solving the hierarchy problem, \ac{SUSY} addresses some
of the other aforementioned issues. The lightest \ac{SUSY} particle
would be a candidate for dark matter if it were stable \cite{SUSY-primer}. On top of that,
\ac{SUSY} allows the electroweak and strong coupling
constants to intersect at a common energy scale of $\mathcal{O}(10^{16}\,\GeV)$ \cite{GUT-LEP}. 

The simplest supersymmetric extension of the \ac{SM}, the \ac{MSSM} \cite{SUSY-primer}, only adds the minimum
number of particles and fields required to formulate a supersymmetric theory.
%And conserves R-parity, see section 6.2 of the SUSY primer

\subsection{The Higgs sector of the \acs{MSSM}}
\label{sec:theory_MSSM_H}
At least two Higgs doublets are required in the
Higgs sector of the \ac{MSSM}, and so in its simplest form two complex Higgs doublets are added \cite{MSSM-carena-haber},
\begin{equation}\label{eqn:mssm_higgsdoublets}
\Phi_d = \begin{pmatrix} \phi_d^0 \\
\phi_d^- \end{pmatrix} \text{ and } \Phi_u = \begin{pmatrix} \phi_u^+ \\
\phi_u^0 \end{pmatrix}.
\end{equation}
Note that $\phi_d^0$ couples exclusively to down-type fermions, and $\phi_u^0$ couples
exclusively to up-type fermions.
%This leads to potential terms in the MSSM Lagrangian of the form:
%\begin{equation}\label{eqn:mssm_lagrangian_potential}
%V = \mu(\phi_u^+\phi_d^- - \phi_u^0\phi_d^0),
%\end{equation}
%with $\mu$ a mass parameter, equivalent to the \ac{SM} Higgs mass parameter.
Electroweak symmetry is broken as in the \ac{SM} case and
minimising the potential associated with the two Higgs doublets gives,
\begin{equation}\label{eqn:mssm_minimpot}
\langle 0|\Phi_d| 0 \rangle = \begin{pmatrix}v_d\\
0 \end{pmatrix} \text{ and } \langle 0 |\Phi_u|0\rangle  = \begin{pmatrix} 0\\
v_u \end{pmatrix}.
\end{equation}
This can be used to define
\begin{equation}\label{eqn:tanb_def}
\tan{\beta} \equiv \frac{v_u}{v_d}.
\end{equation}
Of the eight degrees of freedom due to the introduction of two complex
doublets, three become longitudinal states of the $\PW^{\pm}$ and \PZ bosons.
The remaining five degrees of freedom lead to five physical Higgs bosons: a charged
Higgs boson pair $\PHiggs^{\pm}$, the neutral pseudoscalar \PHiggsps
and the neutral scalars \PHiggslight and \PHiggs. These arise from the mixing of the
fields as:
\begin{align}\label{eqn:MSSMHiggsMixing}
&\PHiggs^+ = \phi_u^+\cos{\beta} + \phi_d^{-\dagger}\sin{\beta},\\
&\PHiggs^- = \phi_u^{+\dagger}\cos{\beta} + \phi_d^{-}\sin{\beta},\\
&\PHiggsps = \sqrt{2}(\text{Im}\phi_d^0\sin{\beta} + \text{Im}\phi_u^0\cos{\beta}),\\
&\PHiggslight = -(\sqrt{2}\text{Re}\phi_d^0 - v_d)\sin{\alpha} + (\sqrt{s}\text{Re}\phi_u^0 -v_u)\cos{\alpha},\\
&\PHiggs = (\sqrt{2}\text{Re}\phi_d^0-v_d)\cos{\alpha}+(\sqrt{2}\text{Re}\phi_u^0-v_u)\sin{\alpha}.
\end{align}
These mixtures depend on \tanb~and the mixing angle $\alpha$ which is found
by diagonalising
the neutral scalar Higgs squared-mass matrix,
\begin{equation}\label{eqn:treelevel_mass}
\mathcal{M}_{\text{tree}}^2 = \begin{pmatrix} 
m_{\PHiggsps}^2\sin{\beta}^2 + m_{\PZ}^2\text{cos}^2\beta & -(m_{\PHiggsps}^2+m_{\PZ}^2)\sin{\beta}\cos{\beta}\\
-(m_{\PHiggsps}^2+m_{\PZ}^2)\sin{\beta}\cos{\beta} & m_{\PHiggsps}^2\text{cos}^2\beta+m_{\PZ}^2\text{sin}^2{\beta} \end{pmatrix},
\end{equation}
of which \PHiggs and \PHiggslight are the eigenstates.
As the self-interactions of the Higgs fields are not independent parameters,
%they are functions of the electroweak gauge coupling constants 
at tree level the MSSM Higgs sector is determined by two free parameters, \tanb~and
one of the Higgs boson masses, conventionally chosen as \mA.
The masses of the neutral scalars \PHiggs and \PHiggslight are the eigenvalues  
of $\mathcal{M}_{\text{tree}}^2$,
\begin{equation}\label{eqn:scalarmass}
m_{\PHiggs,\PHiggslight}^2 = \frac{1}{2}[m_{\PHiggsps}^2+m_{\PZ}^2 \pm \sqrt{(m_{\PHiggsps}^2+m_{\PZ}^2)^2-4m_{\PZ}^2m_{\PHiggsps}^2\text{cos}^2 2\beta}~].
\end{equation}
The masses of the charged Higgs bosons are
\begin{equation}\label{mssm_chargedhigss_mass}
m_{\PHiggs^{\pm}}^2 = m_{\PHiggsps}^2+m_{\PW}^2.
\end{equation}
%while the neutral scalar \PHiggslight and \PHiggs are eigenstates of the
%squared-mass matrix at tree level:
%\begin{equation}\label{eqn:treelevel_mass}
%\mathcal{M}_{\text{tree}}^2 = \begin{pmatrix} 
%m_{A}^2\sin{\beta}^2 + m_{Z}^2\cos{\beta}^2 & -(m_{A}^2+m_{Z}^2)\sin{\beta}\cos{\beta}\\
%-(m_{A}^2+m_{Z}^2)\sin{\beta}\cos{\beta} & m_{A}^2\cos{\beta}^2+m_{Z}^2\sin{\beta}^2 \end{pmatrix}.
%\end{equation}
%Such that $m_{\PHiggs,\PHiggslight}$ are eigenvalues of $\mathcal{M}_{\text{tree}}^2$,
%\begin{equation}\label{eqn:scalarmass}
%m_{\PHiggs,\PHiggslight}^2 = \frac{1}{2}(m_{A}^2+m_{Z}^2 \pm \sqrt{(m_A^2+m_Z^2)^2-4m_Z^2m_A^2\cos{2\beta}^2}).
%\end{equation}
A consequence of equation \ref{eqn:scalarmass} is a tree-level upper bound on the mass of the light Higgs
boson,
\begin{equation}\label{eqn:mh_upper}
m_{\PHiggslight} \leq m_{\PZ},
\end{equation}
and a constraint on $\alpha$,
\begin{equation}\label{eqn:alpha_constraint}
\text{cos}^2(\beta-\alpha) = \frac{m_{\PHiggslight}^2(m_{\PZ}^2-m_{\PHiggslight}^2)}{m_{\PHiggsps}^2(m_{\PHiggs}^2-m_{\PHiggslight}^2)}.
\end{equation}

The requirement that $m_{\PHiggslight}$ be less than the mass of the \PZ boson, $91.2\,\GeV$, seems 
problematic at first, as it is known that there is a Higgs state with a mass of $125\,\GeV$. However,
the tree-level masses and couplings of the Higgs bosons in the \ac{MSSM} can be
significantly altered by radiative corrections. The dominant effect is from
incomplete cancellation of the stop- and top-loop corrections. This can increase the mass
of the light Higgs boson up to a maximum of $135\,\GeV$.

The tree-level couplings of the neutral Higgs bosons to bosons and fermions 
are modified with respect to the couplings of the Higgs boson in the \ac{SM} by the multiplicative factors
given in table \ref{tab:mssm_couplings} \cite{YR4}.
\begin{table}[htp]
\label{tab:mssm_couplings}
\begin{center}
\caption[Tree-level couplings in the MSSM, as multiplicative factors with respect to the couplings of the Higgs boson in the SM.]{Tree-level couplings in the MSSM, as multiplicative factors with
respect to the couplings of the Higgs boson in the \ac{SM}.}
\begin{tabular}{p{2cm}p{4cm}p{4cm}p{4cm}}
\toprule
Particle & Coupling to bosons & Coupling to up-type quarks & Coupling to down-type quarks and leptons \\
\midrule
\PHiggsps & 0 & $\cot{\beta}$ & $ \tan{\beta}$\\
\PHiggs & $\cos{(\beta-\alpha)}$ & $\frac{\sin{\alpha}}{\sin{\beta}}$ & $\frac{\cos{\alpha}}{\cos{\beta}}$\\
\PHiggslight & $\sin{(\beta-\alpha)}$ & $\frac{\cos{\alpha}}{\sin{\beta}}$ & $-\frac{\sin{\alpha}}{\cos{\beta}}$\\
\bottomrule
\end{tabular}
\label{tab:mssm_couplings}
\end{center}
\end{table}

In the \textit{decoupling limit}, when $m_{\PHiggsps} \gg m_{\PZ}$ and $\cos{(\beta-\alpha)}$ tends
to zero, the mixing
angle $\alpha \approx \beta - \pi/2$. This reduces the couplings in table
\ref{tab:mssm_couplings} to the \ac{SM} couplings for the \PHiggslight boson, 
with negligible couplings to bosons for the two remaining neutral
Higgs bosons; couplings to down-type quarks and leptons enhanced by a factor \tanb;
and couplings to up-type quarks enhanced by a factor $\frac{1}{\tan{\beta}}$. This motivates
the choice of decay into $\Pgt\Pgt$ as a search channel for the neutral Higgs bosons in the \ac{MSSM}.
Figure \ref{fig:mssm_brtautau} shows the branching ratios of the \PHiggs boson 
in the $m_{\PHiggslight}^{\text{mod+}}$ scenario
which will be discussed in more detail in section \ref{sec:theory_BSM_models_mhmodp}. 
The branching ratio into di-tau pairs is large for \tanb~$= 30$. The branching ratios 
 of the \PHiggsps boson follow a similar pattern.
%Using trigon identities: cos alpha/cos beta = cos (beta-pi/2)/cos beta = cosbeta*cospi/2 + sin(beta)*sin(pi/2) over cos beta = sin beta /cos beta = tan beta
%sin alpha = sin (beta-pi/2) = sinbeta cos pi/2 - cosbetasinpi/2 = -cosbeta 
%such that -sinalpha/cosbeta = 1
%cos alpha/sin beta = sin beta/sin beta = 1
%sin beta-alpha = sin pi/2 = 1
%cos beta-alpha = cos pi/2 = 0
%sin alpha/sin beta = -cos beta/sin beta=-1/tan(beta)

\begin{figure}[h!]
\begin{center}
\subfloat[\tanb=5]{\includegraphics[width=0.5\textwidth]{./Theory/Figures/YR4HXS_BRSummary_H_mhmodp_tanbeta5_FeynHiggs_HDecay.pdf}}
\subfloat[\tanb=30]{\includegraphics[width=0.5\textwidth]{./Theory/Figures/YR4HXS_BRSummary_H_mhmodp_tanbeta30_FeynHiggs_HDecay.pdf}}
\end{center}
\caption[Branching ratios of the \PHiggs boson in the $m_{\PHiggslight}^{\text{mod+}}$ scenario]{Branching ratios of the \PHiggs boson in the $m_{\PHiggslight}^{\text{mod+}}$ scenario,
at (a) \tanb=5~and (b) \tanb=30. The branching ratio into $\tau\tau$ (blue), $\Pbottom\APbottom$ (red) 
and $\mu\mu$ (yellow) is enhanced at high \tanb, while the branching ratio into 
WW (brown) and \ttbar (green) is reduced \cite{MSSM-xswg-twiki}.}
\label{fig:mssm_brtautau}
\end{figure}

The dominant neutral MSSM Higgs boson production processes at hadron colliders
are slightly different from the \ac{SM} Higgs boson production processes. 
Because the pseudoscalar A does not couple to the
vector bosons, and in the decoupling limit the coupling of \PHiggs to vector
bosons is suppressed with respect to the \ac{SM} expectation, \ac{VBF} and \PW or \PZ associated
production ($\PW\PHiggs$ or $\PZ\PHiggs$) are not as important as for the production of the \ac{SM} Higgs boson. As in the \ac{SM}, gluon fusion production is 
the dominant production mode at low \tanb.
An example tree-level Feynman diagram for gluon fusion production, which proceeds via
a quark loop, is given in figure \ref{fig:production_mssm}a.
At low \tanb~values this loop is dominated by top quarks, while at 
high \tanb~the b-quark loop dominates. This is a result of the enhanced coupling to down-type fermions at high \tanb. 
Due to the negative top-bottom loop interference effect the gluon fusion cross section decreases with increasing \tanb~up
to around \tanb~$=3$, where the cross section starts to increase with increasing \tanb~again.

Another consequence of the enhanced couplings to down-type fermions at high \tanb~is the dominance of b-associated production, where the Higgs boson is produced through
bottom quark fusion. The tree-level Feynman diagram for this process is shown in figure \ref{fig:production_mssm}b.

\begin{figure}[h!]
\begin{center}
\subfloat[Gluon fusion production]{\includegraphics[width=0.5\textwidth]{./Theory/Figures/CMS-PAS-HIG-16-037_Figure_001-a.pdf}}
\subfloat[b-Associated production]{\includegraphics[width=0.5\textwidth]{./Theory/Figures/CMS-PAS-HIG-16-037_Figure_001-b.pdf}}
\end{center}
\caption[Tree-level Feynman diagrams of gluon fusion and b-associated production of neutral Higgs bosons in the MSSM]{Tree-level Feynman diagrams of (a) gluon fusion production and (b) b-associated
production of neutral Higgs bosons in the MSSM. The squarks in the gluon fusion production loop do not contribute to the production of the \PHiggsps boson.}
\label{fig:production_mssm}
\end{figure}

\subsection{Two Higgs doublet models}
\label{sec:theory_2HDM}
A different approach, also leading to an extended Higgs sector with
respect to the \ac{SM}, is found in \acp{2HDM} \cite{2HDM-I,2HDM-II}.
As the name suggests \acp{2HDM} are a generic class of models in which
there is not one, but two Higgs doublets. The addition of a second
Higgs doublet can not just be motivated by \ac{SUSY}, but also by other models.
An important point is that despite the presence of a second 
Higgs doublet, there are no explicit \ac{SUSY} particles in the \ac{2HDM}.

There are several different types of \ac{2HDM}, all distinguished by the couplings
of the different Higgs states to bosons and fermions. In its most general
form, there are nine free parameters in the potential of a CP-conserving 2HDM. 
These are taken as \tanb~and the mixing angle $\alpha$, already discussed in 
the context of the \ac{MSSM}, the masses of the five Higgs bosons,
and quartic couplings appearing in the potential. The most studied of the
\ac{2HDM}s, the type-II \ac{2HDM}, has a structure very similar to 
the MSSM Higgs sector. The tree-level couplings of the neutral Higgs bosons to 
vector bosons and fermions are the same as the tree-level \ac{MSSM} couplings 
in table \ref{tab:mssm_couplings}. %The decay \AtoZh is also possible,
%the \AtoZh~coupling is proportional to $\cos{(\beta-\alpha)}, like the \Htohh~coupling.

The behaviour of the \ac{2HDM} couplings in the \textit{alignment limit},
where $\cos{(\beta-\alpha)} = 0$, differs from that of the \ac{MSSM} couplings.
In particular, in the alignment limit of the \ac{2HDM} the decays \AtoZh~and \Htohh~vanish.
This is due to the fact that these couplings are
proportional to $\cos{(\beta-\alpha)}$ and there are no corrections
from \ac{SUSY} particles to make the branching ratios non-zero.

The inclusive cross section times branching ratio (\xsbr) for production of an \PHiggs boson 
and decays into various final states is shown in figure \ref{fig:2hdm_Hxsbr}a for an 
example of a type-II \ac{2HDM}, for two values of \tanb. Figure \ref{fig:2hdm_Hxsbr}b 
shows the inclusive \xsbr for production and decay
of an \PHiggsps boson in the same type-II \ac{2HDM} example, at two values of \tanb. At
low \tanb~the \xsbr for \Htohh production and decay and \AtoZh production
and decay are enhanced for $250\leq m_{\PHiggs}\leq 350\,\GeV$ and $220 \leq m_{\PHiggsps} \leq 350\,\GeV$, respectively.

\begin{figure}[h!]
\begin{center}
\subfloat[\PHiggs boson]{\includegraphics[width=0.516\textwidth]{./Theory/Figures/typeII2HDMxsbr.png}}
\subfloat[\PHiggsps boson]{\includegraphics[width=0.484\textwidth]{./Theory/Figures/typeII2HDMxsbrA.png}}
\end{center}
\caption[Inclusive production \xsbr for an example of a type-II 2HDM, for the \PHiggs and \PHiggsps boson.]{Inclusive production \xsbr for an example
of a type-II \ac{2HDM}, for (a) the \PHiggs boson and (b) the \PHiggsps boson. The
top figures use \tanb=1 and $\cos{(\beta-\alpha)}=-0.11$, the bottom
figures are made for \tanb=10 and $\cos{(\beta-\alpha)}=-0.02$. At low \tanb~the \Htohh and
\AtoZh decays are enhanced for masses up to $350\,\GeV$ \cite{2HDM-II}.}
\label{fig:2hdm_Hxsbr}
\end{figure}


\section{MSSM benchmark scenarios}
\label{sec:theory_BSM_models}
Because the MSSM contains a large number of
SUSY-breaking parameters that affect the Higgs
sector, it is usual to define benchmark scenarios 
in which the only free parameters are \mA~and \tanb.
In these scenarios the SUSY parameters entering in 
the radiative corrections are fixed.

The parameters that need to be fixed in the benchmark scenarios are the 
\ac{SUSY} breaking scale $M_{\text{SUSY}}$, which is also the mass scale of the third generation squarks;
the higgsino mass parameter $\mu_{\text{MSSM}}$; the U(1) and SU(2) gaugino mass parameters, $M_1$ and $M_2$;
the masses of the stau, $m_{\tilde{\ell}_3}$, and the gluino, $m_{\tilde{g}}$; the 
trilinear couplings of the stops, sbottoms and staus to the Higgs field ($A_{\Ptop}$, $A_{\Pbottom}$ and $A_{\Pgt}$); and the 
stop, sbottom and stau mixing parameters ($X_{\Ptop}$, $X_{\Pbottom}$ and $X_{\tau}$).

Some of these parameters can be related to each other. For example,
$X_{\Ptop}$, $X_{\Pbottom}$ and $X_{\Pgt}$ can be expressed as:
\begin{equation}\label{eqn:trilinear_couplings}
\begin{split}
&X_{\Ptop} = A_{\Ptop}-\mu_{\text{MSSM}}\cot{\beta},\\
&X_{\Pbottom} = A_{\Pbottom}-\mu_{\text{MSSM}}\tan{\beta},\\
&X_{\Pgt} = A_{\Pgt} - \mu_{\text{MSSM}}\tan{\beta}.
\end{split}
\end{equation}
In addition, $M_1$ is fixed via the unification
relation,
\begin{equation}
M_1 = \frac{5}{3}M_2\tan{\theta_W}^2.
\end{equation}
%where $\theta_w$ is defined by $\cos{\theta_w} = \frac{m_W}{m_Z}$.
Some additional parameters which only have a
small effect on the MSSM Higgs boson sector are 
fixed to values compatible with 
exclusion limits from direct searches in all benchmark scenarios. The masses
of the first and second generation squarks are set to $1.5\,\TeV$, 
and the masses of the first and second generation sleptons to $500\,\GeV$. The trilinear
couplings of the first and second generation squarks and sleptons are taken to be zero.
%Since the discovery of the 125 GeV Higgs boson, only the MSSM benchmark
%scenarios 
%which contain a light Higgs boson with a mass of around 125 GeV
%are still accessible.
% in the mmax scenario the benchmark values have been chosen h
%such that the mass of the light CP-even Higgs boson is maximized for fixed tanβ and large
%MA (the scale of the soft SUSY-breaking masses in the stop and sbottom sectors, which
%sets the mass scale for the corresponding supersymmetric particles, has been fixed to 1 TeV
%in this scenario). This scenario is useful to obtain conservative bounds on tanβ for fixed
%values of the top-quark mass 

\subsection{The $m_{\PHiggslight}^{\text{mod+}}$ scenario}
\label{sec:theory_BSM_models_mhmodp}
The $m_{\PHiggslight}^{\text{mod+}}$ scenario \cite{MSSM-benchmark-scenarios}
is a modification of the $m_{\PHiggslight}^{\text{max}}$ scenario \cite{MSSM-mhmax}. The $m_{\PHiggslight}^{\text{max}}$ scenario, which
was used for interpretations of \ac{MSSM}
 Higgs boson searches at LEP and the Tevatron, allows
the mass of the light Higgs boson to reach the highest a-priori expected 
value of around $135\,\GeV$ for high \mA. There is only a small area of the \mA-\tanb~plane
in this scenario where the mass of the light Higgs boson is compatible with the
observed $125\,\GeV$ state. The modifications to the parameters of the $m_{\PHiggslight}^{\text{max}}$ 
scenario address this issue. In the $m_{\PHiggslight}^{\text{mod+}}$ scenario $M_{\text{SUSY}}$ is chosen
to be $1\,\TeV$. The stop mixing parameter is positive, $X_{\Ptop}= 1.5 M_{\text{SUSY}}$.
%This gives better agreement with muon g-2 results while the mhmod- scenario
%negative stop mixing (-1.9 M_SUSY) results in better agreements with B(b->sgamma) 
%meausrements 
The remaining parameters are set as $\mu_{\text{MSSM}}=200\,\GeV$, $m_{\tilde{g}} = 1.5\,\TeV$,
$m_{\tilde{\ell}_3} = 1\,\TeV$, and $A_{\Pbottom}=A_{\Ptop}=A_{\Pgt}$. 
%Figure \ref{fig:mhmodp_mh}a
%shows the mass of the light Higgs boson in the $m_{h}^{\text{mod+}}$ scenario, showing
%that its mass is compatible with 125 GeV over a large part of the parameter space. Masses within 
%a window $\pm 3$ GeV around 125 GeV are considered compatible with 125 GeV, where the 3 GeV
%variation corresponds to the theoretical uncertainty on the \ac{MSSM} prediction for \mh.


Figure \ref{fig:mhmodp_xs} shows the cross sections of
gluon fusion and b-associated production of the \PHiggs boson 
in the $m_{\PHiggslight}^{\text{mod+}}$ scenario. Gluon fusion 
production dominates at low \tanb, while b-associated production has a larger cross section
at high \tanb. The behaviour of the production cross sections of the \PHiggsps boson
is similar.

\begin{figure}[h!]
\begin{center}
\subfloat[$\sigma(\Pg\Pg\rightarrow \PHiggs)$]{\includegraphics[width=0.5\textwidth]{./Theory/Figures/xs_ggH_mhmodp.pdf}}
\subfloat[$\sigma(\Pg\Pg\rightarrow \Pbottom\Pbottom\PHiggs)$]{\includegraphics[width=0.5\textwidth]{./Theory/Figures/xsbbH4F_mhmodp.pdf}}
\end{center}
\caption[Production cross sections of  gluon fusion and b-associated
production of the \PHiggs boson in the $m_{\PHiggslight}^{\text{mod+}}$ scenario.]{Production cross section of (a) gluon fusion production and (b) b-associated production of the \PHiggs boson
in the $m_{\PHiggslight}^{\text{mod+}}$ scenario. The gluon fusion production cross section is larger at low \tanb, while
the b-associated production cross section is larger at high \tanb. These values are based on the
calculations in reference \cite{YR3}.}
\label{fig:mhmodp_xs}
\end{figure}

Figure \ref{fig:mhmodp_br} shows the branching ratios of the \PHiggs and \PHiggsps 
bosons into $\tau\tau$. The branching ratios into $\tau\tau$ are enhanced, especially at
high \tanb, showing how this decay channel is useful for MSSM Higgs boson searches.

\begin{figure}[h!]
\begin{center}
\subfloat[$\mathcal{B}(\PHiggs \rightarrow \tau\tau)$]{\includegraphics[width=0.5\textwidth]{./Theory/Figures/brHtautau_mhmodp.pdf}}
\subfloat[$\mathcal{B}(\PHiggsps \rightarrow \tau\tau)$]{\includegraphics[width=0.5\textwidth]{./Theory/Figures/brAtautau_mhmodp.pdf}}
\end{center}
\caption[Branching ratios of the \PHiggs and \PHiggsps bosons into $\tau\tau$.]{Branching ratios of (a) the \PHiggs boson and (b) the \PHiggsps boson into $\tau\tau$. The branching
ratios into $\tau\tau$ are enhanced at high \tanb. These values are based on the
calculations in reference \cite{YR3}.}
\label{fig:mhmodp_br}
\end{figure}


\subsection{MSSM scenarios at low \tanb}
\label{sec:mssm_theory_lowtb}
The mass of the light Higgs boson in the \ac{MSSM} is said to
be compatible with $125\,\GeV$ if the light Higgs boson mass lies within 
$\pm 3\,\GeV$ of $125\,\GeV$. This mass window
corresponds to the theoretical uncertainty on the \ac{MSSM} prediction for \mh.
A large area of the \mbox{\mA-\tanb}~plane in the $m_{\PHiggslight}^{\text{mod+}}$ 
scenario contains a light Higgs boson with a mass compatible with
$125\,\GeV$. However, at low values of \tanb~this is not the case.
This is illustrated in figure \ref{fig:mhmodp_mh}a, where the mass of the light Higgs
boson in the $m_{\PHiggslight}^{\text{mod}+}$ scenario is shown.
The low \tanb~regime can be re-opened
if $M_{\text{SUSY}}$ is allowed to be greater than $3\,\TeV$ \cite{MSSM-reopen}. 
%But not too large, see ref above!
%Figure \ref{fig:tanb_accessibility} shows contours of constant
%light Higgs boson mass as a function of \tanb~and $M_{\text{SUSY}}$.
%It indicates that values of \mh~can be compatible with 125$\pm$3 GeV
%for \tanb~values down to 1 if $M_{\text{SUSY}}$ is around 100-1000 TeV. 
%\tanb~values of 3-5 are even accessible for $M_{\text{SUSY}}$ of around 10 TeV.
%
%\begin{figure}[h!]
%\begin{center}
%\includegraphics[width=0.5\textwidth]{./Theory/Figures/tanb_accessibility.png}
%\end{center}
%\caption{Contours of constant light Higgs boson mass as a function of \tanb~and $M_{\text{SUSY}}$.
%High values of $M_{\text{SUSY}}$ allow for light Higgs boson masses compatible with
%125 GeV down to low values of \tanb~\cite{hMSSM-2}.}
%\label{fig:tanb_accessibility}
%\end{figure}
%where the ±3 GeV variation corresponds to a rough estimate of the theoretical uncertainty of the MSSM prediction for mh, due to the unknown effect of higher-order correction

Two approches for the definition of an \ac{MSSM} scenario that gives
access to the low \tanb~region have been developed, they will be
discussed in the next sections.

\begin{figure}[h!]
\begin{center}
\subfloat[$m_{\PHiggslight}^{\text{mod}+}$ scenario]{\includegraphics[width=0.5\textwidth]{./Theory/Figures/mh_mhmodp.pdf}}
\subfloat[Low-\tanb~scenario]{\includegraphics[width=0.5\textwidth]{./Theory/Figures/mh_lowtbhigh.pdf}}
\end{center}
\caption[The mass of the light Higgs boson, \mh, as a function of \mA~and
\tanb~in the $m_{\PHiggslight}^{\text{mod+}}$ scenario and the low-\tanb~scenario.]{The mass of the light Higgs boson, \mh, as a function 
of \mA~and \tanb~in (a) the $m_{\PHiggslight}^{\text{mod+}}$ scenario and (b) the low-\tanb~scenario. The white areas
indicate masses lower than $122\,\GeV$. The figures show that
the mass of the light Higgs boson is compatible with $125\,\GeV$ over a large
part of the \mA-\tanb~plane in the $m_{\PHiggslight}^{\text{mod+}}$ scenario, except at low \tanb. In the low-\tanb~scenario
the mass of the light Higgs boson is compatible with $125\,\GeV$ nearly everywhere. These values are based on the calculations in reference \cite{YR3}.}
\label{fig:mhmodp_mh}
\end{figure}

\subsubsection{The low-\tanb~scenario}
\label{sec:theory_BSM_model_lowtb}
In the low-\tanb~scenario \cite{Hein-low-tb-high,MSSM-lowtanb}, the SUSY parameters entering
the radiative corrections are tuned to obtain a light 
Higgs boson mass of around $125\,\GeV$ in most of the \mbox{\mA-\tanb}~plane considered.
Figure \ref{fig:mhmodp_mh}b shows that, apart from in a corner of \tanb~$=1$--$4$ and 
\mA~$=150$--$250\,\GeV$, \mh~is compatible with $125\,\GeV$.

%\begin{figure}[h!]
%\begin{center}
%\includegraphics[width=0.5\textwidth]{./Theory/Figures/mh_lowtbhigh.png}
%\end{center}
%\caption{Mass of the light Higgs boson in the \mA-\tanb~plane of the low-\tanb~scenario.
%The mass is compatible with $125\,\GeV$ nearly everywhere \cite{MSSM-lowtanb}.}
%\label{fig:lowtbhigh_mh}
%\end{figure}
To obtain \mh~$\approx125\,\GeV$ over a large part of the parameter
space, 
$M_{\text{SUSY}}$ is not fixed but varies between a few $\TeV$ and $100\,\TeV$, while
varying the parameter $X_{\Ptop}$ as:
\begin{equation}
\begin{split}
\tan{\beta} \leq 2 &: \frac{X_{\Ptop}}{M_{\text{SUSY}}} = 2,\\
2 < \tan{\beta} \leq 8.6 &: \frac{X_{\Ptop}}{M_{\text{SUSY}}} = 0.0375\text{tan}^2\beta - 0.7\tan{\beta} + 3.25,\\
8.6 < \tan{\beta} &: \frac{X_{\Ptop}}{M_{\text{SUSY}}} = 0.
\end{split}
\end{equation}
The other trilinear couplings are set to $2\,\TeV$, with $\mu_{\text{MSSM}}$ set to $1.5\,\TeV$ and $M_2$ to $2\,\TeV$.

The branching ratios of \Htohh and \AtoZh in the low-\tanb~scenario are shown in
figure \ref{fig:lowtbhigh_br}. For both decay channels there are areas in the \mA-\tanb~plane
where the branching ratio is enhanced, indicating how analyses targeting such processes 
can be sensitive in this scenario. %REPHRASE
%The analysis presented in chapter \ref{chap:hhh} is interpreted
%in the low-\tanb~scenario.

\begin{figure}[h!]
\begin{center}
\subfloat[\Htohh]{\includegraphics[width=0.5\textwidth]{./Theory/Figures/lowtbhigh_hhbr.png}}
\subfloat[\AtoZh]{\includegraphics[width=0.5\textwidth]{./Theory/Figures/lowtbhigh_azhbr.png}}
\end{center}
\caption[Branching ratios of \Htohh and \AtoZh in the low-\tanb~scenario.]{Branching ratios of (a) \Htohh and (b) \AtoZh in the low-\tanb~scenario, indicating 
areas where both are significantly enhanced \cite{MSSM-lowtanb}.}
\label{fig:lowtbhigh_br}
\end{figure}

\subsubsection{The hMSSM scenario}
\label{sec:theory_BSM_models_hMSSM}
The hMSSM scenario \cite{hMSSM-1,hMSSM-2} uses a different approach, in
which \mh~$=125\,\GeV$ by construction. 

One of the assumptions of the hMSSM is that 
the mass matrix for the neutral CP-even states can
be decomposed as,
\begin{equation}
\label{eqn:hmssm_massmatrix}
\mathcal{M}^2_{\phi} = \mathcal{M}^2_{\text{tree}} + \begin{pmatrix}
\Delta\mathcal{M}^2_{11} & \Delta\mathcal{M}^2_{12} \\
\Delta\mathcal{M}^2_{12} & \Delta\mathcal{M}^2_{22} \end{pmatrix},
\end{equation}
where the $\Delta\mathcal{M}^2_{ij}$ are the radiative corrections.
The second assumption is that only $\Delta\mathcal{M}^2_{22}$ needs to be
taken into account, as this is the element that involves the stop-top correction
and so $\Delta\mathcal{M}^2_{22} \gg \Delta\mathcal{M}^2_{12},\Delta\mathcal{M}^2_{11}$. 
Finally, all SUSY particles are assumed to be heavy enough not to be
detected at the \acs{LHC} and apart from effects on the mass matrix, 
the effects on the Higgs sector can be neglected.
%all SUSY particles are heavy enough to escape detection at the LHC, and their effects on the Higgs sector other than those on the mass matrix, e.g. via direct loop corrections to the Higgs-boson couplings or via modifications of the total decay widths, can be neglected.

Using these assumptions the lightest eigenvalue
of the mass matrix can be inverted to get: 
\begin{equation}
\label{eqn:hmssm_deltam22}
\Delta\mathcal{M}^2_{22} = \frac{m_{\PHiggslight}^2(m_{\PHiggsps}^2+m_{\PZ}^2 - m_{\PHiggslight}^2) - m_{\PHiggsps}^2m_{\PZ}^2\text{cos}^22\beta}{m_{\PZ}^2\text{cos}^2\beta + m_{\PHiggsps}^2\text{sin}^2\beta - m_{\PHiggslight}^2}.
\end{equation}

This can be used to write,
\begin{equation}
\label{eqn:hmssm_mHalpha}
\begin{split}
m^2_{\PHiggs} &= \frac{(m_{\PHiggsps}^2+m_{\PZ}^2-m_{\PHiggslight}^2)(m_{\PZ}^2\text{cos}^2\beta+m_{\PHiggsps}^2\text{sin}^2\beta - m_{\PHiggsps}^2m_{\PZ}^2\text{cos}^22\beta)}{m_{\PZ}^2\text{cos}^2\beta+m_{\PHiggsps}^2\text{sin}^2\beta-m_{\PHiggslight}^2},\\
\tan{\alpha} &= -\frac{(m_{\PZ}^2+m_{\PHiggsps}^2)\cos{\beta}\sin{\beta}}{m_{\PZ}^2\text{cos}^2\beta+m_{\PHiggsps}^2\text{sin}^2\beta-m_{\PHiggslight}^2}.
\end{split}
\end{equation}
Combining this with the $\PHiggs$-$\PHiggslight\PHiggslight$ coupling,
\begin{equation}
\label{eqn:hmssm_Hhh}
\lambda_{\PHiggs\PHiggslight\PHiggslight} = \lambda_{\PHiggs\PHiggslight\PHiggslight,\text{tree}} + 3\frac{\Delta\mathcal{M}^2_{22}\sin{\alpha}}{m_{\PZ}^2\sin{\beta}}\text{cos}^2\alpha,
\end{equation}
gives enough information to determine the cross sections and branching
ratios of all of the five Higgs bosons as a function of \mA~and \tanb. The scenario
is only well defined in regions where the denominator in equations
\ref{eqn:hmssm_deltam22} and \ref{eqn:hmssm_mHalpha}, $m_{\PZ}^2\text{cos}^2\beta+m_{\PHiggsps}^2\text{sin}^2\beta - m_{\PHiggslight}^2$, is non-zero. 
This leads to a minimum accesible \mA~value of \mh~at high \tanb, and
a minimum accessible \mA~of around $151\,\GeV$ for \tanb=1. In addition, the scenario
can be formulated, but is not strictly valid, for values of \tanb~upwards of ten \cite{CMS-PAS-HIG-16-007}.
The reason for this is that direct higher order SUSY corrections to down-type
fermion couplings and corrections due to SUSY particles in loops become relevant
above \tanb=10, but these are omitted in the hMSSM approach.

%to the couplings to the h, as expected from the SM, as will be further discussed in Section 3. This scenario is strictly valid for mA > 130 GeV and tanβ < 10. It can still be formulated for values up to tanβ < 60 though the omission of direct higher order SUSY corrections to down-type fermion couplings (also referred to as ∆β corrections) and corrections due to SUSY particles in loops, which be- come relevant for tan β > 10 question the consistency of the predictions with SUSY. A detailed

The production cross sections of the \PHiggs and \PHiggsps bosons are
qualitatively similar to those in other \ac{MSSM} scenarios. Gluon fusion is dominant
at low \tanb, with b-associated production being more important at high \tanb. 

The branching ratios of the \PHiggs and \PHiggsps bosons into $\tau\tau$ follow a similar
pattern as in other MSSM scenarios, with large branching ratios at high \tanb.
%\begin{figure}[h!]
%\begin{center}
%\subfloat[$\sigma(gg\rightarrow\PHiggsps)$]{\includegraphics[width=0.5\textwidth]{./Theory/Figures/xsggA_hmssm.pdf}}
%\subfloat[$\sigma(bb\rightarrow bb\PHiggs)$]{\includegraphics[width=0.5\textwidth]{./Theory/Figures/xsbbH4F_hmssm.pdf}}
%\end{center}
%\caption{Cross-sections at $\sqrt{s}=13$ TeV as a function of
%\mA~and \tanb~ in 
%the hMSSM scenario for (a) gluon fusion production of the \PHiggsps boson and (b)
%b-associated production (four--flavour scheme) of the \PHiggs boson. We can generally see the cross-sections increase
%with growing \tanb, apart from the gluon fusion cross section which decrease with increasing
%\tanb~for low \tanb.}
%\label{fig:hmssm_xs}
%\end{figure}
%\begin{figure}[h!]
%\begin{center}
%\subfloat[BR$(\PHiggs\rightarrow\Pgt\Pgt)$]{\includegraphics[width=0.5\textwidth]{./Theory/Figures/brHtautau_hmssm.pdf}}
%\subfloat[BR$(\PHiggsps\rightarrow\Pgt\Pgt)$]{\includegraphics[width=0.5\textwidth]{./Theory/Figures/brAtautau_hmssm.pdf}}
%\end{center}
%\caption{Branching ratios of (a) the \PHiggs boson and (b) the \PHiggsps boson into \tautau. The
%branching ratio is enhanced at high \tanb.} 
%\label{fig:hmssm_brtautau}
%\end{figure}

\section{Status of BSM Higgs boson searches}
\label{sec:theory_BSMH_status}
With the data collected by the \ac{LHC}
experiments up to the end of 2012 many searches for \ac{BSM} Higgs
bosons were performed. Figure \ref{fig:bsm_summary} shows the interpretations
of different searches performed with the \acs{CMS} detector during this period
in the $m_{\PHiggslight}^{\text{mod+}}$ (figure~\ref{fig:bsm_summary}a)
and hMSSM (figure~\ref{fig:bsm_summary}b) scenarios. The direct search for heavier Higgs bosons decaying into pairs
of tau leptons sets the most stringent limits at high \tanb, with searches for 
heavy Higgs bosons decaying to $\Pbottom\Pbottom$ and to $\mu\mu$ both excluding smaller parts of the high-\tanb~region.
Searches for heavy Higgs bosons decaying to $\PW\PW$ and $\PZ\PZ$ are able to exclude part of the low-\tanb~region. In the 
hMSSM scenario searches for \Htohh and \AtoZh can exclude a small area at low \tanb~and between \mA~$=250$--$350\,\GeV$.
The red exclusion contour in figure \ref{fig:bsm_summary}b is the reinterpretation of the analysis presented in
chapter \ref{chap:hhh}.

Since the restart of the \acs{LHC} in 2015 searches for \AHtotautau, setting more stringent limits than those shown in figure \ref{fig:bsm_summary},
have been performed with the \acs{CMS} detector. The results of these searches will be presented in chapter \ref{chap:mssm}. 
Similar searches have been performed by ATLAS \cite{ATLASMSSMtautau2016}. Neutral \ac{BSM} Higgs bosons
are being searched for in other channels too, with some results for decays into $\Ptop\Ptop$ \cite{ATLASHttbar}, 
$\PZ\PZ$ \cite{CMSHZZ2016,ATLASHZZ2016}, and $\PW\PW$ \cite{ATLASHeavyHWW} already public.
In addition, searches for charged
Higgs bosons \cite{ATLASHplustaunu,ATLASHplustb,CMSHplustaunu} and
di-Higgs searches \cite{ATLASHbbgamgam,ATLASHgamgamWW,ATLASHhhbbbb,CMSbbgamgam,CMSHbbtautau,CMSHbbWW}, both in 
various final states, have been performed. %As more data are collected during Run 2 and beyond, the \ac{BSM}
%Higgs sector will continue to be probed.

\begin{figure}[h!]
\begin{center}
\subfloat[$m_{\PHiggslight}^{\text{mod+}}$ scenario]{\includegraphics[width=0.5\textwidth]{./Theory/Figures/CMS-PAS-HIG-16-007_Figure_003-a.pdf}}
\subfloat[hMSSM scenario]{\includegraphics[width=0.5\textwidth]{./Theory/Figures/CMS-PAS-HIG-16-007_Figure_003-b.pdf}}
\end{center}
\caption[Summary of the interpretations of BSM Higgs boson searches using data up to the end of 2012 at CMS in the $m_{\PHiggslight}^{\text{mod+}}$ scenario and the hMSSM scenario.]{Summary of the interpretations of \ac{BSM} Higgs boson searches at \acs{CMS} using data
collected up to the end of 2012 in (a) the $m_{\PHiggslight}^{\text{mod+}}$ and (b) the
hMSSM scenario. The different coloured areas indicate the observed and expected exclusion from different searches in these
scenarios. The results from \ac{MSSM} Higgs boson searches with decays into tau leptons are shown in blue and exclude more of the parameter
space than any of the other searches. The \ac{MSSM} Higgs to $\Pbottom\Pbottom$ (cyan) and Higgs to $\mu\mu$ (yellow) searches are also sensitive in part
of the high-\tanb~region, with searches for Higgs to $\PW\PW$ or $\PZ\PZ$ (orange) providing exclusion power at low \tanb~and low mass. In the $m_{\PHiggslight}^{\text{mod+}}$ 
scenario the charged Higgs to $\Pgt\Pgn$ search (magenta) excludes the low mass region for all values of \tanb. Masses below around $300\,\GeV$ in the 
hMSSM scenario are excluded instead by constraints from \ac{SM} Higgs boson measurements (magenta). In this scenario small
areas of the low-\tanb~region are also excluded by the searches for $\PHiggs\rightarrow \PHiggslight\PHiggslight \rightarrow \Pbottom\Pbottom\tau\tau$ and $\PHiggsps\rightarrow \PZ\PHiggslight \rightarrow \ell\ell\tau\tau$ (red)
and the search for $\PHiggs\rightarrow \PHiggslight\PHiggslight \rightarrow \Pbottom\Pbottom\gamma\gamma$ \cite{CMS-PAS-HIG-16-007}.}
\label{fig:bsm_summary}
\end{figure}


