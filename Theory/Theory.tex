\chapter{Introduction and theory}
\label{chap:theory}

\section{The standard model of particle physics}
\label{sec:theory_sm}

\section{Standard model Higgs boson measurements}
\label{sec:theory_smH}

\section{Beyond the standard model}
\label{sec:theory_BSM}

\subsection{The Higgs sector of the MSSM}
\label{sec:theory_MSSM_H}
At least two Higgs doublets are required in the
Higgs sector of the MSSM due to the supersymmetric
background of the theory \cite{MSSM-carena-haber}. Two complex Higgs doublets are added,
\begin{equation}\label{eqn:mssm_higgsdoublets}
\phi_d = \begin{pmatrix} \phi_d^0 \\
\phi_d^- \end{pmatrix} \text{ and } \phi_u = \begin{pmatrix} \phi_u^+ \\
\phi_u^0 \end{pmatrix}
\end{equation}
Note that $\phi_d^0$ couples exclusively to down-type fermions, and $\phi_u^0$ couples
exclusively to up-type fermions.
This leads to potential terms in the MSSM Lagrangian of the form:
\begin{equation}\label{eqn:mssm_lagrangian_potential}
V = \mu(\phi_u^+\phi_d^- - \phi_u^0\phi_d^0),
\end{equation}
with $\mu$ a mass parameter, equivalent to the \ac{SM} Higgs mass parameter.
This provides electroweak symmetry breaking as in the \ac{SM} case and
minimising this potential gives,
\begin{equation}\label{eqn:mssm_minimpot}
\langle 0|\phi_d| 0 \rangle = \begin{pmatrix}v_d\\
0 \end{pmatrix} \text{ and} \langle 0 |\phi_d|\rangle 0 = \begin{pmatrix} 0\\
v_u \end{pmatrix}
\end{equation}
Using this we define
\begin{equation}\label{eqn:tanb_def}
\tan{\beta} \equiv \frac{v_u}{v_d}
\end{equation}
Of the eight degrees of freedom due to the introduction of two complex
doublets, three become longitudinal states of the $\PW^{\pm}$ and \PZ bosons.
The remaining 5 degrees of freedom lead to five physical Higgs bosons, a charged
Higgs boson pair $\PHiggs^{\pm}$, the neutral pseudoscalar \PHiggsps
and the neutral scalars h and H. These are defined in terms
of \tanb~and a mixing angle $\alpha$ which arises from diagonalising
the neutral scalar Higgs squared-mass matrix, see equation \ref{eqn:treelevel_mass}. The five bosons become:
\begin{align}\label{eqn:MSSMHiggsMixing}
&\PHiggs^+ = \phi_u^+\cos{\beta} + \phi_d^{-\dagger}\sin{\beta},\\
&\PHiggs^- = \phi_u^{+\dagger}\cos{\beta} + \phi_d^{-}\sin{\beta},\\
&\PHiggsps = \sqrt{2}(\text{Im}\phi_d^0\sin{\beta} + \text{Im}\phi_u^0\cos{\beta}),\\
&\PHiggslight = -(\sqrt{2}\text{Re}\phi_d^0 - v_d)\sin{\alpha} + (\sqrt{s}\text{Re}\phi+u^0 -v_u)\cos{\alpha}\\
&\PHiggs = (\sqrt{2}\text{Re}\phi_d^0-v_d)\cos{\alpha}+(\sqrt{2}\text{Re}\phi_u^0-v_u)\sin{\alpha}
\end{align}
As the self-interactions of the Higgs fields are not independent parameters
but are functions of the electroweak gauge coupling constants, at tree
level the MSSM Higgs sector parameters are determined by two free parameters, \tanb~and
one of the Higgs boson masses, conventionally chosen as \mA.
Using this the masses of the five Higgs states are found as:
\begin{equation}\label{mssm_chargedhigss_mass}
m_{\PHiggs^{\pm}}^2 = m_{\PHiggsps}^2+m_{\PW}^2,
\end{equation}
while the neutral scalar \PHiggslight and \PHiggs are eigenstates of the tree-level
squared-mass matrix:
\begin{equation}\label{eqn:treelevel_mass}
\mathcal{M}_{\text{tree}}^2 = \begin{pmatrix} 
m_{A}^2\sin{\beta}^2 + m_{Z}^2\cos{\beta}^2 & -(m_{A}^2+m_{Z}^2)\sin{\beta}\cos{\beta}\\
-(m_{A}^2+m_{Z}^2)\sin{\beta}\cos{\beta} & m_{A}^2\cos{\beta}^2+m_{Z}^2\sin{\beta}^2 \end{pmatrix}.
\end{equation}
This leads to the masses of \PHiggslight and \PHiggs as eigenvalue of $\mathcal{M}_{\text{tree}}^2$,
\begin{equation}\label{eqn:scalarmass}
m_{\PHiggs,\PHiggslight}^2 = \frac{1}{2}(m_{A}^2+m_{Z}^2 \pm \sqrt{(m_A^2+m_Z^2)^2-4m_Z^2m_A^2\cos{2\beta}^2}).
\end{equation}
A consequence of this is a tree-level upper bound to the mass of the light Higgs
boson,
\begin{equation}\label{eqn:mh_upper}
m_{h} \leq m_{Z},
\end{equation}
and a constraing on $\alpha$,
\begin{equation}\label{eqn:alpha_constraint}
\cos{(\beta-\alpha)}^2 = \frac{m_h^2(m_Z^2-m_h^2)}{m_A^2(m_H^2-m_h^2)}.
\end{equation}

The constraint on $m_h$ being less than the mass of the Z boson, 91.2 GeV, seems 
problematic at first, as we know there is a Higgs state with a mass of 125 GeV. However,
the tree-level masses and couplings of the Higgs bosons in the MSSM can be
significantly altered by radiative corrections. The dominat effect is from
incomplete cancellation of the stop- and top-loop corrections, which
do not completely cancel as SUSY is a broken symmetry. FIND REFERENCE FOR THIS

The couplings of the neutral Higgs bosons to bosons and fermions 
are modified with respect to the \ac{SM} couplings, by factors
described in table \ref{tab:mssm_couplings} \cite{YR4}.

\begin{table}[htp]
\label{tab:mssm_couplings}
\begin{center}
\caption{Couplings in the MSSM, as multiplicative factors with
respect to the \ac{SM} couplings}
\begin{tabular}{p{2cm}p{4cm}p{4cm}p{4cm}}
\toprule
Particle & Coupling to bosons & Coupling to up-type quarks & Coupling to down-type quarks and leptons \\
\midrule
\PHiggsps & 0 & $\cot{\beta}$ & $ \tan{\beta}$\\
\PHiggs & $\cos{\beta-\alpha}$ & $\frac{\sin{\alpha}}{\sin{\beta}}$ & $\frac{\cos{\alpha}}{\cos{\beta}}$\\
\PHiggslight & $\sin{\beta-\alpha}$ & $\frac{\cos{\alpha}}{\sin{\beta}}$ & $-\frac{\sin{\alpha}}{\cos{\beta}}$\\
\bottomrule
\end{tabular}
\label{tab:mssm_couplings}
\end{center}
\end{table}

In the \textit{decoupling limit}, when \mA$>>m_{Z}$, the mixing
angle $\alpha \approx \beta - \pi/2$. This reduces the couplings in table
\ref{tab:mssm_couplings} to the \ac{SM} couplings for the \PHiggslight boson, 
with negligible couplings to bosons for the two remaining neutral
Higgs bosons, couplings to down-type quarks and leptons enhanced by a factor \tanb,
and couplings to up-type quarks enhanced by a factor $\frac{1}{\tan{\beta}}$. This motivates
the choice of decay into $\Pgt\Pgt$ as search channel for the neutral Higgs bosons in the MSSM.
Figure \ref{fig:mssm_brtautau} shows the branching ratios of the \PHiggs boson 
in the $m_{h}^{\text{mod+}}$ scenario
which will be discussed in more detail in section \ref{sec:theory_BSM_models_mhmodp}. 
The branching ratio into di-$\Pgt$ pairs is large for \tanb=30. The branching ratios 
into $\Pgt\Pgt$ of \PHiggsps are of similar size.
%Using trigon identities: cos alpha/cos beta = cos (beta-pi/2)/cos beta = cosbeta*cospi/2 + sin(beta)*sin(pi/2) over cos beta = sin beta /cos beta = tan beta
%sin alpha = sin (beta-pi/2) = sinbeta cos pi/2 - cosbetasinpi/2 = -cosbeta 
%such that -sinalpha/cosbeta = 1
%cos alpha/sin beta = sin beta/sin beta = 1
%sin beta-alpha = sin pi/2 = 1
%cos beta-alpha = cos pi/2 = 0
%sin alpha/sin beta = -cos beta/sin beta=-1/tan(beta)

\begin{figure}[h!]
\begin{center}
\subfloat[\tanb=5]{\includegraphics[width=0.5\textwidth]{./Theory/Figures/YR4HXS_BRSummary_H_mhmodp_tanbeta5_FeynHiggs_HDecay.pdf}}
\subfloat[\tanb=30]{\includegraphics[width=0.5\textwidth]{./Theory/Figures/YR4HXS_BRSummary_H_mhmodp_tanbeta30_FeynHiggs_HDecay.pdf}}
\end{center}
\caption{Branching ratios of the \PHiggs boson in the $m_{h}^{\text{mod+}}$ scenario,
at (a) \tanb=5~and (b) \tanb=30. The branching ratio into \Pgt\Pgt (blue), $b\bar{b}$ (red) 
and \Pgm\Pgm (yellow) is enhanced at high \tanb, while the branching ratio into 
WW (grey) and \ttbar (green) is reduced \cite{MSSM-xswg-twiki}.}
\label{fig:mssm_brtautau}
\end{figure}

The dominant neutral MSSM Higgs boson production processes at hadron colliders
are slightly different from the \ac{SM} Higgs boson production processes. 
Because the pseudoscalar A does not couple to the
vector bosons, and in the decoupling limit the coupling of \PHiggs to vector
bosons is suppressed with respect to the \ac{SM} expectation, VBF and PW/PZ associated
production do not constitute dominant production processes. Gluon fusion production is 
the dominant production mode at low \tanb, like in the \ac{SM}. An example tree-level Feynman diagram is given in figure \ref{fig:production_mssm}a. In gluon fusion production the Higgs boson
is produced via a quark loop. At low \tanb~values this loop is dominated by top quarks, while at 
high \tanb~the b-quark loop dominates. This is a result of the enhanced coupling to down-type fermions at high \tanb. Another consequence of these enhanced couplings to down-type fermions at high \tanb~is the dominance of b-associated production, where the Higgs boson is produced through
bottom quark fusion. The tree-level Feynman diagram for this process is shown in figure \ref{fig:production_mssm}b.

\begin{figure}[h!]
\begin{center}
\subfloat[gluon fusion production]{\includegraphics[width=0.5\textwidth]{./Theory/Figures/CMS-PAS-HIG-16-037_Figure_001-a.pdf}}
\subfloat[b-associated production]{\includegraphics[width=0.5\textwidth]{./Theory/Figures/CMS-PAS-HIG-16-037_Figure_001-b.pdf}}
\end{center}
\caption{Tree-level Feynman diagrams of (a) gluon fusion production and (b) b-associated
production of neutral Higgs boson in the MSSM \cite{CMS-PAS-HIG-16-037}.}
\label{fig:production_mssm}
\end{figure}

\subsection{\acl{2HDM}s}
\label{sec:theory_2HDM}
A different approach, also leading to an extended Higgs sector with
respect to the \ac{SM} case, is found in the \ac{2HDM} \cite{2HDM-I}.
As the name suggests \ac{2HDM}s are a generic class of models in which
there is not one, but two Higgs doublets. The addition of a second
Higgs doublet can also motivated by \ac{SUSY}, but also by axion models.
An important point to note is that despite the presence of a second 
Higgs doublet, there are no explicit \ac{SUSY} particles in the \ac{2HDM}.
There are many different types of \ac{2HDM}, all distinguished by the couplings
of the different Higgs bosons to bosons and fermions. The most studied of the
\ac{2HDM}s, the type-II \ac{2HDM}, has a structure very similar to 
the MSSM Higgs sector. The couplings of the neutral Higgs bosons to 
vector bosons and fermions are the same as the \ac{MSSM} couplings from table \ref{tab:mssm_couplings}.

The differences between the type II \ac{2HDM} and the \ac{MSSM}
are that $\alpha$ is a free parameter in the \ac{2HDM}, while
in the \ac{MSSM} it is not. In addition the masses of the Higgs bosons
are free parameters. In the \textit{alignment} limit, where $\cos{(\beta-\alpha)}$
approaches 0, the couplings are exactly \ac{SM}-like.


\section{MSSM benchmark scenarios}
\label{sec:theory_BSM_models}
Because the MSSM contains a large number of
SUSY-breaking parameters that affect the Higgs
sector, it us usual to define benchmark scenarios 
in which the only free parameters are \mA~and \tanb.
In these scenarios the SUSY parameters entering in 
the radiative corrections are fixed to define the scenario.

The parameters that need to be fixed in the benchmark scenarios are:
\begin{itemize}
\item The mass of the third generation squarks, given by $M_{\text{SUSY}}$.
\item The higgsino mass parameter, $\mu$.
\item The mass of the stau, the third generation sleptons, $M_{\tilde{\ell_3}}$.
\item The mass of the gluino, $M_{\tilde{g}}$.
\item The U(1) gaugino mass parameter, $M_1$.
\item The SU(2) gaugino mass parameter, $M_2$.
\item The trilinear couplings of the stops, sbottoms and staus to the Higgs: $A_t$, $A_b$ and $A_{\Pgt}$.
\item The stop, sbottom and stau mixing parameters, $X_t$, $X_b$ and $X_{\Pgt}$.
\end{itemize}

Some of these parameters can be expressed in terms of relations to other parameters. 
$X_t$, $X_b$ and $X_{\Pgt}$ can be expressed as,
\begin{equation}\label{eqn:trilinear_couplings}
\begin{split}
&X_t = A_t-\mu\cot{\beta},\\
&X_b = A_b-\mu\tan{\beta},\\
&X_{\Pgt} = A_{\Pgt} - \mu\tan{\beta}.
\end{split}
\end{equation}
In addition, $M_1$ is fixed via the unification
relation,
\begin{equation}
M_1 = \frac{5}{3}M_s\tan{\theta_w}^2,
\end{equation}
where $\theta_w$ is defined by $\cos{\theta_w} = \frac{m_W}{m_Z}$.

Some additional parameters which have only a
small effect on the MSSM Higgs boson sector are alos considered.
Because their effect is so small they are
fixed in all benchmark scenarios at values compatible with 
exclusion limits from direct searches:
\begin{itemize}
\item Masses of the first and second generation squarks, $M_{\tilde{q}_{1,2}} = 1.5$ TeV.
\item Masses of the first and second generation sleptons, $M_{\tilde{\ell}_{1,2}} = 500$ GeV.
\item Trilinear couplings of first and second generation squarks and sleptons, $A_f=0$.
\end{itemize}

After the discovery of the 125 GeV Higgs boson the MSSM benchmark
scenarios need to accommodate this particle, which means only the 
scenarios which contain a light Higgs boson with a mass of around 125 GeV
are still accessible.

% in the mmax scenario the benchmark values have been chosen h
%such that the mass of the light CP-even Higgs boson is maximized for fixed tanβ and large
%MA (the scale of the soft SUSY-breaking masses in the stop and sbottom sectors, which
%sets the mass scale for the corresponding supersymmetric particles, has been fixed to 1 TeV
%in this scenario). This scenario is useful to obtain conservative bounds on tanβ for fixed
%values of the top-quark mass 

\subsection{The $m_{h}^{\text{mod+}}$ scenario}
\label{sec:theory_BSM_models_mhmodp}
The $m_{h}^{\text{mod+}}$ scenario \cite{MSSM-benchmark-scenarios}
is a modification of the $m_h^{\text{max}}$ scenario \cite{MSSM-mhmax}. This scenario, which
was used for interpretations of MSSM Higgs boson searches at LEP, allows
the mass of the light Higgs boson to reach the highest a-priori expected 
value of around 135 GeV for high \mA. There is only a small area of the \mA-\tanb~plane
in this scenario where the mass of the light Higgs boson is compatible with the
observed 125 GeV state. The modifications to the parameters of the $m_{h}^{\text{max}}$ 
scenario address this issue. In the $m_{h}^{\text{mod+}}$ scenario $M_{\text{SUSY}}$ is chosen
to be 1 TeV. The stop mixing parameter is positive, $X_t= 1.5 M_{\text{SUSY}}$.
%This gives better agreement with muon g-2 results while the mhmod- scenario
%negative stop mixing (-1.9 M_SUSY) results in better agreements with B(b->sgamma) 
%meausrements 
The remaining parameters are set as $\mu=200$ GeV, $m_{\tilde{g}} = 1.5$ TeV,
$m_{\tilde{\ell}_3} = 1$ TeV, and $A_b=A_t=A_{\Pgt}$. Figure \ref{fig:mhmodp_mh}
shows the mass of the light Higgs boson in the $m_{h}^{\text{mod+}}$ scenario, showing
that its mass is compatible with 125 GeV over a large part of the parameter space.
\begin{figure}[h!]
\begin{center}
\includegraphics[width=0.5\textwidth]{./Theory/Figures/mh_mhmodp.pdf}
\end{center}
\caption{The mass of the light Higgs boson, \mh, as a function 
of \mA~and \tanb~in the $m_{h}^{\text{mod+}}$ scenario. The white areas
indicate masses lower than 122 GeV. The figure shows that
the mass of the light Higgs boson is compatible with 125 GeV over a large
part of the \mA-\tanb~plane.}
\label{fig:mhmodp_mh}
\end{figure}

Figure \ref{fig:mhmodp_xs} shows the cross--sections of
gluon fusion production of the \PHiggs boson and (four--flavour scheme) b-associated
production of the \PHiggsps boson in the $m_h^{\text{mod+}}$ scenario. Gluon fusion 
production dominates at low \tanb, while b-associated production has a larger cross--section
at high \tanb.

\begin{figure}[h!]
\begin{center}
\subfloat[$\sigma(gg\rightarrow \PHiggs)$]{\includegraphics[width=0.5\textwidth]{./Theory/Figures/xs_ggH_mhmodp.pdf}}
\subfloat[$\sigma(gg\rightarrow bb\PHiggsps)$]{\includegraphics[width=0.5\textwidth]{./Theory/Figures/xs_bbA4FS_mhmodp.pdf}}
\end{center}
\caption{Production cross--section of (a) gluon fusion production of the \PHiggs boson and
(b) b--associated production (four--flavour scheme) of the \PHiggsps boson in the $m_{h}^{\text{mod+}}$ scenario. The gluon fusion production cross section is larger at low \tanb, while
the b-associated production cross-section is larger at high \tanb.}
\label{fig:mhmodp_xs}
\end{figure}

Figure \ref{fig:mhmodp_br} shows the branching ratio of the \PHiggs and \PHiggsps 
into $\tau\tau$. The branching ratios into $\tau\tau$ are enhanced, especially at
high \tanb, showing how this decay channel is useful for MSSM Higgs boson searches.

\begin{figure}[h!]
\begin{center}
\subfloat[BR($\PHiggs \rightarrow \tau\tau$)]{\includegraphics[width=0.5\textwidth]{./Theory/Figures/brHtautau_mhmodp.pdf}}
\subfloat[BR($\PHiggsps \rightarrow \tau\tau$)]{\includegraphics[width=0.5\textwidth]{./Theory/Figures/brAtautau_mhmodp.pdf}}
\end{center}
\caption{Branching ratio of (a) \PHiggs and (b) \PHiggsps into $\tau\tau$. The branching
ratios into $\tau\tau$ are enhanced at high \tanb.}
\label{fig:mhmodp_br}
\end{figure}


\subsection{MSSM scenarios at low \tanb}
A large area of the \mA-\tanb~plane in the $m_{h}^{\text{mod+}}$ 
scenario contains a light Higgs boson with a mass compatible with
125 GeV. However, at low values of \tanb this is not the case, with
values of \mh~dropping below 122 GeV. The low \tanb~regime can be re-opened
if $M_{\text{SUSY}}$ is allowed to be greater than 3 TeV \cite{MSSM-reopen}. 
MORE ABOUT FINE-TUNING
%But not too large, see ref above!
Figure \ref{fig:tanb_accessibility} shows contours of constant
light Higgs boson mass as a function of \tanb~and $M_{\text{SUSY}}$.
It indicates that values of \mh~can be compatible with 125$\pm$3 GeV
for \tanb~values down to 1 if $M_{\text{SUSY}}$ is around 100-1000 TeV. 
\tanb~values of 3-5 are even accessible for $M_{\text{SUSY}}$ of around 10 TeV.

\begin{figure}[h!]
\begin{center}
\includegraphics[width=0.5\textwidth]{./Theory/Figures/tanb_accessibility.png}
\end{center}
\caption{Contours of constant light Higgs boson mass as a function of \tanb~and $M_{\text{SUSY}}$.
High values of $M_{\text{SUSY}}$ allow for light Higgs boson masses compatible with
125 GeV down to low values of \tanb~\cite{hMSSM-2}.}
\label{fig:tanb_accessibility}
\end{figure}

%where the ±3 GeV variation corresponds to a rough estimate of the theoretical uncertainty of the MSSM prediction for mh, due to the unknown effect of higher-order correction

Two approches for the definition of an MSSM scenario that gives
access to the low-\tanb~region have been developed, they will be
discussed in the next sections.

\subsubsection{The low-\tanb-scenario}
\label{sec:theory_BSM_model_lowtb}
In the low-\tanb-scenario \cite{Hein-low-tb-high,MSSM-lowtanb}, the SUSY parameters entering
the radiative corrections are tuned to obtain light 
Higgs boson masses of around 125 GeV in most of the \mA-\tanb~plane considered.
Figure \ref{fig:lowtbhigh_mh} shows that, apart from in a corner of \tanb~1-4 and 
\mA~150-250 GeV, \mh~is compatible with 125 GeV.

\begin{figure}[h!]
\begin{center}
\includegraphics[width=0.5\textwidth]{./Theory/Figures/mh_lowtbhigh.png}
\end{center}
\caption{Mass of the light Higgs boson in the \mA-\tanb~plane of the low-\tanb~scenario.
The mass is compatible with 125 GeV nearly everywhere \cite{MSSM-lowtanb}.}
\label{fig:lowtbhigh_mh}
\end{figure}

To obtain \mh$\approx$125 GeV over a large part of the parameter
space, the SUSY parameters are chosen such that SUSY BREAKING MASSES ETC?
and $M_{\text{SUSY}}$ is not fixed but varies between a few TeV and 100 TeV, while
varying the parameter $X_t$ as:
\begin{equation}
\begin{split}
\tan{\beta} \leq 2 &: \frac{X_t}{M_{\text{SUSY}}} = 2\\
2 < \tan{\beta} \leq 8.6 &: \frac{X_t}{M_{\text{SUSY}}} = 0.0375\text{tan}^2\beta - 0.7\tan{\beta} + 3.25\\
8.6 < \tan{\beta} &: \frac{X_t}{M_{\text{SUSY}}} = 0.
\end{split}
\end{equation}
The other trilinear couplings are set to 2 TeV, with $\mu$ set to 1.5 TeV and $M_2$ to 2 TeV.

The branching fractions of \Htohh and \AtoZh in the low-\tanb~scenario are shown in
figure \ref{fig:lowtbhigh_br}. For both decay channels there are areas in the \mA-\tanb~plane
where the branching ratio is enhanced, indicating how analyses targeting such processes 
can be sensitive in this scenario. %REPHRASE
The analysis presented in chapter \ref{chap:hhh} is interpreted
in the low-\tanb~scenario.

\begin{figure}[h!]
\begin{center}
\subfloat[\Htohh]{\includegraphics[width=0.5\textwidth]{./Theory/Figures/lowtbhigh_hhbr.png}}
\subfloat[\AtoZh]{\includegraphics[width=0.5\textwidth]{./Theory/Figures/lowtbhigh_azhbr.png}}
\end{center}
\caption{Branching fractions of (a) \Htohh and (b) \AtoZh in the low-\tanb~scenario, indicating 
areas where both are significantly enhanced \cite{MSSM-lowtanb}.}
\label{fig:lowtbhigh_br}.
\end{figure}

\subsubsection{The hMSSM scenario}
\label{sec:theory_BSM_models_hMSSM}
The hMSSM scenario \cite{hMSSM-1,hMSSM-2} uses a different approach by starting from 
the REPHRASE tree-level expressions for masses and mixing and
the mass of the light Higgs boson \mh=125 GeV.

One of the assumptions of the hMSSM is that the
the mass matrix for the neutral CP-even states can
be decomposed as,
\begin{equation}
\label{eqn:hmssm_massmatrix}
\mathcal{M}^2_{\phi} = \mathcal{M}^2_{\text{tree}} + \begin{pmatrix}
\Delta\mathcal{M}^2_{11} & \Delta\mathcal{M}^2_{12} \\
\Delta\mathcal{M}^2_{12} & \Delta\mathcal{M}^2_{22} \end{pmatrix},
\end{equation}
where the $\Delta\mathcal{M}^2_{ij}$ are the radiative corrections.
The second assumption is that only $\Delta\mathcal{M}^2_{22}$ needs to be
taken into account, as this is the element that involves the stop-top correction
and so $\Delta\mathcal{M}^2_{22} >> \Delta\mathcal{M}^2_{12},\Delta\mathcal{M}^2_{11}$. 
Finally, all SUSY particles are assumed to be heavy enough not to be
detected at the \acs{LHC} and apart from effects on the mass matrix, 
the effects on the Higgs sector can be neglected.
%all SUSY particles are heavy enough to escape detection at the LHC, and their effects on the Higgs sector other than those on the mass matrix, e.g. via direct loop corrections to the Higgs-boson couplings or via modifications of the total decay widths, can be neglected.

Using these assumptions we can invert the lightest eigenvalue
of the mass matrix to get: 
\begin{equation}
\label{eqn:hmssm_deltam22}
\Delta\mathcal{M}^2_{22} = \frac{m_h^2(m_A^2+m_Z^2 - m_h^2) - m_A^2m_Z^2\text{cos}^22\beta}{m_Z^2\text{cos}^2\beta + m_A^2\text{sin}^2\beta - m_h^2},
\end{equation}

This means we can write,
\begin{equation}
\label{eqn:hmssm_mHalpha}
\begin{split}
&m^2_H = \frac{(m_A^2+m_Z^2-m_h^2)(m_z^2\cos{\beta}^2+m_A^2\sin{\beta}^2 - m_A^2m_Z^2\cos{2\beta}^2}{m_Z^2\cos{\beta}^2+m_A^2\sin{\beta}^2-m_h^2},\\
&\tan{\alpha} = -\frac{(m_Z^2+m_A^2)\cos{\beta}\sin{\beta}}{m_Z^2\cos{\beta}^2+m_A^2\sin{\beta}^2-m_h^2}.
\end{split}
\end{equation}
Combining this with the Hhh coupling,
\begin{equation}
\label{eqn:hmssm_Hhh}
\lambda_{Hhh} = \lambda_{Hhh,tree} + 3\frac{\Delta\mathcal{M}^2_{22}\sin{\alpha}}{m_Z^2\sin{\beta}}\cos{\alpha}^2,
\end{equation}
gives enough information to determine the cross sections and branching
ratios of all the five Higgs bosons as a function of \mA~and \tanb. The scenario
is only well defined in regions where the denominator in equations
\ref{eqn:hmssm_deltam22} and \ref{eqn:hmssm_mHalpha}, $m_Z^2\cos{\beta}^2+m_A^2\sin{\beta}^2 - m_h^2$, is nonzero. 
This leads to a minimum accesible \mA~value of \mh~at high \tanb, and
a minimum accessible \mA~of around 151 GeV for \tanb=1. In addition, the scenario
can be formulated, but is not strictly valid, for values of \tanb upwards of 10 \cite{CMS-PAS-HIG-16-007}.
The reason for this is that direct higher order SUSY corrections to down-type
fermion couplings and corrections due to SUSY particles in loops become relevant
above \tanb=10, but these are omitted in the hMSSM approach.

%to the couplings to the h, as expected from the SM, as will be further discussed in Section 3. This scenario is strictly valid for mA > 130 GeV and tanβ < 10. It can still be formulated for values up to tanβ < 60 though the omission of direct higher order SUSY corrections to down-type fermion couplings (also referred to as ∆β corrections) and corrections due to SUSY particles in loops, which be- come relevant for tan β > 10 question the consistency of the predictions with SUSY. A detailed

The production cross--sections at $\sqrt{s}=14$ TeV for gluon fusion production of the
H and A bosons, as well as for b-associated production, are shown in figure \ref{fig:hmssm_xs}.
As expected, gluon fusion production is dominant at low \tanb,with b-associated
production being more important at high \tanb. The cross sections increase for increasing \tanb,
apart from the gluon fusion cross section at low \tanb, where it decreases as \tanb~increases
due to a negative top-bottom intereference effect.

\begin{figure}[h!]
\begin{center}
\subfloat[$\sigma(gg\rightarrow\PHiggsps)$]{\includegraphics[width=0.5\textwidth]{./Theory/Figures/xsggA_hmssm.pdf}}
\subfloat[$\sigma(bb\rightarrow bb\PHiggs)$]{\includegraphics[width=0.5\textwidth]{./Theory/Figures/xsbbH4F_hmssm.pdf}}
\end{center}
\caption{Cross-sections at $\sqrt{s}=13$ TeV as a function of
\mA~and \tanb~ in 
the hMSSM scenario for (a) gluon fusion production of the \PHiggsps boson and (b)
b-associated production (four--flavour scheme) of the \PHiggs boson. We can generally see the cross-sections increase
with growing \tanb, apart from the gluon fusion cross section which decrease with increasing
\tanb~for low \tanb.}
\label{fig:hmssm_xs}
\end{figure}

Figure \ref{fig:hmssm_brtautau} shows the branching ratios of \PHiggsps and \PHiggs into \Pgt\Pgt.
The branching ratios are large over a large part of the mA-\tanb~plane, but mostly so
at high \tanb.

\begin{figure}[h!]
\begin{center}
\subfloat[BR$(\PHiggs\rightarrow\Pgt\Pgt)$]{\includegraphics[width=0.5\textwidth]{./Theory/Figures/brHtautau_hmssm.pdf}}
\subfloat[BR$(\PHiggsps\rightarrow\Pgt\Pgt)$]{\includegraphics[width=0.5\textwidth]{./Theory/Figures/brAtautau_hmssm.pdf}}
\end{center}
\caption{Branching ratios of (a) the \PHiggs boson and (b) the \PHiggsps boson into \tautau. The
branching ratio is enhanced at high \tanb.} 
\label{fig:hmssm_brtautau}
\end{figure}


\section{Status of BSM Higgs boson searches}
\label{sec:theory_BSMH_status}
With the data collected during 
Run 1 of the \ac{LHC} many searches for BSM Higgs
bosons were already performed. The state of play with the full
Run-1 dataset collected by \ac{CMS} can be seen in figure \ref{fig:bsm_summary},
which shows the interpretations of different searches in the $m_{h}^{\text{mod+}}$ (figure \ref{fig:bsm_summary}a)
and hMSSM scenarios (figure \ref{fig:bsm_summary}b). The direct search for heavier Higgs bosons decaying into pairs
of tau leptons (shown in dark blue) sets the most stringent limits at high \tanb,with searches for 
heavy Higgs bosons decaying to $b\bar{b}$ and to $\mu\mu$ both excluding smaller parts of the high \tanb-region.
Searches for heavy Higgs bosons decaying to WW and ZZ are able to exclude part of the low-\tanb-region. In the 
hMSSM scenario searches for \Htohh and \AtoZh can exclude a small area at low \tanb~and between \mA=250-350 GeV.
The red exclusion contour in figure \ref{fig:bsm_summary}b is the re-interpretation of the analysis presented in
chapter \ref{chap:hhh}.

In Run-2 of the \ac{LHC} searches for \AHtotautau, setting more stringent limits than those shown in \ref{fig:bsm_summary},
have been performed in \ac{CMS}. The results of these searches will be presented in chapter \ref{chap:mssm}. Other searches
for heavy Higgs bosons with decays to other final states are also in progress as the Run-2 data keep
amassing. 

\begin{figure}[h!]
\begin{center}
\subfloat[$m_{h}^{\text{mod+}}$ scenario]{\includegraphics[width=0.5\textwidth]{./Theory/Figures/CMS-PAS-HIG-16-007_Figure_003-a.pdf}}
\subfloat[hMSSM scenario]{\includegraphics[width=0.5\textwidth]{./Theory/Figures/CMS-PAS-HIG-16-007_Figure_003-b.pdf}}
\end{center}
\caption{Summary of the interpretations of Run 1 BSM Higgs searches at \ac{CMS} in (a) the $m_{h}^{\text{mod+}}$ and (b) the
hMSSM scenario. The different coloured areas indicate the observed and expected exclusion from different searches in these
scenarios. The results from MSSM Higgs searches with decays into $\tau$ leptons are shown in blue and exclude more of the parameter
space than any of the other searches. The MSSM Higgs to $b\bar{b}$ (cyan) and Higgs to $\mu\mu$ (yellow) searches are also sensitive in part
of the high-\tanb region, with Higgs to WW or ZZ (orange) providing exclusion power at low \tanb~and low mass. In the $m_{h}^{\text{mod+}}$ 
scenario the charged Higgs to $\Pgt\Pgn$ search (magenta) excludes the low-mass region for all values of \tanb. Masses below around 300 GeV in the 
hMSSM scenario are excluded instead by constraints from standard model Higgs boson measurements (magenta). In this scenario small
areas of the low-\tanb~region are also excluded by the search for $H\rightarrow hh \rightarrow bb\tau\tau$ and $A\rightarrow Zh \rightarrow \ell\ell\tau\tau$ (red)
and the search for $H\rightarrow hh \rightarrow bb\gamma\gamma$ \cite{CMS-PAS-HIG-16-007}.}
\label{fig:bsm_summary}
\end{figure}


