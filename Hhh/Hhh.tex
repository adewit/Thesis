\chapter{\texorpdfstring{Search for \hhhtautaubb}{Search for H -> hh -> tautaubb}}
\label{sec:hhh}

In this chapter the search for a heavy Higgs boson decaying to a pair of 125 GeV Higgs bosons, with one of these Higgs bosons decaying to 
a pair of b-jets and the other decaying to a pair of $\tau$ leptons is discussed. The results of this search are model-independent
upper limits on heavy Higgs production cross--section times branching ratio into h(125)h(125)$\rightarrow b\bar{b} \tau\tau$. In addition 
the results are interpreted in the context of the MSSM and a type-II Two Higgs Doublet Model (2HDM), these interpretations are made
in combination with the results of a search for \azhlltautau. This search will be summarised in this
chapter, but will not constitute the main focus.

The discovery of the 125 GeV Higgs boson by the ATLAS and CMS collaborations in 2012 (CITATION) has opened up
new possibilities for probing the Higgs sector beyond the Standard Model. In some MSSM scenarios, and some more
generic type-II 2HDM's, a heavy neutral Higgs boson H can decay to a pair of 125 GeV Higgs bosons for low values
of \tanb, probing a region of phase space not yet excluded by stringent existing limits. In regions where
the decay \hhh is enhanced, the \azh decay also has a large branching ratio, indicating the usefulness
of both channels for probing the low-\tanb region. The $b\bar{b}$\tautau final state is chosen for the combination
of the large $h(125) \rightarrow b\bar{b}$ branching ratio and the cleaner \htautau final state.

BLAH BLAH BLAH

\section{Datasets and Monte Carlo samples}
\label{sec:hhh_datasets}
The dataset used for this analysis corresponds to the full dataset collected by the CMS experiment during the 2012 $p-p$ 
running period of the LHC. 
Signal and background events were generated using several different MC event generators. The \texttt{MadGraph}
CITATION matrix element generator was used to generate samples of \wjets, \zellell, \ttbar and $ZZ$,$ZW$ and $WW$
events. In addition to samples with a mixture of jet multiplicities ('inclusive' samples), samples binned in jet multiplicity
were used for the \wjets and \zellell backgrounds. This increases the number of background events
in signal regions with multiple jets. The samples binned in jet multiplicity are combined with the
inclusive samples such that the fraction of events with each jet multiplicity is preserved.

Single top samples were produced with the \texttt{POWHEG} CITATION generator. Samples of $gg\rightarrow$\hhhtautaubb
were generated in steps of 10 GeV between $m_H = 260 - 350$~GeV using \texttt{PYTHIA 6} CITATION. In all of the samples
\texttt{TAUOLA} CITATION is used to decay $\tau$s, and parton showering and hadronisation are modelled using \texttt{PYTHIA 6}.
Minimum bias events generated using \texttt{PYTHIA 6} are added to all MC samples to model additional
interactions. 
\vfill

\section{Event selection and categorisation}
\label{sec:hhh_selection}

\section{Data to Monte Carlo correction factors}
\label{sec:hhh_datamc}

\section{Discriminating variable}
\label{sec:hhh_discr}

\section{Background estimation}
\label{sec:hhh_backgrounds}

\section{Systematic uncertainties}
\label{sec:hhh_uncs}

\section{\texorpdfstring{Overview of \azhlltautau}{Overview of A->Zh->lltautau}}
\label{sec:hhh_azh}

\section{Results}
\label{sec:hhh_results}

\subsection{Signal extraction}
\label{sec:hhh_results_extraction}

\subsection{Model-independent results}
\label{sec:hhh_results_modelindep}


\subsection{Model-dependent results}
\label{sec:hhh_results_modeldep}


