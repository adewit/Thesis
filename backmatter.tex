
%% You're recommended to use the eprint-aware biblio styles which
%% can be obtained from e.g. www.arxiv.org. The file mythesis.bib
%% is derived from the source using the SPIRES Bibtex service.
\bibliographystyle{lucas_unsrt}
\bibliography{references}

%% I prefer to put these tables here rather than making the
%% front matter seemingly interminable. No-one cares, anyway!
%\listoffigures
%\listoftables

\chapter*{List of Acronyms}
\addcontentsline{toc}{chapter}{List of Acronyms}
\begin{acronym}[CERN]
\acro{MC}[MC]{Monte Carlo}
\acro{CMS}[CMS]{Compact Muon Solenoid}
\acro{LHC}[LHC]{Large Hadron Colider}
\acro{CERN}[CERN]{the European Organisation for Nuclear Research }
\acro{PS}[PS]{Proton Synchrotron}
\acro{SPS}[SPS]{Super Proton Synchrotron}
\acro{PSB}[PSB]{Proton Synchrotron Booster}
\acro{LS1}[LS1]{Long Shutdown 1}
\acro{MSSM}[MSSM]{Minimal Supersymmetric standard model}
\acro{SUSY}[SUSY]{Supersymmetry}
\acro{2HDM}[2HDM]{Two Higgs Doublet Model}
\acro{ECAL}[ECAL]{Electromagnetic calorimeter}
\acro{HCAL}[HCAL]{Hadron calorimeter}
\acro{TIB}[TIB]{Tracker Inner Barrel}
\acro{TID}[TID]{Tracker Inner Disk}
\acro{TOB}[TOB]{Tracker Outer Barrel}
\acro{TEC}[TEC]{Tracker EndCap}
\acro{EB}[EB]{ECAL Barrel}
\acro{EE}[EE]{ECAL Endcaps}
\acro{HB}[HB]{Hadron Barrel Calorimeter}
\acro{HO}[HO]{Hadron Outer Calorimeter}
\acro{HE}[HE]{Hadron Endcap Calorimeter}
\acro{HF}[HF]{Hadron Forward Calorimeter}
\acro{BSM}[BSM]{Beyond the Standard Model}
\acro{L1}[L1]{Level-1}
\acro{HLT}[HLT]{High-level trigger}
\acro{CSCs}[CSCs]{Cathode Strip Chambers}
\acro{DT}[DT]{Drift Tube}
\acro{RPCs}[RPCs]{Resistive Plate Chambers}
\acro{DAQ}[DAQ]{Data Acquisition}
\acro{WLCG}[WLCG]{Worldwide LHC Computing Grid}
\acro{RF}[RF]{radio frequency}
\acro{PF}[PF]{Particle Flow}
\acro{BDTs}[BDTs]{Boosted Decision Trees}
\acro{CTF}[CTF]{Combinatorial Track Finder}
\acro{GSF}[GSF]{Gaussian Sum Filter}
%\acro{KF}[KF]{Kalman Filter}
\acro{DA}[DA]{Deterministic Annealing}
\acro{CHS}[CHS]{Charged hadron subtraction}
\end{acronym}


%% If you have time and interest to generate a (decent) index,
%% then you've clearly spent more time on the write-up than the 
%% research ;-)
%\printindex
