
%% You're recommended to use the eprint-aware biblio styles which
%% can be obtained from e.g. www.arxiv.org. The file mythesis.bib
%% is derived from the source using the SPIRES Bibtex service.
\bibliographystyle{lucas_unsrt}
\bibliography{references}

%% I prefer to put these tables here rather than making the
%% front matter seemingly interminable. No-one cares, anyway!
%\listoffigures
%\listoftables

\chapter*{List of Acronyms}
\addcontentsline{toc}{chapter}{List of Acronyms}
\begin{acronym}[CERN]
\acro{EM}[EM]{electromagnetic}
\acro{QFT}[QFT]{quantum field theory}
\acro{QCD}[QCD]{quantum chromodynamics}
\acro{QED}[QED]{quantum electrodynamics}
\acro{LEP}[LEP]{Large Electron-Positron}
\acro{MC}[MC]{Monte Carlo}
\acro{CKM}[CKM]{Cabibbo-Kobayashi-Maskawa}
\acro{CMS}[CMS]{Compact Muon Solenoid}
\acro{LHC}[LHC]{Large Hadron Collider}
\acro{CERN}[CERN]{the European Organisation for Nuclear Research }
\acro{PS}[PS]{Proton Synchrotron}
\acro{SPS}[SPS]{Super Proton Synchrotron}
\acro{PSB}[PSB]{Proton Synchrotron Booster}
\acro{LS1}[LS1]{Long Shutdown 1}
\acro{MSSM}[MSSM]{minimal supersymmetric standard model}
\acro{SM}[SM]{standard model}
\acro{SUSY}[SUSY]{supersymmetry}
\acro{2HDM}[2HDM]{two Higgs doublet model}
\acro{ECAL}[ECAL]{electromagnetic calorimeter}
\acro{HCAL}[HCAL]{hadron calorimeter}
\acro{TIB}[TIB]{Tracker Inner Barrel}
\acro{TID}[TID]{Tracker Inner Disk}
\acro{TOB}[TOB]{Tracker Outer Barrel}
\acro{TEC}[TEC]{Tracker EndCap}
\acro{EB}[EB]{ECAL Barrel}
\acro{EE}[EE]{ECAL Endcap}
\acro{HB}[HB]{Hadron Barrel Calorimeter}
\acro{HO}[HO]{Hadron Outer Calorimeter}
\acro{HE}[HE]{Hadron Endcap Calorimeter}
\acro{HF}[HF]{Hadron Forward Calorimeter}
\acro{BSM}[BSM]{beyond the standard model}
\acro{L1}[L1]{Level-1}
\acro{HLT}[HLT]{High-level trigger}
\acro{CSC}[CSC]{Cathode Strip Chamber}
\acro{DT}[DT]{Drift Tube chamber}
\acro{RPC}[RPC]{Resistive Plate Chamber}
\acro{DAQ}[DAQ]{Data Acquisition}
\acro{WLCG}[WLCG]{Worldwide LHC Computing Grid}
\acro{RF}[RF]{radio frequency}
\acro{PF}[PF]{Particle Flow}
\acro{BDT}[BDT]{boosted decision tree}
\acro{KF}[KF]{Kalman Filter}
\acro{CTF}[CTF]{Combinatorial Track Finder}
\acro{GSF}[GSF]{Gaussian Sum Filter}
%\acro{KF}[KF]{Kalman Filter}
\acro{DA}[DA]{Deterministic Annealing}
\acro{CHS}[CHS]{charged hadron subtraction}
\acro{CSV}[CSV]{Combined Secondary Vertex}
\acro{CL}[CL]{confidence level}
\acro{IVF}[IVF]{Inclusive Vertex Finder}
\acro{UE}[UE]{underlying event}
\acro{MPF}[MPF]{missing transverse energy projection fraction}
\end{acronym}

\appendix
\renewcommand{\chaptername}{Appendix}
\chapter{\texorpdfstring{Overview of systematic uncertainties in the MSSM \AHtotautau analysis}{Overview of systematic uncertainties in the MSSM A/H to tautau analysis}}
\label{appendix:uncerts}
An overview of the systematic uncertainties that are applied in the MSSM \AHtotautau analysis (chapter \ref{chap:mssm})
are summarised in this appendix. The uncertainties are given in
table \ref{tab:SystematicUncertainties_mt} for the $\mu\tau_h$
channel, in table \ref{tab:SystematicUncertainties_et} for the $e\tau_h$ channel,
in table \ref{tab:SystematicUncertainties_tt} for the $\tau_h\tau_h$ channel and
in table \ref{tab:SystematicUncertainties_em} for the $e\mu$ channel. Additionally,
the systematic uncertainties for the $Z\rightarrow\mumu$ control region are shown
in table \ref{tab:SystematicUncertainties_zmm}. Correlations are also indicated.
Some of the uncertainties
are different for different processes, where this is the case ranges are given.
Uncertainties marked "Fully correlated" are correlated between the different
decay channels and event categories, those marked "Fully uncorrelated" are
uncorrelated between the different decay channels and event categories. "Cats:C,chns:U"
indicates that the uncertainty is correlated between categories but uncorrelated between
the different decay channels.
\begin{table}[!h]
\begin{center}
\caption{\footnotesize Systematic uncertainties that affect the estimated number of signal
or background events in the $\mu \tau_{h}$ channel.} 
 {\tiny
 \begin{tabular}{p{2cm}|p{1cm}p{1cm}p{1cm}p{1cm}|p{1cm}p{1cm}p{1cm}p{1cm}|p{3cm}}
\toprule
     & \multicolumn{8}{|c}{Event yield uncertainty by event category} &  \\
    \midrule
\textbf{ Process }
    & \multicolumn{4}{|c}{\textbf{No b-tag}} & \multicolumn{4}{|c}{\textbf{B-tag}} & \textbf{Correlation}           \\
     & SR & OS high $m_{\text{T}}$ & SS low $m_{\text{T}}$ & SS high $m_{\text{T}}$ & SR & OS high $m_{\text{T}}$ & SS low $m_{\text{T}}$ & SS high $m_{\text{T}}$ & \\
    \midrule
    \multicolumn{10}{l}{\textbf{Integrated luminosity 13 TeV}} \\
    All from MC    & 6.2\%  &6.2\%  & 6.2\%    & 6.2\% & 6.2\% & 6.2\% & 6.2\% & 6.2\% & Fully correlated\\
    \midrule
    \multicolumn{10}{l}{\textbf{Jet energy scale }}\\
    All from MC & 0--1\% & 0--2\% & 0--1\% & 0--1\%& 0--12\% & 3--11\% & 0--15\% & 0--14\% & Fully correlated \\
    \midrule
    \multicolumn{10}{l}{\MET~\textbf{scale} }\\
    ZTT,W     & 2\% & & 2\% & &2\% &  & 2\% & & Corr. between chn/cat; TTT,TTJ, VVT,VVJ                         \\
    TT(T/J),VV(T/J) & 2\% & & 2\% & &2\% & & 2\% & & uncorr. from ZTT,W \\
    \midrule
    \multicolumn{4}{l}{\MET~\textbf{resolution}} \\
    ZTT,W     & 2\% & & 2\% & &2\% &  & 2\% & & Corr. between chn/cat; TTT,TTJ,VVT,VVJ                         \\
    TT(T/J),VV(T/J) & 2\% & & 2\% & &2\% & & 2\% & & uncorr. from ZTT,W \\
    \midrule
    \multicolumn{10}{l}{\textbf{Muon identification and trigger} }\\
    All from MC & 2\% & 2\% &2\% &2\%  & 2\%  & 2\% & 2\% & 2\% & Fully correlated      \\
    \midrule
    \multicolumn{10}{l}{\textbf{Tau-lepton identification}}\\
    ZTT,TTT,VVT, signal     & 5\% & 5\% & 5\% & 5\%   & 5\%  & 5\% & 5\% & 5\% & Fully correlated \\
    ZTT,TTT,VVT, signal     & 3\% & 3\% & 3\% & 3\%   & 3\%  & 3\% & 3\% & 3\% & Cats:C,chns:U    \\
    \midrule
    \multicolumn{10}{l}{\textbf{b-Tagging efficiency} }\\
    All from MC & 0--3\% & 0--3\% & 0--3\% & 0--2\% & 1--3\% &  0--3\%& 0--3\% & 0--3\%& Fully correlated\\
    W &  &  &  &  & 4\% & 4\% & 6\% & 6\%& Fully correlated\\
    \midrule
    \multicolumn{10}{l}{\textbf{light jet mis-tagging rate } }\\
    All from MC & & & & & 1--7\% & & & & Fully correlated\\
    \midrule
    \multicolumn{10}{l}{\textbf{Tau-lepton energy scale}}\\
    ZTT,TTT,VVT, signal     & shape & shape & shape & shape  & shape & shape & shape & shape & Cats:C,chns:U   \\
    \midrule
    \multicolumn{10}{l}{\textbf{High}-$p_{\text{T}}$ $\Pgth$\textbf{ ID efficiency } } \\
    ZTT,TTT,VVT, signal    & shape & shape & shape & shape  & shape & shape & shape & shape & Cats:C,chns:U   \\
    \midrule
    \multicolumn{10}{l}{\textbf{Top quark} $p_{\text{T}}$ \textbf{reweighting} }\\
    TTT,TTJ  & shape & shape & shape & shape & shape & shape & shape & shape & Fully correlated    \\
    \midrule
    \multicolumn{10}{l}{\textbf{Drell-Yan reweighting } }\\
    ZTT       & shape & shape & shape & shape  & shape & shape & shape & shape &Fully correlated              \\
    \midrule
    \multicolumn{10}{l}{\textbf{Jet} $\rightarrow\Pgth$ \textbf{fake rate reweighting } }\\
    W         & shape &  &  &   & shape &  &  &  &Fully correlated              \\
    \midrule
    \multicolumn{10}{l}{\textbf{Normalisation, }$\PZ$ \textbf{production} }\\
    ZL,ZJ       & 4\% & 4\% & 4\% & 4\% & 4\%  & 4\% & 4\% & 4\% & Fully correlated              \\
    ZTT         &  & 4\% & 4\% & 4\% &  & 4\% & 4\% & 4\% & Fully correlated              \\
   \multicolumn{10}{l}{ $Z\to\Pgt\Pgt$\textbf{: $Z\to\Pgt\Pgt / Z\to\Pgm\Pgm$ extrapolation } } \\
    ZTT         & 3\% & 3\% & 3\% & 3\% & 5\% & 5\% & 5\% & 5\% & Fully uncorrelated           \\
    \multicolumn{10}{l}{\textbf{Normalisation,} \ttbar}\\
    TTT,TTJ     & 6\% & 6\% & 6\% & 6\%   & 6\% & 6\% &6\% & 6\%  & Fully correlated                \\
    \multicolumn{10}{l}{\textbf{Normalisation, di-boson + single-top} } \\
    VVT,VVJ     & 5\% & 5\% & 5\% & 5\%   & 5\% & 5\% & 5\% & 5\% & Fully correlated         \\
   \multicolumn{10}{l}{ \textbf{Normalisation,} $\PZ\to\Pgm\Pgm$: $\Pgm$ \textbf{misidentified as} $\Pgth$ }\\
    ZL     & 30\%  & 30\% & 30\% & 30\%   & 30\% & 30\% & 30\% &30\% &Cats:C,chns:U        \\
    \multicolumn{10}{l}{\textbf{Normalisation,} $\PZ$\textbf{+jets : jet misidentified as} $\Pgth$ } \\
    ZJ,TTJ     & 20\%  & 20\% &20\% &20\%      & 20\% & 20\% &20\% &20\%  & Cats:C,chns:U  \\
    \midrule
    \multicolumn{10}{l}{\textbf{W OS/SS ratio (stat) } } \\
    W & 2\% & 2\% & & &11\% &11\% & & & Fully uncorrelated \\
    \multicolumn{10}{l}{\textbf{W OS/SS ratio (syst) } }\\
    W & 8\% & 8\% & & &10\% &10\% & & &Fully uncorrelated \\
    \multicolumn{10}{l}{\textbf{W low/high} $m_{T}$ \textbf{ratio (stat)}}\\
    W & 2\% & & 2 \% & &14\% & & 14\% & & Fully uncorrelated \\
    \multicolumn{10}{l}{\textbf{W low/high} $m_{T}$ \textbf{ratio (syst)} }\\
    W & 20\% & &20\% & & 20\% & & 20\% & & Fully uncorrelated \\
    %\multicolumn{10}{l}{\textbf{W b-tag extrap (stat)}}\\
    %W &  & &  & &12\% & 8\%& 22\% & 14\%& Fully uncorrelated \\
    %\multicolumn{10}{l}{\textbf{W b-tag extrap (syst)} }\\
    %W &  & & & & 10\% & & 10\% & & Fully uncorrelated \\
    \multicolumn{10}{l}{\textbf{QCD OS/SS ratio (syst) }}\\
    QCD & 4\% & 4\% & & & 60\% & 60\% & & &Fully uncorrelated \\
\bottomrule
\end{tabular}
}
\label{tab:SystematicUncertainties_mt}
\end{center}
\end{table}


\begin{table}[!h]
\begin{center}
\caption{\footnotesize Systematic uncertainties that affect the estimated number of signal
or background events in the \etau channel.}
 {\tiny
 \begin{tabular}{p{2cm}|p{1cm}p{1cm}p{1cm}p{1cm}|p{1cm}p{1cm}p{1cm}p{1cm}|p{3cm}}
\toprule
     & \multicolumn{8}{|c}{Event yield uncertainty by event category} &  \\
    \midrule
\textbf{ Process }
    & \multicolumn{4}{|c}{\textbf{No b-tag}} & \multicolumn{4}{|c}{\textbf{B-tag}} & \textbf{Correlation}           \\
     & SR & OS high $m_{\text{T}}$ & SS low $m_{\text{T}}$ & SS high $m_{\text{T}}$ & SR & OS high $m_{\text{T}}$ & SS low $m_{\text{T}}$ & SS high $m_{\text{T}}$ & \\
    \midrule
    \multicolumn{10}{l}{\textbf{Integrated luminosity 13 TeV}} \\
    All from MC    & 6.2\%  & 6.2\%  & 6.2\%    & 6.2\% & 6.2\% & 6.2\% & 6.2\% & 6.2\% & Fully correlated\\
    \midrule
    \multicolumn{10}{l}{\textbf{Jet energy scale }}\\
    All from MC & 0--1\% & 0--1\% & 0--2\% & 0--1\%& 1--14\% & 0--12\% & 0--15\% & 0--15\% & Fully correlated \\
    \midrule
    \multicolumn{10}{l}{\MET~\textbf{scale}}\\
    ZTT,W     & 2\% & & 2\% & &2\% &  & 2\% & & Corr. between chn/cat, TTT,TTJ,VVT,VVJ                         \\
    TT(T/J),VV(T/J) & 2\% & & 2\% & &2\% & & 2\% & & uncorr. from ZTT,W \\
    \midrule
    \multicolumn{10}{l}{\MET~\textbf{resolution}} \\
    ZTT,W     & 2\% & & 2\% & &2\% &  & 2\% & & Corr. between chn/cat, TTT,TTJ,VVT,VVJ          \\
    TT(T/J),VV(T/J) & 2\% & & 2\% & &2\% & & 2\% & & uncorr. from ZTT,W \\
    \midrule
    \multicolumn{10}{l}{\textbf{Electron identification and trigger} }\\
    All from MC & 2\% & 2\% &2\% &2\%  & 2\%  & 2\% & 2\% & 2\% & Fully correlated      \\
    \midrule
    \multicolumn{10}{l}{\textbf{Tau-lepton identification}}\\
    ZTT,TTT,VVT, signal     & 5\% & 5\% & 5\% & 5\%   & 5\%  & 5\% & 5\% & 5\% & Fully correlated \\
    ZTT,TTT,VVT, signal     & 3\% & 3\% & 3\% & 3\%   & 3\%  & 3\% & 3\% & 3\% & Cats:C,chns:U    \\
    \midrule
    \multicolumn{10}{l}{\textbf{b-Tagging efficiency} }\\
    All from MC & 0--3\% & 0--3\% & 0--4\% & 0--4\% & 1--4\% & 0--2\% & 0--2\% & 0--2\% & Fully correlated\\
    W &  &  &  &  & 5\% & 2\% & 10\% & 3\% & Fully correlated\\
    \midrule
    \multicolumn{10}{l}{\textbf{Light jet mis-tagging rate } }\\
    All from MC & & & & & 0--3\% & & & & Fully correlated\\
    \midrule
    \multicolumn{10}{l}{\textbf{Tau-lepton energy scale}}\\
    ZTT,TTT,VVT, signal     & shape & shape & shape & shape  & shape & shape & shape & shape & Cats:C,chns:U   \\
    \midrule
    \multicolumn{10}{l}{\textbf{High}-$p_{\text{T}}$ $\Pgth$\textbf{ ID efficiency } } \\
    ZTT,TTT,VVT, signal    & shape & shape & shape & shape  & shape & shape & shape & shape & Cats:C,chns:U   \\
    \midrule
    \multicolumn{10}{l}{\textbf{Top quark} $p_{\text{T}}$ \textbf{reweighting} }\\
    TTT,TTJ  & shape & shape & shape & shape & shape & shape & shape & shape & Fully correlated    \\
    \midrule
    \multicolumn{10}{l}{\textbf{Drell-Yan reweighting } }\\
    ZTT       & shape & shape & shape & shape  & shape & shape & shape & shape &Fully correlated              \\
    \midrule
    \multicolumn{10}{l}{\textbf{Jet} $\rightarrow\tau_h$ \textbf{fake rate reweighting } }\\
    W         & shape &  &  &   & shape &  &  &  &Fully correlated              \\
    \midrule
    \multicolumn{10}{l}{\textbf{Normalisation, }$\PZ$ \textbf{production} }\\
    ZL,ZJ       & 4\% & 4\% & 4\% & 4\% & 4\%  & 4\% & 4\% & 4\% & Fully correlated              \\
    ZTT         &  & 4\% & 4\% & 4\% &  & 4\% & 4\% & 4\% & Fully correlated              \\
   \multicolumn{10}{l}{ $Z\to\Pgt\Pgt/Z\to\Pgm\Pgm$\textbf{ extrapolation} } \\
    ZTT         & 3\% & 3\% & 3\% & 3\% & 5\% & 5\% & 5\% & 5\% & Fully uncorrelated           \\
    \multicolumn{10}{l}{\textbf{Normalisation,} \ttbar}\\
    TTT,TTJ     & 6\% & 6\% & 6\% & 6\%   & 6\% & 6\% &6\% & 6\%  & Fully correlated                \\
    \multicolumn{10}{l}{\textbf{Normalisation, di-boson + single-top} } \\
    VVT,VVJ     & 5\% & 5\% & 5\% & 5\%   & 5\% & 5\% & 5\% & 5\% & Fully correlated         \\
   \multicolumn{10}{l}{ \textbf{Normalisation,} $\PZ\to\Pe\Pe$: $\Pe$ \textbf{misidentified as} $\Pgth$ }\\
    ZL     & 30\%  & 30\% & 30\% & 30\%   & 30\% & 30\% & 30\% &30\% & Cats:C,chns:U       \\
    \multicolumn{10}{l}{\textbf{Normalisation,} $\PZ$\textbf{+jets : jet misidentified as} $\Pgth$ } \\
    ZJ,TTJ     & 20\%  & 20\% &20\% &20\%      & 20\% & 20\% &20\% &20\%  & Cats:C,chns:U       \\
    \midrule
    \multicolumn{10}{l}{\textbf{W OS/SS ratio (stat) } } \\
    W & 2\% & 2\% & & &14\% &14\% & & & Fully uncorrelated \\
    \multicolumn{10}{l}{\textbf{W OS/SS ratio (syst) } }\\
    W & 8\% & 8\% & & &10\% &10\% & & &Fully uncorrelated \\
    \multicolumn{10}{l}{\textbf{W low/high} $m_{T}$ \textbf{ratio (stat)}}\\
    W & 2\% & & 2 \% & &17\% & & 17\% & & Fully uncorrelated \\
    \multicolumn{10}{l}{\textbf{W low/high} $m_{T}$ \textbf{ratio (syst)} }\\
    W & 20\% & &20\% & & 20\% & & 20\% & & Fully uncorrelated \\
    %\multicolumn{10}{l}{\textbf{W b-tag extrap (stat)}}\\
    %W &  & &  & &11\% &14\% & 21\% & 16\%& Fully uncorrelated \\
    %\multicolumn{10}{l}{\textbf{W b-tag extrap (syst)} }\\
    %W &  & & & & 10\% & & 10\% & & Fully uncorrelated \\
    \multicolumn{10}{l}{\textbf{QCD OS/SS ratio (syst) }}\\
    QCD & 12\% & 12\% & & & 60\% & 60\% & & &Fully uncorrelated \\
    \bottomrule
\end{tabular}
}
\label{tab:SystematicUncertainties_et}
\end{center}
\end{table}


\begin{table}[!h]
\begin{center}
\caption{Systematic uncertainties that affect the estimated number of signal
or background events in the \tautau channel.}
{\tiny
\begin{tabular}{l|cc|p{5cm}}
\toprule
     & \multicolumn{2}{|c}{Event yield uncertainty by event category} & \\
    \midrule
    Process & No b-tag & B-tag & Correlation                   \\
    \midrule
    \multicolumn{4}{l}{\textbf{Integrated luminosity 13 TeV}}\\
    All from MC      & 6.2\%      & 6.2\%  & Fully correlated                           \\
    \midrule
    \multicolumn{4}{l}{\textbf{Jet energy scale}} \\
    All from MC      & 0--4\%      & 0--9\%  & Fully correlated                    \\
    \midrule
    \multicolumn{4}{l}{\MET~\textbf{scale}} \\
    ZTT,W     & 2\%     & 2\% & Correlated between chn/cat,                          \\
    TTT,TTJ,VVT,VVJ     & 2\%     & 2\% & ZTT/W uncorrelated from TTT,TTJ,VVT,VVJ                          \\
    \midrule
    \multicolumn{4}{l}{\MET~\textbf{resolution}} \\
    ZTT,W     & 2\%     & 2\% & Correlated between chn/cat,                          \\
    TTT,TTJ,VVT,VVJ     & 2\%     & 2\% & ZTT/W uncorrelated from TTT,TTJ,VVT,VVJ                          \\
    \midrule
    \multicolumn{4}{l}{\textbf{Tau-lepton identification and trigger}} \\
    ZTT,TTT,VVT, signal         & 10\%    & 10\%  & Fully correlated                      \\
    ZTT,TTT,VVT, signal         & 9.2\%     & 9.2\%   & Correlated between categories, uncorrelated between channels \\
    \midrule
    \multicolumn{4}{l}{\textbf{b-Tagging efficiency} }\\
    All from MC   & 0--4\%     & 0--11\%  & Fully correlated                  \\
    \midrule
   \multicolumn{4}{l}{\textbf{Light jet mis-tagging rate}} \\
    All from MC    &  0--1\%    & 0--12\%  & Fully correlated                  \\
    \midrule
    \multicolumn{4}{l}{\textbf{Tau-lepton energy scale}} \\
    ZTT,TTT,VVT, signal       & shape      & shape & Correlated between categories, uncorrelated between channels                       \\
    \midrule
    \multicolumn{4}{l}{\textbf{High}-$p_{\text{T}}$ $\Pgth$ \textbf{ID efficiency}} \\
    ZTT,TTT,VVT, signal       & shape      & shape & Correlated between categories, uncorrelated between channels   \\
    \midrule
    \multicolumn{4}{l}{\textbf{Drell-Yan reweighting } } \\
    ZTT       & shape      & shape & Fully correlated                       \\
    \midrule
    \multicolumn{4}{l}{\textbf{Top quark} $p_{\text{T}}$ \textbf{reweighting } } \\
    TTT,TTJ & shape & shape & Fully correlated                  \\
    \midrule
    \multicolumn{4}{l}{\textbf{Normalisation, }$\PZ$ \textbf{production}}\\
    ZL,ZJ       & 4\%      & 4\%  & Fully correlated                    \\
    \midrule
    \multicolumn{4}{l}{$Z\to\Pgt\Pgt/Z\to\Pgm\Pgm$\textbf{: extrapolation}}\\
    ZTT         & 3\%        & 5\%  & Fully uncorrelated                   \\
    \midrule
    \multicolumn{4}{l}{\textbf{Normalisation, }\ttbar}\\
    TTT,TTJ        & 6\%       & 6\%  & Fully correlated                       \\
    \midrule
    \multicolumn{4}{l}{\textbf{Normalisation, di-boson + single-top}}\\
    VVT,VVJ &  5\%        & 5\%       & Fully correlated                        \\
    \midrule
    \multicolumn{4}{l}{\textbf{Normalisation, }$\PZ\to\Pe\Pe$: $\Pe$\textbf{ misidentified as }$\Pgth$}\\
    ZL & 10\%     & 10\%       & Correlated between categories, uncorrelated between channels                        \\
    \midrule
    \multicolumn{4}{l}{\textbf{Normalisation, }$\PZ$\textbf{+jets : jet misidentified as} $\Pgth$}\\
    ZJ,W,TTJ & 20\%     & 20\%       & Correlated bewteen categories, uncorrelated between channels     \\
    \midrule
    \multicolumn{4}{l}{\textbf{Normalisation, QCD multijet (stat) }} \\
    QCD  & 2\% & 20\%  & Fully uncorrelated\\
    \multicolumn{4}{l}{\textbf{Normalisation, QCD multijet (syst) }} \\
    QCD  & 12\% & 14\%  & Fully uncorrelated\\
    \midrule
    \multicolumn{4}{l}{\textbf{Normalisation, \Wjets }}\\
    W & 4\% & 4\% & Fully correlated \\
\bottomrule
\end{tabular}}
\label{tab:SystematicUncertainties_tt}
\end{center}
\end{table}


\begin{table}[!h]
\begin{center}
\caption{Systematic uncertainties that affect the estimated number of signal
or background events in the \emu channel.}
{\scriptsize
\begin{tabular}{l|cc|p{5cm}}
   \toprule
     & \multicolumn{2}{|c}{Event yield uncertainty by event category} &   \\
    \midrule
    \textbf{Process}
    &  \textbf{No b-tag} & \textbf{B-tag} & \textbf{Correlation}                   \\
    \midrule
    \multicolumn{4}{l}{\textbf{Integrated luminosity 13 TeV}}\\
    All from MC      & 6.2\%      & 6.2\% & Fully correlated                            \\
    \midrule
    \multicolumn{4}{l}{\textbf{Jet energy scale}}\\
    All from MC   & 0--1\% & 1--69\% &Fully correlated \\
    \midrule
    \multicolumn{4}{l}{\MET~\textbf{scale}} \\
    ZTT,W    & 2\%     & 2\% & Correlated between chn/cat,                          \\
    TT,VV    & 2\%     & 2\% & ZTT,W uncorrelated from TT,VV \\
    \midrule
   \multicolumn{4}{l}{\MET~\textbf{resolution}} \\
    ZTT,W & 2 \%    & 2\%  & Correlated between chn/cat,\\
    TT,VV & 2\%     & 2\%  & ZTT,W uncorrelated from TT,VV\\
    \midrule
    \multicolumn{4}{l}{\textbf{Electron identification and trigger}}\\
    All from MC       & 2\%        & 2\% & Fully correlated                              \\
    \midrule
    \multicolumn{4}{l}{\textbf{Muon identification and trigger}  }\\
    All from MC      & 2\%     & 2\%  & Fully correlated                      \\
    \midrule
    \multicolumn{4}{l}{\textbf{b-Tagging efficiency}} \\
    All from MC     & 0--3\%     & 1--3\%  & Fully correlated                  \\
    \midrule
    \multicolumn{4}{l}{\textbf{Light jet mis-tagging rate}}\\
    All from MC      &     & 0--10\% & Fully correlated                    \\
    \midrule
    \multicolumn{4}{l}{\textbf{Electron energy scale}} \\
    ZTT,signal      & shape      & shape  & Fully correlated                \\
    \midrule
    \multicolumn{4}{l}{\textbf{Drell-Yan reweighting }}\\
     ZTT       & shape      & shape & Fully correlated                       \\
   \midrule
    \multicolumn{4}{l}{\textbf{Top quark} $p_{\text{T}}$ \textbf{reweighting}}\\
    TT & shape & shape  & Fully correlated                 \\
    \midrule
    \multicolumn{4}{l}{\textbf{Normalisation, }$\PZ$ \textbf{production} }\\
    ZLL      & 4\%      & 4\%  & Fully correlated                   \\
    \midrule
    \multicolumn{4}{l}{$Z\to\Pgt\Pgt/Z\to\Pgm\Pgm$\textbf{: extrapolation} }\\
     ZTT        & 3\%        & 5\%   & Fully uncorrelated                      \\
    \midrule
    \multicolumn{4}{l}{\textbf{Normalisation, }\ttbar}\\
    TT        & 6\%       & 6\% & Fully correlated                        \\
    \midrule
    \multicolumn{4}{l}{\textbf{Normalisation, di-boson + single-top}} \\
    VV        & 5\%       & 5\% & Fully correlated                       \\
    \midrule
    \multicolumn{4}{l}{\textbf{Normalisation, QCD multijet }}\\
    QCD & 23\% & 34\% & Fully uncorrelated\\
    \midrule
    \multicolumn{4}{l}{\textbf{Normalisation, \Wjets }}\\
    W+jets & 4\% & 4\% & Fully correlated\\
\bottomrule
\end{tabular}}
\label{tab:SystematicUncertainties_em}
\end{center}
\end{table}

\begin{table}[!h]
\begin{center}
\caption{Systematic uncertainties that affect the estimated number of
or background events in the $Z\rightarrow\mu\mu$ channel.}
{\scriptsize
\begin{tabular}{l|cc|p{5cm}}
   \toprule
     & \multicolumn{2}{|c}{Event yield uncertainty by event category} &   \\
    \midrule
    \textbf{Process}
    &  \textbf{No b-tag} & \textbf{B-tag} & \textbf{Correlation}                   \\
    \midrule
    \multicolumn{4}{l}{\textbf{Integrated luminosity 13 TeV}}\\
    All from MC (not ZL)     & 6.2\%      & 6.2\% & Fully correlated                            \\
    \midrule
    \multicolumn{4}{l}{\textbf{Jet energy scale}}\\
    All from MC   & 0--1\% & 1--69\% &Fully correlated \\
    \midrule
    \multicolumn{4}{l}{\MET~\textbf{scale}} \\
    ZTT,W    & 2\%     & 2\% & Correlated between chn/cat,                          \\
    TT,VV    & 2\%     & 2\% & ZTT,W uncorrelated from TT,VV \\
    \midrule
   \multicolumn{4}{l}{\MET~\textbf{resolution}} \\
    ZTT,W & 2 \%    & 2\%  & Correlated between chn/cat,\\
    TT,VV & 2\%     & 2\%  & ZTT,W uncorrelated from TT,VV\\
    \midrule
    \multicolumn{4}{l}{\textbf{Muon identification and trigger}}\\
    All from MC       & 2--4\%        & 2--4\% & Fully correlated                              \\
    \midrule
    \multicolumn{4}{l}{\textbf{b-Tagging efficiency}} \\
    All from MC     & 0--3\%     & 1--3\%  & Fully correlated                  \\
    \midrule
    \multicolumn{4}{l}{\textbf{Light jet mis-tagging rate}}\\
    All from MC      &     & 0--10\% & Fully correlated                    \\
    \midrule
    \multicolumn{4}{l}{\textbf{Normalisation, }\ttbar}\\
    TT        & 6\%       & 6\% & Fully correlated                        \\
    \midrule
    \multicolumn{4}{l}{\textbf{Normalisation, di-boson + single-top}} \\
    VV        & 5\%       & 5\% & Fully correlated                       \\
    \midrule
    \multicolumn{4}{l}{\textbf{Normalisation, \Wjets }}\\
    W+jets & 4\% & 4\% & Fully correlated\\
    \bottomrule
\end{tabular}}
\label{tab:SystematicUncertainties_zmm}
\end{center}
\end{table}





%% If you have time and interest to generate a (decent) index,
%% then you've clearly spent more time on the write-up than the 
%% research ;-)
%\printindex
