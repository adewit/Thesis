%% Title
\titlepage[Imperial College London\\ Department of Physics]{
  A dissertation submitted to Imperial College London\\ for the degree of Doctor of Philosophy}

\begin{abstract}[]
The copyright of this thesis rests with the author and is made available under a
Creative Commons Attribution Non-Commercial No Derivatives licence. Researchers
are free to copy, distribute or transmit the thesis on the condition that they
attribute it, that they do not use it for commercial purposes and that they do
not alter, transform or build upon it. For any reuse or redistribution,
researchers must make clear to others the licence terms of this work.
\end{abstract}

% Abstract
\begin{abstract}%[\smaller \thetitle\\ \vspace*{1cm} \smaller {\theauthor}]
  %\thispagestyle{empty}
The Compact Muon Solenoid (CMS) detector is a general-purpose detector located at the Large Hadron
Collider at CERN. It was designed to test the predictions of the standard model 
and to search for the Higgs boson in addition to searches for new physics beyond
the standard model.
The analyses presented in this thesis are searches for neutral
Higgs bosons with tau leptons in the final state. These searches use proton-proton collision data recorded
by CMS during 2012, 2015 and 2016. The searches are performed in the
context of the minimal supersymmetric standard model (MSSM). One of the analyses, using $19.7\,\invfb$ 
of data collected
during 2012 at $\sqrt{s}=8\,\TeV$, is a search for a heavy neutral Higgs boson
decaying into two $125\,\GeV$ Higgs bosons with standard model-like properties,
with two b-quarks and two tau leptons in the final state. The other analyses use 
$2.3\,\invfb$ of data collected during 2015 and $12.9\,\invfb$ of data collected during 2016, both
at $\sqrt{s}=13\,\TeV$. These are searches for neutral MSSM
Higgs bosons in the mass range of \mbox{$90\,\GeV$--$3.2\,\TeV$} directly decaying
into pairs of tau leptons. 
No significant excess is observed in any of the searches. Upper limits at the 95\% confidence 
level are set on the cross section times branching ratio for the production of Higgs bosons in the MSSM.
In addition, limits are set in the \mbox{\mA-\tanb}~parameter space of 
several MSSM benchmark scenarios, and in the \mbox{\cosba-\tanb}~parameter space of 
a two Higgs doublet model.

\end{abstract}


% Declaration
\begin{declaration}
  I declare that the work in this thesis is mine. Figures and results taken from other sources
include the appropriate reference in the text or figure caption. Figures labelled `CMS' are sourced
directly from CMS publications. The `CMS Preliminary' label indicates figures which
have been made public via a preliminary public document or a public website, but are not
included in a publication in a peer-reviewed journal. Figures labelled `CMS' or `CMS preliminary', 
including those made by myself, include the relevant reference in the figure caption.
Chapters \ref{chap:theory}--\ref{chap:objects} do not contain original work by myself, however they 
give information on the theoretical background, the CMS detector and the standard reconstruction algorithms
used within the CMS collaboration which underpin my own work described in later chapters.
The work for the \Htohhtobbtautau and \AHtotautau analyses presented in chapters~\ref{chap:hhh} and \ref{chap:mssm} was 
carried out as part of the CMS \Htotautau working group, and in collaboration with other members of the group.
For the results of the \Htohhtobbtautau analysis I contributed to the data analysis of the \etau and \mutau
final states of the di-tau pair. I 
was responsible for the statistical inference of the analyses in the \etau, \mutau and \tautau final states
of the di-tau pair. This included
the interpretation of the results in MSSM and 2HDM benchmark scenarios. The 
work is included in reference \cite{CMS-HIG-14-034}. For the results of the \AHtotautau analysis
I was responsible for all stages of the analysis, including the optimisation
of object selection and event categorisation, as well as studies of background
methods for the \etau, \mutau and \tautau final states in the public result,
evaluation of systematic uncertainties in the \etau, \mutau, \tautau and \emu final states, and production of the statistical
results, including model interpretations. This work is included in the most recent
preliminary result \cite{CMS-PAS-HIG-16-037}, as well as a predating preliminary
result using a smaller dataset \cite{CMS-PAS-HIG-16-006}.
The work on the combination of \AHtotautau analyses presented in chapter~\ref{sec:mssm_combination} has
not been made public, and draws on the results of \cite{CMS-PAS-HIG-16-037} and \cite{CMS-PAS-HIG-16-006}.
\enlargethispage{2\baselineskip}

  \begin{flushright}
  Adinda de Wit
  \end{flushright}
\end{declaration}


%% Acknowledgements
\begin{acknowledgements}
I would like to thank Imperial College and the Imperial College HEP group 
for giving me the opportunity to carry out this research, and especially
for the valuable experience of spending two years at CERN. Thanks
to my supervisors, David Colling for helping secure this funding and his support, and Gavin Davies
for many useful discussions and general advice.
The work in this thesis would not have come to fruition without Rebecca, who taught me 
all the important aspects of doing an analysis as well as always being
ready to discuss ideas, and I am very grateful to her for that. My thanks extend
to Andrew in this regard, who was also always available to discuss physics and coding issues.
I would also like to thank the people I've worked with in the CMS $\PHiggs\rightarrow\tau\tau$ group,
particularly those involved in the most recent MSSM result. 

Thanks to my friends, nearby and far away, for all of the good times and laughter.
Most importantly, I would like to thank those closest to me for their love: My mother, who is always there
for me despite the fact I live so far away; my father, whose love of science I inherited and who
would certainly be a little bit jealous of the computing power I get to play with on a daily basis; my sister, 
who always understands me as she's followed a similar path; and Andrew, for making the days so much brighter.

\begin{flushright}
Adinda de Wit
\end{flushright}
\end{acknowledgements}


%% Preface
%\begin{preface}
%\end{preface}
%% ToC
\tableofcontents
\listoffigures
\listoftables




%% Strictly optional!
%% I don't want a page number on the following blank page either.
%\thispagestyle{empty}
