\chapter{The \acs{CMS} experiment and the \acs{LHC}}
\label{chap:CMSLHC}

\section{The \acs{LHC}}
\label{sec:CMSLHC_LHC}

The \acf{LHC} \cite{lhc-machine}  is a 26.7 km long synchrotron hadron accelerator and collider below the surface of the 
Franco-Swiss border near Geneva. It
is installed in the tunnel that previously housed the Large Electron Positron accelerator
that was operated by \acf{CERN} between 1989 and 2000.

The \ac{LHC} was designed to collide beams of protons with each other at centre-of-mass
energies of up to 14 TeV. It therefore consists of two rings in which the two beams
of protons are individually accelerated. In addition to colliding beams of protons, the \ac{LHC}
is also used for proton--lead and lead--lead collisions.

Protons originate from hydrogen gas, where electrons are stripped off the 
hydrogen atoms using an electric field. The protons then pass through an injector
chain which increases the energy of the protons in several steps. After the protons
are created from the hydrogen gas, they are accelerated to a centre of mass energy of
50 MeV in the Linac2 linear accelerator. They are then passed on to the
\acf{PSB} which accelerates the protons until they reach a centre of mass energy of 1.4 GeV.
The next accelerator in the chain, the \acf{PS}, further accelerates the protons to 25 GeV,
with the \acf{SPS} bringing the energy up to 450 GeV. When this energy has been 
reached the protons are injected into the \ac{LHC}, where they are further accelerated to 
the required centre of mass energy. lbha blah blah. The beams collide at four points
around the ring, where the collisions are recorded by the ATLAS, CMS, ALICE, and LHCb detectors.

As the processes the \ac{LHC} was built to study have a small cross-section
compared with the total proton-proton inelastic cross-section (add figure!), the
LHC operates at high instantaneous luminosity. blah blah blah blah

After an initial testing phase, the \ac{LHC} becan its first physics run in May 2010 with 
a centre of mass energy of 7 TeV. The CMS experiment recorded an integrated luminosity of 45 pb$^{-1}$
during 2010. In 2011 the \ac{LHC} continued operating at a centre-of-mass energy of 7 TeV, delivering an integrated 
luminosity of 6.1 fb$^{-1}$ to the CMS experiment. The centre of mass energy was increased to 8 TeV
for the 2012 data-taking period, during which an integrated luminosity of 23.3 fb$^{-1}$ was delivered.
This completed Run 1 of the \ac{LHC}, after which the 2-year \ac{LS1} was used to 
upgrade the LHC and the detectors for collisions at a centre of mass energy of 13 TeV.

In April 2015 the \ac{LHC} restarted with collisions at a centre-of-mass energy of 13 TeV





