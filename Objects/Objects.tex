\chapter{Object Reconstruction}
\label{chap:objects}

In this chapter the physics objects necessary to be able to perform searches with \Pgt leptons
in the final state, and how they are reconstructed in \ac{CMS}, will be discussed. The 
descriptions correspond to the algorithms as used during Run 2. Where there are differences
between the algorithms used in Run 1 and Run 2 this will be discussed at the end
of the relevant section.

\section{Particle Flow}
\label{sec:objects_pf}

\section{Tracks and vertices}
\label{sec:objects_pv}

\section{Electrons}
\label{sec:objects_ele}

\section{Muons}
\label{sec:objects_muo}

\section{Hadronic taus}
\label{sec:objects_tau}
Taus are unstable particles and they decay before reaching the detector. In 17.4 \% of 
cases they decay to muons and neutrinos, with an additional 17.8 \% decaying to electrons
plus neutrinos. The remaining 64.8 \% decay hadronically. Hadronically decaying
taus are characterised by narrow jets containing either one or three charged
particles ($\pi^{\pm}, K^{\pm}$) and 0,1, or two neutral pions. An overview
of the possible decay modes is given in table \ref{tab:hadronic_tau_decays}

\begin{table}[htp]
\begin{center}
\caption{Summary of hadronic tau decay modes, with the branching fraction, and intermediate resonance where relevant, indicated \cite{pdg-2014}}
\begin{tabular}{@{}lll@{}}
%\toprule
\textbf{Decay mode} & \textbf{Resonance} &\textbf{Branching fraction [\%]}\\
\midrule
\Ptaupm $\rightarrow$ h$^{\pm}$\Pnut & & 11.5\%\\
\Ptaupm $\rightarrow$ h$^{\pm}$\Ppizero\Pnut& $\rho$(770) & 26.0\% \\
\Ptaupm $\rightarrow$ h$^{\pm}$\Ppizero\Ppizero\Pnut & a$_{1}$(1260) & 9.5\% \\
\Ptaupm $\rightarrow$ h$^{\pm}$h$^{\mp}$h$^{\pm}$\Pnut & a$_{1}$(1260) & 9.8\% \\
\Ptaupm $\rightarrow$ h$^{\pm}$h$^{\mp}$h$^{\pm}$\Ppizero\Pnut & & 4.8\%\\
Other modes with hadrons & & 3.2\% \\
\midrule
Total & & 64.8\% \\
%\bottomrule
\end{tabular}
\label{tab:hadronic_tau_decays}
\end{center}
\end{table}


%CP from what I wrote in the HIG-16-006/HIG-16-037 analysis notes:
Hadronic tau decays are reconstructed using the \texttt{hadrons plus strips} (HPS) algorithm~\cite{cms-tau-run1,cms-tau-2015}, which uses particle flow candidates to reconstruct and identify hadronic tau decays.
In the reconstruction step, particle-flow charged and neutral particles are grouped into combinations compatible with specific hadronic tau decay modes, and the 4-momentum
of the candidates is computed. In the identification step, discriminators that separate hadronic taus from quark and gluon jets and from electrons and muons are evaluated. \\
Taus in the hadrons plus strips algorithm are seeded by jets clustered with the anti-$k_{\text{T}}$ algorithm with a distance parameter $\Delta R = 0.4$.

To reconstruct the energy deposits $\pi^0$ candidates leave in the ECAL, photon and electron constituents of the jet that seeds the $\tau_{h}$ reconstruction are clustered into strips.
The e or $\gamma$ with highest $p_{\text{T}}$ that is not yet included in a strip is used to build a new strip.
The $\eta$ and $\phi$ of this candidate determine the initial position of the strip, the next highest $p_{\text{T}}$ e or $\gamma$  within an $\eta \times \phi$ window centered on the strip location is added to the strip and the position is recomputed as the energy-weighted average of the electron/photon constituents in the strip.
This procedure is repeated until there are no more electrons or photons with $p_{\text{T}} > 0.5$~GeV  within the strip window. The $\Delta \eta$ and $\Delta \phi$ of the strip are varied based on the $p_{\text{T}}$ or $E_{\text{T}}$ to be added to the strip and on the energy the strip already has, as\\

\begin{equation}\label{eqn:dynamicstrip}
\begin{split}
&\Delta \eta  = f(p_{\text{T}}^{e/\gamma}) + f(p_{\text{T}}^{\text{strip}})\\
&\Delta \phi  = g(p_{\text{T}}^{e/\gamma}) + g(p_{\text{T}}^{\text{strip}})\\
\end{split}
\end{equation}

Where $p_{\text{T}}^{e/\gamma}$ is the transverse momentum of the candidate to be added to the strip
and $p_{\text{T}}^{\text{strip}}$ is the transverse momentum of the strip before merging a new candidate in.\\
In addition, the strip size is bounded as $0.05 < \Delta\eta < 0.15$, $0.05 \Delta\phi < 0.3$.

The functions $f(p_{\text{T}})$ and $g(p_{\text{T}})$ are defined as~\ref{eqn:dynamicstripfg}.

\begin{equation}\label{eqn:dynamicstripfg}
\begin{split}
&f(p_{\text{T}}) = 0.2\cdot p_{\text{T}}^{-0.66}\\
&g(p_{\text{T}}) = 0.35\cdot p_{\text{T}}^{-0.71}\\
\end{split}
\end{equation}

If the $\Sigma p_{\text{T}}$ of the strip is at least 2.5~GeV, it is considered as a $\pi^0$ candidate.\\
 To reconstruct hadronic taus, charged particles and strips are combined into different signatures which are said to be
compatible with a certain decay mode if the set of cuts listed below is satisfied. If a candidate satisfies more than one of the hypotheses, the one that maximises the $p_{\text{T}}$ is retained.\\

The decay modes considered for reconstructing taus are:\\
\begin{itemize}
\item \textbf{One prong, 0 $\pi^0$ :} One charged particle, no strips.
\item \textbf{One prong, 1 $\pi^0$ :} One charged particle + one strip with mass $ 0.3 < m_{\tau} < 1.3 \sqrt{p_{\text{T}}/100}$~GeV. The mass window upper limit is constrained to lie between 1.3 and 4.2 GeV.
\item \textbf{One prong, 2 $\pi^0$ :} One charged particle + two strips. The $\tau_{h}$ mass should be $0.4 < m_{\tau} < 1.2\sqrt{p_{\text{T}}/100}$~GeV. The upper limit on the mass window is constrained to lie between 1.2 and 4.0~GeV.
\item \textbf{Three prong, 0 $\pi^0$: } Three charged particles with mass $0.8 < m_{\tau} < 1.5$~GeV. The tracks are required to originate within $\Delta z<0.4$~cm of the same vertex.
\end{itemize}

% End of C&P

\section{Jets and b-tagging}
\label{sec:objects_jets}

\subsection{Jet energy corrections}
\label{sec:objects_jets_jec}

\section{Missing energy}
\label{sec:objects_met}

\subsection{Recoil corrections}
\label{sec:objects_met_recoilcorr}

