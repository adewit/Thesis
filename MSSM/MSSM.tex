\chapter{\texorpdfstring{Search for MSSM \AHtotautau}{Search for MSSM A/H -->tautau}}
\label{chap:mssm}
In this chapter the search for neutral Higgs bosons \PHiggs or \PHiggsps
decaying to a pair of \Pgt leptons will be discussed. The results presented
here correspond to those from an analysis peformed on a dataset collected
during the first half of the 2016 p--p running period of the \ac{LHC}. This dataset
corresponds to an integrated luminosity of 12.9 fb$^{-1}$, and the analysis is
detailed in reference \cite{CMS-PAS-HIG-16-037}. The \etau, \mutau, \tautau and \emu final
states of the di-\Pgt pair were studied. The results of this search are 
model--independent upper limits on $\sigma \times$ BR of \PHiggs or \PHiggsps 
production in the gluon fusion (gg$\phi$) and b-associated (bb$\phi$) production 
modes (see chapter \ref{chap:theory}) and decay into $\Pgt\Pgt$. 
In addition to the upper limits on cross--section times branching ratio of these two processes, 
the results are interpreted in MSSM benchmark scenarios.
A very similar analysis, not described in detail here,
was performed on a dataset corresponding to 2.3 fb$^{-1}$ collected at $\sqrt{s}=13$ TeV during the 2015 \ac{LHC} p--p running period.  
More information can be found in reference \cite{CMS-PAS-HIG-16-006}.

For the analysis described in this chapter, as well as the 2015 analysis not detailed
here, I was responsible for all stages of the analysis. This included optimisation 
of object selection and categorisation,
as well as studies of background methods for the \mutau, \etau and \tautau channels,
evaluation of systematic uncertainties in all four channels, and production of
the statistical results, including model interpretations.

\section{Datasets and MC samples}
\label{sec:mssm_datasets}
The dataset used for this analysis corresponds to an integrated 
luminosity of 12.9 fb$^{-1}$ collected at a centre--of--mass
energy of 13 TeV during the first half of
the 2016 p--p running period of the \ac{LHC}. %Section
%\ref{sec:mssm_combination} includes the combination of this
%dataset with a dataset collected during the 2015 p--p running
%period of the \ac{LHC}. This dataset corresponds to an integrated
%luminosity of 2.3 fb$^{-1}$ collected at $\sqrt{s} = 13$ TeV.

Signal and background events were generated using
various \ac{MC} event generators. Signal samples
for the $\Pg\Pg \rightarrow \phi$ and $\Pg\Pg \rightarrow b\bar{b}\phi$
production processes of $\PHiggs/\PHiggsps$ were generated using \texttt{PYTHIA 8.1} \cite{pythia81}.
These samples were produced for a range of masses between 90 GeV and 3.2 TeV.
This range is slightly wider than the range in which MSSM Higgs sector
benchmark computations are available, 90 GeV -- 2 TeV, and 
improves on the mass range of 90 GeV to 1 TeV that was studied during Run 1 in the context of BSM Higgs searches.
SM Higgs signal samples are used for the interpretation of the analysis results in 
MSSM benchmark scenarios. These samples are generated using \texttt{Powheg} \cite{powheg1,powheg2,powheg3}.

Single-top and \ttbar background samples are generated using \texttt{Powheg},
with \texttt{MadGraph5\_aMC@NLO} \cite{amcnlo} used for the generation of di--boson background
samples and part of the W+$\gamma$ background, which is included in the \emu channel only.
\Wjets and \Zll backgrounds are generated with the \texttt{MadGraph 5} \cite{madgraph}
matrix element generator. Both samples containing a mixture
of jet multiplicities (`inclusive' samples) and samples with 1,2,3 or 4 jets (`exclusive' samples)
are generated. These exclusive samples increase the available number of
background events with higher jet multiplicities. %note LO samples used because to get same number of effective
%events would need to generate more NLO samples
The `inclusive' and `exclusive' samples are reweighted
before they are combined so as to preserve the correct
fractions of events with each jet multiplicity.
\texttt{MadGraph 5} 
is also used to generate part of the W+$\gamma$ background
samples.

Parton showering and hadronisation are modelled using \texttt{PYTHIA 8.1} for all 
samples, and  minimum--bias events generated with \texttt{PYTHIA 8.1} are
added to all \ac{MC} samples to model additional interactions. A reweighting
is then applied to the \ac{MC} samples to make the pile--up distribution match
the pile--up distribution observed in data.

\section{Event selection and categorisation}
\label{sec:mssm_eventsel}
In this section an overview of the event selection,
and how it was optimised, is given. More detailed information
on how the objects used are reconstructed, 
and a more in--depth description of the identification criteria, 
is given in chapter \ref{chap:objects}.

\subsection{Pair selection and vetos}
\label{sec:mssm_eventsel_pairs}
The basis of the event selection is the selection of \mutau, \etau,
\tautau and \emu pairs by the requirements described
in the following subsections. After applying trigger
requirements, identification criteria and kinematic cuts
more than one possible candidate pair can exist. If this
is the case for an event, the pair is chosen as follows:
\begin{itemize}
\setlength{\itemsep}{-\baselineskip}
\item Prefer the pair with most isolated candidate 1 (muon for \mutau and \emu channels,
electron for \etau channel, either of the $\Pgt_h$s in \tautau channel). For electrons
and muons this means the object with the smallest relative isolation value is preferred, for $\Pgt_h$s the
object with raw MVA isolation value nearest to 1.
\item If the isolation of candidate 1 is the same in both pairs, prefer the pair with highest candidate 1 \pT.
\item If the \pT~of candidate 1 is the same in both pairs, prefer the pair with most isolated
candidate 2 (e for the \emu channel, the other $\Pgt_h$ for the \tautau channel, 
and the $\Pgt_h$ for \etau and \mutau channels).
\item If the isolation of candidate 2 is the same in both pairs, prefer the pair with highest candidate 2 \pT.
\end{itemize}

To prevent overlap with other channels, events are rejected
if there are muons and electrons, other than those in the selected pair,
with \pT~$>10$ GeV and passing loose ID and isolation requirements.
This also reduces backgrounds from \WZ events. 
In addition to this, \Zee events are 
reduced in the \etau channel by rejecting events
where an opposite--sign pair of electrons of \pT~$>15$ GeV
and passing very loose ID and isolation requirements can be formed. 
In the \mutau channel the contribution from \Zmm events
is reduced in a similar fashion. 

\subsection{\texorpdfstring{Event selection in the \mutau channel}{Event selection in the mu tau channel}}
\label{sec:mssm_eventsel_mt}
The first selection layer in the \mutau channel is a trigger
that only requires a muon at \ac{L1}. At the level of the \ac{HLT}, 
loose identification and isolation criteria are then applied to this muon.

The offline event selection proceeds to require an oppositely charged
muon and hadronically decaying tau, which are well--separated ($\Delta R > 0.5$).
The minimum \pT~of the muon is required to be 23 GeV, with $|\eta| < 2.1$. %due to trigger conditions
Additional ``medium'' identification requirements are placed on the muon, in addition
to requiring the muon to be consistent with having originated from the primary vertex, such that
the impact parameters $d_{xy} < 0.045$ cm and $d_{z} < 0.2$ cm. The muon is also required
to be isolated, with $I_{\text{rel}}^{\mu} < 0.15$, where the isolation variable is calculated using
a cone size of $\Delta R = 0.4$.

The hadronic tau needs to have a \pT~greater than 30 GeV, $|\eta|<2.3$,
is required to be reconstructed by the HPS algorithm and to pass the medium
working point of the MVA isolation discriminator. The hadronic tau also needs
to have originiated from the primary vertex and so the impact paramter $d_{z}$ is 
required to be less than 0.2 cm. Finally the hadronic tau is required
to satisfy the very loose working point of the anti--electron discriminator
and the tight working point of the anti--muon discriminator.

The selection on the muon's relative isolation is chosen after comparing
the performance of several isolation variables, using different relative isolation selections.
This was based on a simple \mutau channel
event selection as described above, only varying the isolation requirements.
The selection was optimised based on two figures of merit, both of
which consider \Ztautau as the signal and the other backgrounds
as background. The first figure of merit used is the number of signal events divided by the
square root of the number of background events, $S/\sqrt{B}$, after
the selection. The aim with this figure of merit is to choose
the isolation selection that maximises it. The second figure of merit
is the uncertainty interval on the best-fit value for a maximum--likelihood fit
to the \Ztautau signal strength, which should be minimised by the choice of
isolation selection.

The isolation variables studied include
the $\Delta\beta$ corrected isolation variable analogous
to the one given for electrons in equation \ref{eqn:electron_reliso} with 
isolation cone sizes in $\Delta R$ of 0.3 and 0.4. In addition
to this a very similar
variable, which does not just consider charged \ac{PF} hadrons but
also electrons and muons in the isolation sum, is considered. This
variable is also studied for both cone sizes of 0.3 and 0.4. Finally,
a relative isolation variable based solely on the \pT~of tracks from the primary
vertex,
\begin{equation}\label{eqn:reltrkiso}
I_{\text{trk}} = \frac{\Sigma p_{\text{T}}^{\text{tracks from PV within cone of 0.3 of the muon}}}{p_{\text{T}}^{\mu}}, \end{equation} 
is studied. The $S/\sqrt{B}$ for the different isolation variables is shown in figure \ref{fig:mssm_selection_mt_muons}a, with the
size of the error interval from the maximum--likelihood fit to the \Ztautau signal strength shown in 
\ref{fig:mssm_selection_mt_muons}b. From the fact that both figures of merit do not
change so much with varying isolation requirements we can see that this analysis
is not so sensitive to fake, nonisolated muons. The choice of isolation variable is therefore
driven mostly by simplicity and consistency with other \ac{CMS} analyses.

\begin{figure}[h!]
\begin{center}
\subfloat[$S/\sqrt{B}$]{\includegraphics[width=0.7\textwidth]{./MSSM/Figures/s_over_root_b_mt_m.pdf}}~\\
\subfloat[Size of error interval]{\includegraphics[width=0.7\textwidth]{./MSSM/Figures/mt_muons_iso_errint.pdf}}
\end{center}
\caption{(a) The $S/\sqrt{B}$ for \Ztautau signal and (b) the size of the error interval on 
the best--fit value of a maximum--likeihood fit to the \Ztautau signal strength,
for various selections of the $\Delta\beta$--corrected
isolation variable with a cone size of $\Delta R = 0.4$ (solid circles), 
the $\Delta\beta$--corrected isolation variable with a cone size of $\Delta R = 0.3$ (open circles),
the $\Delta\beta$--corrected isolation variable including electrons and muons in the isolation
sum, for a cone size of $\Delta R =0.4$ (upward facing triangles) and $\Delta R =0.3$ (downward facing triangles) and
the tracker--based relative isolation (solid squares).}
\label{fig:mssm_selection_mt_muons}
\end{figure}

A topological selection on the \mT~variable, 
introduced in section \ref{sec:hhh_selection_categories}, is made on 
events in the \mutau channel. This selection and the
choice of hadronic tau isolation working point provide two competing
effects and therefore they are optimised in a 2D--optimisation. 
The pre--fit expected upper limits on $\sigma\times$BR for both gg$\phi$ and bb$\phi$
production with decay into $\Pgt\Pgt$ at several mass points 
in the 90 GeV -- 3.2 TeV range are used as a figure of merit.
Because of the use of such a wide range of signal masses, it
is not possible to choose a selection that is optimal
for every mass point. This is illustrated in figure \ref{fig:mssm_selection_mt_taumt}:
figure \ref{fig:mssm_selection_mt_taumt}a shows the upper limit on $\sigma\times$BR for
the gluon fusion production process, at a mass $m_{\phi} = 160$ GeV, with
figure \ref{fig:mssm_selection_mt_taumt}b showing this upper limit at $m_{\phi} = 2.9$ TeV.
Figure \ref{fig:mssm_gradcuts_mt} shows the effect of gradually loosening the \mT~and hadronic
tau isolation working point selection on both the gg$\phi$ and bb$\phi$ production processes. 

\begin{figure}[h!]
\begin{center}
\subfloat[$m_{\phi} = 160$ GeV]{\includegraphics[width=0.5\textwidth]{./MSSM/Figures/optimisation_ggh_mt_160_range12.pdf}}
\subfloat[$m_{\phi} = 2.9$ TeV]{\includegraphics[width=0.5\textwidth]{./MSSM/Figures/optimisation_ggh_mt_2900_range12.pdf}}
%\subfloat[$m_{\phi} = 1600$ GeV]{\includegraphics[width=0.5\textwidth]{./MSSM/Figures/optimisation_ggh_mt_1600_range12.pdf}}
%\subfloat[$m_{\phi} = 2900$ GeV]{\includegraphics[width=0.5\textwidth]{./MSSM/Figures/optimisation_ggh_mt_2900_range12.pdf}}
\end{center}
\caption{Pre--fit expected limit on $\sigma\times$BR for the gg$\phi$ production process with decay into $\Pgt\Pgt$,
as a function of tau isolation working point (from very loose to very very tight) and
of \mT~selection from \mT$<20$ GeV to \mT$<40$ GeV, normalised to the best limit in the plane, indicated by the asterisk. This is shown
for (a) $m_{\phi}$ = 160 GeV and (b) $m_{\phi}$ = 2.9 TeV. The most optimal combination of \mT~selection and 
hadronic tau isolation working point varies, with looser selections preferred at higher mass.}
\label{fig:mssm_selection_mt_taumt}
\end{figure}
~
\begin{figure}[h!]
\begin{center}
\subfloat[gg$\phi$]{\includegraphics[width=0.5\textwidth]{./MSSM/Figures/mssm_gradcuts_ggH_mt.pdf}}
\subfloat[bb$\phi$]{\includegraphics[width=0.5\textwidth]{./MSSM/Figures/mssm_gradcuts_bbH_mt.pdf}}
\end{center}
\caption{Pre-fit expected limits on $\sigma \times$ BR in the \mutau channel for (a) gg$\phi$ and
(b) bb$\phi$ production, for increasingly looser \mT~selection and tau isolation working point, starting
at tight working point and \mT$<30$ GeV in green, medium working point and \mT$<40$ GeV in red,
loose working point and \mT$<50$ GeV in blue and very loose working point and \mT$<60$ GeV in black. Loosening
the tau isolation working point and \mT~selection gradually improves limits at higher mass, and
degrades them at low mass.}
\label{fig:mssm_gradcuts_mt}
\end{figure}

From these figures
we can observe that looser selections are preferred for higher masses. This behaviour can be
explained by considering the function of each of the selections. Tightening the hadronic
tau isolation working point reduces the selection of fake taus and increases the proportion
of backgrounds which contain real hadronic taus. The \mT~selection reduces
the \Wjets background. As the backgrounds are concentrated at the lower
end of the spectrum of mass--like discriminating variables, such tighter
selections are preferred where signal and backgrounds overlap. However, at higher
mass points the signal--sensitive range of the discriminating variable is 
low in backgrounds and there is more benefit to loosening the selection
to increase the signal efficiency than to keep the tighter selection
to reduce an already low background.
Using this, the selection was optimised for mass points of around 1 TeV, which leads to 
the use of the medium hadronic tau isolation working point and a selection on \mT$<40$ GeV.

The choice of the hadronic tau \pT~selection is driven by the effect on the low mass region:
raising the \pT~selection by 10 GeV from the minimum of 20 GeV does not affect the sensivity
of the analysis for high mass points, while recovering some of the loss of loosening 
tau isolation working point and \mT~cut. The reason for this is that when using \pT$>30$ GeV 
fewer background events are selected. The effect of the increased hadronic tau \pT~selection on the 
upper limits for the gg$\phi$ production process is shown
in figure \ref{fig:mssm_tauptcut}a, and on the bb$\phi$ production process
in figure \ref{fig:mssm_tauptcut}b. At low mass the improvement in the gg$\phi$ limits
obtained by increasing the \pT~selection is sizeable, for bb$\phi$ events the effect is much
less pronounced.

\begin{figure}[h!]
\begin{center}
\subfloat[gg$\phi$]{\includegraphics[width=0.5\textwidth]{./MSSM/Figures/mssm_optimisation_mediummt40_vs_tightmt30pt20_mt_ggH_remake.pdf}}
\subfloat[bb$\phi$]{\includegraphics[width=0.5\textwidth]{./MSSM/Figures/mssm_optimisation_mediummt40_vs_tightmt30pt20_mt_bbH_remake.pdf}}
\end{center}
\caption{Pre-fit expected limits on $\sigma\times$BR in the \mutau channel for (a) gg$\phi$ production and (b) bb$\phi$ production. The
limits when using the medium hadronic tau isolation working point and \mT$<40$ GeV selection using a minimum
hadronic tau \pT~selection of 20 GeV (red circles), 30 GeV (blue circles) and 40 GeV (black circles) are shown. The green
circles show the limits using tighter tau isolation working point and \mT~selection. The limits on
the gg$\phi$ production process improve at low masses when increasing the minimum hadronic tau \pT~selection,
the effect on the bb$\phi$ production process is less pronounced.}
\label{fig:mssm_tauptcut}
\end{figure}

\subsection{\texorpdfstring{Event selection in the \etau channel}{Event selection in the e tau channel}}
\label{sec:mssm_eventsel_et}
For selection of events in the \etau channel, the first step is
a trigger that only requires an electron at \ac{L1}. Loose ID and isolation
criteria are made on the electron at the \ac{HLT}.

The offline event selection proceeds to require an oppositely charged
electron and hadronically decaying tau, which are well--separated ($\Delta R > 0.5$).
The minimum \pT~of the electron is required to be 26 GeV, with $|\eta| < 2.1$. %due to trigger conditions
This electron has to be consistent with having originated from the primary vertex, which
is achieved by requiring the impact parameters to satisfy $d_{xy} < 0.045$ cm and $d_{z} < 0.2$ cm, and it
is required to pass additional identification criteria. A selection on the MVA identification, which
selects electrons from \Zee events with 80\% efficiency, is used. In addition the
electron needs to be isolated, $I_{\text{rel}}^{\Pe} < 0.1$, with the relative isolation variable calculated with a cone size of 0.3.

The hadronic tau is required to have a \pT~greater than 30 GeV, $|\eta|<2.3$,
and is required to be reconstructed by the HPS algorithm and to pass the medium
working point of the MVA isolation discriminator. It also needs
to have originiated from the primary vertex and so the impact paramter $d_{z}$ is 
required to be less than 0.2 cm. Finally, the hadronic tau must 
pass the tight working point of the anti--electron discriminator
and the loose working point of the anti--muon discriminator.

Analogously to the muons, it was found that the analysis is not highly sensitive
to the choice of electron isolation variable and selection.
A topological selection on the \mT~variable is also made in this channel to reject \Wjets events. 
The used value of \mT~$< 50$ is obtained by performing a 2D optimisation where the \mT~selection
and hadronic tau isolation working point are varied at the same time. The same considerations as
for the \mutau channel apply, as illustrated in figure \ref{fig:mssm_gradcuts_et}.
The choice of the hadronic tau \pT~selection of 30 GeV is
made to recover some of the sensitivity lost at low mass by using looser \mT~and hadronic
tau isolation selections. This is illustrated in figure \ref{fig:mssm_tauptcut_et}.

\begin{figure}[h!]
\begin{center}
\subfloat[gg$\phi$]{\includegraphics[width=0.5\textwidth]{./MSSM/Figures/mssm_gradcuts_ggH_et.pdf}}
\subfloat[bb$\phi$]{\includegraphics[width=0.5\textwidth]{./MSSM/Figures/mssm_gradcuts_bbH_et.pdf}}
\end{center}
\caption{Pre-fit expected limits on $\sigma \times$ BR in the \etau channel for (a) gg$\phi$ and
(b) bb$\phi$ production, for increasingly looser \mT~selection and tau isolation working point, starting
at tight working point and \mT$<40$ GeV in green, medium working point and \mT$<50$ GeV in red,
loose working point and \mT$<60$ GeV in blue and very loose working point and \mT$<70$ GeV in black. Loosening
the tau isolation working point and \mT~selection gradually improves limits at higher mass, and
degrades them at low mass.}
\label{fig:mssm_gradcuts_et}
\end{figure}
\begin{figure}[h!]
\begin{center}
\subfloat[gg$\phi$]{\includegraphics[width=0.5\textwidth]{./MSSM/Figures/mssm_optimisation_mediummt50_vs_tightmt40pt20_et_ggH_remake.pdf}}
\subfloat[bb$\phi$]{\includegraphics[width=0.5\textwidth]{./MSSM/Figures/mssm_optimisation_mediummt50_vs_tightmt40pt20_et_bbH_remake.pdf}}
\end{center}
\caption{Pre-fit expected limits on $\sigma\times$BR in the \etau channel for (a) gg$\phi$ production and (b) bb$\phi$ production. The
limits using the medium hadronic tau isolation working point and \mT$<40$ GeV selection using a minimum
hadronic tau \pT~cut of 20 GeV (red circles), 30 GeV (blue circles) and 40 GeV (black circles) are shown. The green
circles show the limits using tighter tau isolation working point and \mT~selection. The limits on
the gg$\phi$ production process improve at low masses when increasing the hadronic tau \pT~cut,
the effect on the bb$\phi$ production process is less pronounced.}
\label{fig:mssm_tauptcut_et}
\end{figure}

\subsection{\texorpdfstring{Event selection in the \tautau channel}{Event selection in the tau tau channel}}
\label{sec:mssm_eventsel_tt}
Selection of events in the \tautau channel starts with a trigger
that requires two hadronic taus at \ac{L1}. At the level of the \ac{HLT}, loose identification
and isolation criteria are applied to these hadronic taus.

The offline event selection continues by requiring both hadronic taus to be oppositely charged, separated
by $\Delta R > 0.5$, have \pT~of at least 40 GeV and satisfy $|\eta|<2.1$. Both taus need
to pass decay mode finding and have originated from the primary vertex, with the impact parameter $d_{z} < 0.2$ cm. 
Both taus need to pass the tight working point of the isolation discriminator, the very loose 
working point of the anti--electron discriminator and the loose working point of the anti--muon discriminator. 
The tight working point of the isolation discriminator is chosen to optimise the sensitivity
of this channel for higher masses, while retaining the second best senstivity in the intermediate
mass range, as seen in figure \ref{fig:mssm_tauid_tt} for the gg$\phi$ process.

\begin{figure}[h!]
\begin{center}
\includegraphics[width=0.5\textwidth]{./MSSM/Figures/ggh_tt_isolation.png}
\end{center}
\caption{Expected upper limits for gg$\phi$ process in the \tautau channel, using the 
loose (green), medium (red), tight (blue, very tight (black) and very very tight (gold)
working point.}
\label{fig:mssm_tauid_tt}
\end{figure}

\subsection{\texorpdfstring{Event selection in the \emu channel}{Event selection in the e mu channel}}
\label{sec:mssm_eventsel_em}
Events in the \emu channel are first selected by two triggers
that both require an electron and muon at \ac{L1} and apply loose ID and
isolation requirements to these objects at \ac{HLT}. The \pT~requirements on
the electron and muon are different in the two triggers, thus using the combination of the two
improves the signal efficiency.

The offline requirements on the \emu pair are that the electron 
and muon are oppositely charged and separated by $\Delta R >$ 0.3. The 
electron needs to have a minimum \pT~of 13 GeV and have $|\eta|< 2.5$. As
in the \etau channel the electron's impact parameters have to 
satisfy $d_{xy} < 0.045$ cm and $d_{z}<0.2$ cm. A selection is
made on the electron identification MVA, using a working point that 
is 80\% efficient for electrons in \Zee decays. The electron
also needs to be isolated, $I_{\text{rel}}^{\Pe}<0.15$, where the relative isolation variable is calculated
with a cone size of 0.3.

The muon needs to have a \pT~of at least 10 GeV, $|\eta|<2.4$ and 
its impact parameters need to be $d_{xy}<0.045$ cm and $d_{z}<0.2$ cm.
In addition to the kinematic requirements the muon must
pass the medium identification criteria and be isolated with $I_{\text{rel}}^{\Pgm} < 0.2$,
where the relative isolation variable is calculated using a cone size of 0.4.

Because a combination of triggers is used, one with electron \pT~$>$ 23 GeV and muon \pT~$>8$ GeV
and one with muon \pT~$>$ 23 GeV and electron \pT~$>$ 13 GeV, the object firing
the higher \pT~trigger leg needs to have a transverse momentum
of at least 24 GeV on top of the requirements already discussed. This avoids
events in the lower part of the trigger turn-on curve being selected.

In addition to the selection described so far,
 a topological selection is made on the $D_{\zeta}$ variable \cite{cdf-dzeta},
\begin{align}\label{eqn:mssm_em_dzeta}
D_{\zeta} = P_{\zeta} - 1.85 P_{\zeta}^{\text{vis}}, \notag \\
\text{where } P_{\zeta} = (\vec{p}_{\text{T},1}^{\text{  vis}} + \vec{p}_{\text{T},2}^{\text{  vis}} + E_{\text{T}}^{\text{miss}})\frac{\vec{\zeta}}{|\vec{\zeta}|}, \notag \\
\text{and } P_{\zeta}^{\text{vis}} = (\vec{p}_{\text{T},1}^{\text{  vis}}+\vec{p}_{\text{T},2}^{\text{  vis}})\frac{\vec{\zeta}}{|\vec{\zeta}|}.
\end{align}
This means $D_{\zeta}$ is defined as the projection of the transverse momenta of the visible tau decay products plus missing energy \pT+\MET~onto 
the axis $\vec{\zeta}$, minus the projection of only the transverse momenta of the visible tau decay products onto this axis.
The axis $\vec{\zeta}$ is the axis that bisects
the directions $\vec{p}_{\text{T,1}}^{\text{  vis}}$ and $\vec{p}_{\text{T,2}}^{\text{  vis}}$
of the visible decay products in the transverse plane. This definition is illustrated in figure
\ref{fig:mssm_dzeta}a, with figure \ref{fig:mssm_dzeta}b showing the $D_{\zeta}$ distribution
for \emu events. The used selection, $D_{\zeta} > -20$ GeV rejects $\ttbar$, \Wjets and diboson
events. 
%The momentum of the neutrinos should not be in the opposite direction of the sum of the visible decay products
%as the neutrinos and visible tau decay products travel at small angles from the original tau.

\begin{figure}[h!]
\begin{center}
\subfloat[Reconstruction of $D_{\zeta}$]{\includegraphics[width=0.5\textwidth]{./MSSM/Figures/PZeta.pdf}}
\subfloat[$D_{\zeta}$ distribution in \emu channel]{\includegraphics[width=0.5\textwidth]{./MSSM/Figures/pzeta_inclusive_em_2016.pdf}}
\end{center}
\caption{(a) reconstruction of $D_{\zeta}$ \cite{cdf-dzeta} and (b) $D_{\zeta}$ distribution in the 
\emu channel\cite{CMS-PAS-HIG-16-037}.}
\label{fig:mssm_dzeta}
\end{figure}

\subsection{Categorisation}
\label{sec:mssm_eventsel_categories}
As the analysis targets both the gluon fusion
and the b--associated production modes of MSSM Higgs
bosons, two exclusive categories are defined based on the 
number of b--tagged jets. A jet is considered b--tagged
if it passes the medium working point of the \ac{CSV}v2 discriminator. 
The categories are defined as:
\begin{itemize}
\setlength{\itemsep}{-\baselineskip}
\item \textbf{No b-tag}: 0 b--tagged jets. This category targets the gg$\phi$ production mode but is also sensitive to some of the bb$\phi$ signal.
\item \textbf{B-tag}: At least 1 b--tagged jet, at most 1 
jet with \pT$>30$ GeV. Due to the different phase space considered for
b--tagging than for default jet reconstruction this requirement does not necessarily
mean there will be only one b--tagged jet in the event. The requirement reduces the sizeable \ttbar
background.
\end{itemize}

\begin{figure}[h!]
\begin{center}
\subfloat[Number of jets]{\includegraphics[width=0.5\textwidth]{./MSSM/Figures/n_jets_inclusive_mt_2016_log.pdf}}
\subfloat[Number of b--tagged jets]{\includegraphics[width=0.5\textwidth]{./MSSM/Figures/n_bjets_inclusive_mt_2016_log.pdf}}
\end{center}
\caption{(a) Number of jets and (b) number of b--tagged jets in the \mutau channel. The gluon fusion signal (dashed
blue line) and b--associated signal (dashed purple line) are overlaid on the expected background distribution. The signals
are normalised to 100 times their cross--section times branching ratio at \mA=1 TeV and \tanb=50 in the $m_{\text{h}}^{\text{mod+}}$
scenario. We can see how the signal--sensitive bins motivate the choice of categories.}
\label{fig:mssm_cats_mt}
\end{figure}


Figure \ref{fig:mssm_cats_mt}a shows the number of jets 
in the \mutau channel, with figure \ref{fig:mssm_cats_mt}b showing
the number of b--tagged jets in that channel. The gluon fusion
and b--associated signals are overlaid on these distributions, normalised
to 100 times their cross--section times branching ratio at \mA=1 TeV and 
\tanb=50 in the $m_{\text{h}}^{\text{mod+}}$ scenario. From the distribution of the
gluon fusion signal in figure \ref{fig:mssm_cats_mt}b we can 
see that the vast majority of these signal events do not have any b-tagged jets, which
motivates the choice of the no b-tag category to target this production mode. From
the b--associated signal in the same figure we also see that the majority of bb$\phi$
signal events do not have any b-tagged jets either: either the b-jets in these events 
are too soft, or they do not pass the b-tagging requirements. This means the no b-tag 
category is also sensitive to some of the bb$\phi$ signal. The other observation we 
can make from the bb$\phi$ signal in figure \ref{fig:mssm_cats_mt}b is that there is not
such a large proportion of events with more than 1 b-tagged jet. From figure \ref{fig:mssm_cats_mt}a
we see that the \ttbar background starts to become more sizeable for events with more than
one jet. The definition of the b-tag category is therefore chosen with the inclusion
of the jet veto, such that the region where the \ttbar background becomes larger is excluded, 
while retaining a large proportion of the bb$\phi$ signal.

\section{\ac{MC} to data correction factors}
\label{sec:mssm_mccorrs}
Because \ac{MC} samples are used for the signal prediction, and 
to estimate some of the backgrounds, it is important to correct for possible 
mis-modelling with respect to the data. Dedicated control regions in
data are used to derive the \ac{MC} to data correction factors
that are applied to the \ac{MC} samples.
\subsubsection*{Tracking efficiency}
A discrepancy between the track reconstruction efficiency
in data and \ac{MC} for electrons and muons was found. It is corrected for using $\eta$--dependent
scale factors.% \cite{CMS-PAS-HIG-16-037}.
\subsubsection*{Electron, muon and tau ID and isolation}
Identification and isolation efficiencies in data and \ac{MC} are measured
for electrons, muons and hadronic taus. A tag--and--probe
method using \Zeenog (\Zmmnog) events is used to measure
the efficiencies for electrons (muons). The hadronic tau identification
and isolation efficiency is measured using a tag--and--probe method
making use of $Z\rightarrow\Pgt\Pgt\rightarrow\Pgm\Pgt_{h}$ events.
The measured efficiencies are used to construct scale factors to be
applied to the \ac{MC} samples as $SF = \frac{\epsilon_{\text{Data}}}{\epsilon_{\text{MC}}}$.
\subsubsection*{Trigger efficiency}
The electron, muon and hadronic tau trigger efficiencies are also measured
using tag--and--probe methods with the types of events as described for the identification
and isolation efficiencies. Because no trigger simulation is applied to
the \ac{MC} samples, the efficiency $\epsilon_{\text{Data}}$ is simply applied
to the \ac{MC} events. Because two electron/muon cross--triggers are 
used to select events in the \emu channel, the efficiencies of the different
cross trigger legs need to be combined. With one of the cross triggers having a
minimum \pT~of
23 GeV on the muon leg and 12 GeV on the electron leg, and the
other having a minimum \pT~of 23 GeV on the electron leg and 8 GeV on the muon
leg, the combined efficiency of the two triggers becomes:
\begin{equation}\label{eqn:mssm_em_trigeff}
\begin{split}
\epsilon_{\text{data}}  = \epsilon_{\text{data}}(\text{Mu23})\cdot\epsilon_{\text{data}}(\text{Ele12}) + \epsilon_{\text{data}}(\text{Mu8})\cdot\epsilon_{\text{data}}(\text{Ele23})~\\ - \epsilon_{\text{data}}(\text{Mu23})\cdot\epsilon_{\text{data}}(\text{Ele23}).
\end{split}
\end{equation}
\subsubsection*{$\Pe\rightarrow\Pgt_{h}$ fake rate and $\Pgm\rightarrow\Pgt_{h}$ fake rate}
The $\Pe\rightarrow\Pgt_{h}$ and $\Pgm\rightarrow\Pgt_{h}$ fake rates,
after applying the anti--electron and anti--muon discriminators, are measured
using a tag--and--probe method with \Zeenog and \Zmmnog events. Scale
factors, applied to \ac{MC} events where the tau is faked by an electron or muon,
are derived as the ratio between the fake rate in data and the fake rate in \ac{MC} events.
\subsubsection*{\MET~recoil corrections}
Differences in \MET~resolution and response between data and \ac{MC} 
are accounted for by the application of recoil corrections
to signal, \Wjets and \Ztautau events. More detail
about these corrections is given in section \ref{sec:objects_met_recoilcorr}.
\subsubsection*{B-tag scale factors}
To correct for the difference in b--tagging efficiency
and light jet mis--tagging rates between data and \ac{MC} events,
\pT- and $\eta$-dependent scale factors are derived as 
described in reference \cite{cms-btag-run2}. They 
are applied using the promote--demote method
as outlined in equation \ref{eqn:promotedemote}.
\subsubsection*{Top-quark \pT~reweighting}
A reweighting which was derived during \ac{LHC} Run-1
to better match the top quark \pT~distribution in \ac{MC}
to that observed in data is applied to \ttbar events. Despite
the fact that the correction was derived during Run-1, it still
improves the data/\ac{MC} agreement in a \ttbar enriched
control region.
\subsubsection*{Drell-Yan shape reweighting}
As the \ac{MC} samples used for the \Ztautau estimate
do not model the data well for events with high di-lepton
mass and high Z \pT, a reweighting is applied to correct for this.
These weights are derived in bins of $\PZ/\Pphotonx$ mass and \pT~
using \Zmm events in data, in such a way that they do not
change the overall Drell-Yan normalisation. The weights
are then applied to the \Ztautau and \Zellell background processes.

\section{Discriminating variable}
\label{sec:mssm_discrvar}
The discriminating variable used for signal extraction is the total transverse mass,
\begin{equation}\label{eqn:mttot}
m_{\text{T}}^{\text{tot}} = \sqrt{(m_{\text{T}}(E_{\text{T}}^{\text{miss}},\tau_1^{\text{vis}}))^2+
(m_{\text{T}}(E_{\text{T}}^{\text{miss}},\tau_2^{\text{vis}}))^2 + (m_{\text{T}}(\Pgt_1^{\text{vis}},\Pgt_2^{\text{vis}}))^2}.
\end{equation}

$m_{\text{T}}(1,2)$ is defined as,
\begin{equation}\label{eqn:mttot_12}
m_{\text{T}}(1,2) = \sqrt{2p_{\text{T},1}p_{\text{T},2}(1-\cos{(\Delta\phi(1,2))})}.
\end{equation}

This means that $m_{\text{T}}(E_{\text{T}}^{\text{miss}},\Pgt_1^{\text{vis}})$ is equivalent
to the \mT~defined for the \etau and \mutau channels in equation \ref{eqn:hhh_selection_mt}.
The \mTtot~variable provides good separation between signal and QCD multi--jet events
in the \etau, \mutau and \tautau channels, and between
signal and \ttbar events in the \emu channel.

\section{Background estimation}
\label{sec:mssm_bkgs}
To estimate the backgrounds to the analysis a mixture of data--driven
methods and \ac{MC} samples is employed. The methods used for background 
estimation will be described in this section.

\subsection{Generator matching}
\label{sec:mssm_bkgs_genmatch}
For backgrounds estimated from \ac{MC}, it is possible
that it is more correct to split a background process
into different sub--processes that should be treated separately in the fit to data.
Taking the \mutau 
channel as an example, a sample of Drell--Yan events
will contain \Ztautau events, where one of the taus decays
hadronically and the other tau decays to a muon, but also
\Zmm events where one of the muons fakes a tau, or with an 
additional jet in the event faking a hadronic tau and one of the
muons not being properly reconstructed. Similarly, \ttbar background
events can be split into those with genuine taus, and those
where a jet mimics a hadronic tau. In such cases dividing
events from the same production process into sub--samples
based on generator information allows for a more correct
treatment of systematic uncertainties.

To determine the generator--level particle
that a reconstructed electron, muon, or hadronic tau
originates from, reconstructed objects are matched
to a set of generator level objects within a cone of $\Delta R = 0.2$.
Five categories of generator-level object are considered for matching:
prompt electrons and muons,
that is electrons and muons not originating from a hadron, tau or muon decay; electrons and muons
from tau decays; and generator-level hadronic taus. These generator-level
hadronic taus are rebuilt by summing the four--momenta
of the visible decay products of the generator-level tau.

If there is at least one generator-level object
within a cone of $\Delta R = 0.2$ around the direction of the reconstructed
object, the generator-level object nearest the reconstructed particle is chosen
as the one the reconstructed particle is matched to. 
If there are no 
generator-level particles in the cone at all it is said to 
have originated from pile--up or a jet at generator level.

The generator--level object type matched to a 
reconstructed particle is used to perform the splitting
of background samples, where this is used it is indicated.

\subsection{\texorpdfstring{\Ztautau}{Z to tau tau}}
\label{sec:mssm_bkgs_ztt}
For all channels, both shape and normalisation of the \Ztautau background 
are estimated from the Drell-Yan
\ac{MC} samples described in section \ref{sec:mssm_datasets}.
For the \mutau and \etau channels, events from these samples 
in which the reconstructed hadronic tau is matched to 
a generator-level hadronic tau are considered part of the \Ztautau
background. In the \tautau channel both reconstructed
hadronic taus are required to be matched to a generator-level hadronic tau, and
in the \emu channel the \Ztautau component of the Drell-Yan background 
is taken as those events where the reconstructed electron is not matched to
a prompt electron at generator level and the reconstructed
muon is not matched to a prompt muon. 
Events in the Drell-Yan samples selected in the event selection
but not satisfying the generator matching requirements are considered
as the, much smaller, \Zll background.

To correct and constrain the \Ztautau normalisation in the
b-tag and no b-tag categories of all channels, \Zmm control
regions are considered in a simultaneous fit with the signal region. More detail is given in 
section \ref{sec:mssm_sigext_ctrl}.

\subsection{\texorpdfstring{\Wjets and QCD in the \etau and \mutau channels}{W+jets and QCD in the e tau and mu tau channels}}
\label{sec:mssm_bkgs_mtet_wjetsqcd}
A data--driven approach is used for the estimation of
both the \Wjets and QCD backgrounds in the \etau and \mutau channels. 
The estimates of the normalisations of the two backgrounds are tied
together due to the presence of some QCD contamination in the \Wjets--dominated
control region. The shape of the \Wjets background is taken
from the \ac{MC} samples. The QCD shape taken from same--sign
data, that is events selected in observed data with the opposite charge requirement
on the di-$\Pgt$ pair inverted but otherwise identical cuts to the signal region. Contributions
from other backgrounds in this region are subtracted.

\subsubsection{\texorpdfstring{\Wjets normalisation}{W+jets normalisation}}
\label{sec:mssm_bkgs_mtet_wjetsnorm}
The \Wjets normalisation is derived using a high-\mT~ control region, where
selected events are required to satisfy \mT$>70$ GeV. The 
\Wjets
contribution in this region is enhanced, however there is still
some contribution from QCD events in this region. This is visible in 
figure \ref{fig:mssm_bkgs_wjets_mutau_mt} which shows the \mT~distribution in the
\mutau channel, indicating the signal region and the high-\mT~control region. 

\begin{figure}[h!]
\begin{center}
\includegraphics[width=0.5\textwidth]{./MSSM/Figures/CMS-PAS-HIG-16-037_Figure_002.pdf}
\end{center}
\caption{Distribution of the transverse mass \mT~in the \mutau channel, indicating
the signal region and the high \mT~control region. This distribution is plotted
before dividing the events into categories \cite{CMS-PAS-HIG-16-037}.}
\label{fig:mssm_bkgs_wjets_mutau_mt}
\end{figure}

Because of this contribution from QCD events in the high \mT~region we
have, for \mT$>70$ GeV:
\begin{equation}\label{eqn:wjets_ss_norm}
\begin{split}
N_{\text{data}}^{\text{SS, high } m_{\text{T}}} - N_{\text{other}}^{\text{SS,
 high } m_{\text{T}}} & =
N_{\text{QCD}}^{\text{SS, high } m_{\text{T}}} + N_{\text{W}}^{\text{SS, high } m_{\text{T}}} ~\\
N_{\text{data}}^{\text{OS, high } m_{\text{T}}} - N_{\text{other}}^{\text{OS,
 high } m_{\text{T}}} & = N_{\text{QCD}}^{\text{OS, high } m_{\text{T}}} +
N_{\text{W}}^{\text{OS, high } m_{\text{T}}} \\
& = R_{\text{QCD}}^{\text{OS/SS}}\cdot N_{\text{QCD}}^{\text{SS, high } m_{\text{T}}} +
R_{\text{W}}^{\text{OS/SS}} \cdot N_{\text{W}}^{\text{SS, high } m_{\text{T}}} ~\\
\Rightarrow N_{\text{W}}^{\text{SS, high } m_{\text{T}}}  &= \frac{N_{\text{data}}^{\text{OS,
 high } m_{\text{T}}}  - N_{\text{other}}^{\text{OS, high } m_{\text{T}}}  -
R_{\text{QCD}}^{\text{OS/SS}}\cdot(N_{\text{data}}^{\text{SS, high } m_{\text{T}}}  -
N_{\text{other}}^{\text{SS, high } m_{\text{T}}} )}{R_{\text{W}}^{\text{OS/SS}} -
R_{\text{QCD}}^{\text{OS/SS}}} ,
\end{split}
\end{equation}
where $R_{\text{W}}^{\text{OS/SS}}$ is the ratio between opposite-sign and same-sign \Wjets events
and $R_{\text{QCD}}^{\text{OS/SS}}$ the ratio between opposite-sign and same-sign QCD events. Using 
equations \ref{eqn:wjets_ss_norm}, the number of \Wjets events in the
opposite-sign, high \mT~region is given by $R_{\text{W}}^{\text{OS/SS}}\cdot N_{\text{W}}^{\text{SS, high } m_{T}}$. 
This is extrapolated to the number of \Wjets events in the signal region at low \mT~as:
\begin{equation}\label{eqn:wjets_os_norm}
N_{\text{W}}^{\text{OS, low} m_{T}} = \frac{N_{\text{W,MC}}^{\text{OS, low} m_{T}}}{N_{\text{W,MC}}^{\text{OS, high} m_{T}}}\cdot R_{\text{W}}^{\text{OS/SS}} \cdot N_{\text{W}}^{\text{SS, high }m_{T}},
\end{equation}
which means the estimate of the number of \Wjets events in the opposite-sign, high \mT, region
is multiplied by a high \mT-to-low \mT~extrapolation factor determined from the \ac{MC} samples.

The use of the method presented relies on knowledge of $R_{\text{W}}^{\text{OS/SS}}$,
$R_{\text{QCD}}^{\text{OS/SS}}$, and on these two ratios not being too similar to each other. 
The ratio $R_{\text{QCD}}^{\text{OS/SS}}$ is measured in an anti--isolated
control region in data; this measurement is described in section \ref{sec:mssm_bkgs_etmt_qcdosss}. $R_{\text{W}}^{\text{OS/SS}}$ is 
taken from the \Wjets \ac{MC} samples, and is found to be between 4 and 5, while $R_{\text{QCD}}^{\text{OS/SS}}$
is close to 1.

The method described so far works well in the no b-tag categories, but in the b-tag category of both channels
the \ttbar background dominates. This is illustrated in figure \ref{fig:bkgs_highmtctrl}, where
the high-\mT~region in the no b-tag category (figure \ref{fig:bkgs_highmtctrl}a) is compared with the high-\mT~region in the 
b-tag category (figure \ref{fig:bkgs_highmtctrl}b) of the \etau channel. It is clear that the \ttbar 
background is much larger than the \Wjets background in the b-tag high-\mT~control region.
For this reason the estimate of the number
of \Wjets events in this category is made with a relaxed category selection where the b-tagging
requirement itself is removed, but the jet requirements still stand. Events with at least one 
jet with \pT$>20$ GeV and with $|\eta|<2.4$, but at most one jet with \pT$>30$ GeV and $|\eta|<4.7$, are therefore selected. The final \Wjets estimate in the b-tag category signal
region is determined by the estimate given by the number of \Wjets events
estimated in the 1-jet selection, multiplied by an extrapolation factor 
$\frac{N_{\text{W,MC}}^{\text{OS,low } m_{T},\text{b-tag category}}}{N_{\text{W,MC}}^{\text{OS, low }m_{\text{T}},\text{1-jet selection}}}$.
\begin{figure}[h!]
\begin{center}
\subfloat[No b-tag]{\includegraphics[width=0.5\textwidth]{./MSSM/Figures/htt_et_10_shapes_postfit.pdf}}
\subfloat[B-tag]{\includegraphics[width=0.5\textwidth]{./MSSM/Figures/htt_et_13_shapes_postfit.pdf}}
\end{center}
\caption{High-\mT~control region in the (a) no b-tag and (b) b-tag categories of the \etau
channel. In the no b-tag category the \Wjets background, drawn together with the small
 di-boson plus single-top backgrounds as the `electroweak' background component, dominates. In the b-tag
category the \ttbar background is larger than the \Wjets background.}
\label{fig:bkgs_highmtctrl}
\end{figure}

\subsubsection{QCD OS/SS ratio}
\label{sec:mssm_bkgs_etmt_qcdosss}
The ratio of opposite--sign to same--sign QCD events, $R_{\text{QCD}}^{\text{OS/SS}}$ is measured
in two anti-isolated sidebands by performing a fit to the distribution of the
visible mass of the di-tau pair in 
opposite-sign events. The QCD template, which is taken from same--sign events 
assuming $R_{\text{QCD}}^{\text{OS/SS}} = 1$ is treated as the signal, and a binned maximum
likelihood fit to the QCD signal strength is performed while allowing the other
backgrounds to float within reasonable normalisation uncertainties. The
resulting best fit value of the signal strengh parameter is taken as $R_{\text{QCD}}^{\text{OS/SS}}$.
Of the two sidebands used, one is nearer the signal region ($0.15<I_{\text{rel}}^{\Pgm}<0.25$ for the \mutau channel and $0.1<I_{\text{rel}}^{\Pe}<0.2$ for the \etau channel)
and one further away ($0.25<I_{\text{rel}}^{\Pgm}<0.5$ for the \mutau channel and $0.2<I_{\text{rel}}^{\Pe}<0.5$ for the \etau channel). The results of the fits,
performed before the categorisation of events into the b-tag and no b-tag categories, are shown in figures \ref{fig:mssm_qcdosss_mtnear}--\ref{fig:mssm_qcdosss_mtfar}
for the \mutau channel and in figures \ref{fig:mssm_qcdosss_etnear}--\ref{fig:mssm_qcdosss_etfar} for the \etau channel.
$R_{\text{QCD}}^{\text{OS/SS}}$ is found to be consistent between the two sidebands within 10\% (1\%) in the \etau (\mutau) channel.
Adding this in quadrature to the uncertainty on the fit in the sidebands nearest the signal region we find
an overall uncertainty of 12\% in the \etau channel and 4\% in the \mutau channel.
When performing the same fits in the no b-tag and b-tag categories to determine differences between
this pre-categorisation opposite--sign to same--sign ratio, and the ratio as measured in categories it is found that 
the fit is not stable in the b-tag category due to the number of events in this region being too small.
The uncertainty on the ratio is increased for the b-tag category to cover the differences.
The uncertainties are already large enough to cover differences between the pre-categorisation $R_{\text{QCD}}^{\text{OS/SS}}$ and the
fits as performed in the no b-tag category. Therefore $R_{\text{QCD}}^{\text{OS/SS}}$ in the \mutau channel is taken to be
1.18 in both the b-tag and the no b-tag category, with a 60\% uncertainty in the b-tag category and a 4\% uncertainty in the no b-tag category.
In the \etau channel $R_{\text{QCD}}^{\text{OS/SS}}$ is taken to be 1.02 with a 60\% uncertainty in the b-tag category and a 12\% uncertainty
in the no b-tag category.

\begin{figure}[h!]
\begin{center}
\subfloat[Pre-fit distribution]{\includegraphics[width=0.5\textwidth]{./MSSM/Figures/qcdosss_near_mt_1_prefit.png}}
\subfloat[Post-fit distribution]{\includegraphics[width=0.5\textwidth]{./MSSM/Figures/qcdosss_near_mt_1_postfit.png}}
\end{center}
\caption{The visible mass distribution of the di-$\tau$ pair in the \mutau channel for the sideband near the signal region, (a) before
the maximum likelihood fit to the QCD signal strenght and (b) after the fit. The resulting fitted signal strength parameter
is 1.180$\pm$0.044.}
\label{fig:mssm_qcdosss_mtnear}
\end{figure}

\begin{figure}[h!]
\begin{center}
\subfloat[Pre-fit distribution]{\includegraphics[width=0.5\textwidth]{./MSSM/Figures/qcdosss_far_mt_1_prefit.png}}
\subfloat[Post-fit distribution]{\includegraphics[width=0.5\textwidth]{./MSSM/Figures/qcdosss_far_mt_1_postfit.png}}
\end{center}
\caption{The visible mass distribution of the di-$\tau$ pair in the \mutau channel for the sideband further away from the signal region, (a) before
the maximum likelihood fit to the QCD signal strenght and (b) after the fit. The resulting fitted signal strength parameter
is 1.184$\pm$0.035.}
\label{fig:mssm_qcdosss_mtfar}
\end{figure}

\begin{figure}[h!]
\begin{center}
\subfloat[Pre-fit distribution]{\includegraphics[width=0.5\textwidth]{./MSSM/Figures/qcdosss_near_et_1_prefit.png}}
\subfloat[Post-fit distribution]{\includegraphics[width=0.5\textwidth]{./MSSM/Figures/qcdosss_near_et_1_postfit.png}}
\end{center}
\caption{The visible mass distribution of the di-$\tau$ pair in the \etau channel for the sideband near the signal region, (a) before
the maximum likelihood fit to the QCD signal strenght and (b) after the fit. The resulting fitted signal strength parameter
is 1.015$\pm$0.059.}
\label{fig:mssm_qcdosss_etnear}
\end{figure}

\begin{figure}[h!]
\begin{center}
\subfloat[Pre-fit distribution]{\includegraphics[width=0.5\textwidth]{./MSSM/Figures/qcdosss_far_et_1_prefit.png}}
\subfloat[Post-fit distribution]{\includegraphics[width=0.5\textwidth]{./MSSM/Figures/qcdosss_far_et_1_postfit.png}}
\end{center}
\caption{The visible mass distribution of the di-$\tau$ pair in the \etau channel for the sideband further away from the signal region, (a) before
the maximum likelihood fit to the QCD signal strenght and (b) after the fit. The resulting fitted signal strength parameter
is 1.127$\pm$0.045.}
\label{fig:mssm_qcdosss_etfar}
\end{figure}

\subsubsection{QCD normalisation}
\label{sec:mssm_bkgs_etmt_qcdnorm}
The QCD normalisation is estimated by inverting the opposite-sign requirement
of the di-tau pair in the signal region. The yield is taken from the same--sign
region with otherwise identical cuts to the opposite--sign region. The contributions
from other backgrounds in this region are subtracted to give an estimate of the number
of QCD events in the same--sign region. For all backgrounds apart from \Wjets the
yields to subtract are estimated using \ac{MC} samples. The number of \Wjets events
expected in this region is given by
\begin{equation}\label{eqn:wjets_qcdsub}
N_{\text{W,SS low }m_{\text{T}}} = \frac{N_{\text{MC,SS low }m_{\text{T}}}}{N_{\text{MC,SS high }m_{\text{T}}}}N_{\text{W,SS high }m_{\text{T}}}.
\end{equation}
Because opposite--sign and same--sign QCD events do not
necessarily appear in equal amounts the number of QCD
events obtained by subtracting other backgrounds from the
observed number of events in the same--sign region is multiplied
by $R_{\text{QCD}}^{\text{OS/SS}}$ as derived in section \ref{sec:mssm_bkgs_etmt_qcdosss} 
to obtain an estimate of the number of QCD events in the signal region.

\subsubsection{Control regions in the fit}
\label{sec:mssm_bkgs_etmt_ctrl}
Several control regions in data are used for the estimation
of the QCD and \Wjets backgrounds: the same--sign and 
opposite--sign high \mT~regions, as well as the same--sign
low \mT~region. This leads to three control regions used 
in each category of both the \mutau and \etau channels to determine
the initial estimate of the background normalisations.
These control regions are included in a simultaneous fit
with the signal regions to obtain the final results. More detail
can be found in section \ref{sec:mssm_sigext_ctrl}.


\subsection{\texorpdfstring{QCD in the \tautau and \emu channels}{QCD in the tautau and emu channels}}
\label{sec:mssm_bkgs_qcd}
This section describes the QCD background
estimation in the \tautau and \emu channels, which takes a slightly
different form than in the \mutau and \etau channels.

\subsubsection{\texorpdfstring{\tautau channel}{tau tau channel}}
\label{sec:mssm_bkgs_qcd_tt}
In the \tautau channel QCD events constitute by far the dominant background.
The normalisation and shape 
of this background are estimated from a sideband with loosened 
isolation requirements with respect to the signal region. 
In the no b-tag category the sideband used is determined
by having the tau with highest \pT~pass the tight working
point of the tau ID discriminator, as in the nominal selection.
The other hadronic tau in the pair should not pass the tight working point,
but should pass the medium working point. This sideband is 
chosen as it is as close to the signal region as possible, and 
this should minimise biases in the shape of the total transverse mass
distribution. Other backgrounds
in this sideband are subtracted from the observation to give
the QCD estimate. Differences in normalisation due to
the use of a loosened sideband are corrected for by loose-to-tight isolation
scale factor. This correction factor is measured as
\begin{equation}\label{eqn:tautau_qcd}
R_{QCD}^{\text{loose}\rightarrow\text{tight}} = \frac{N_{\text{obs}}^{SS,\text{nominal isolation}}-N_{\text{other bkgs}}^{SS,\text{loosened isolation}}}{N_{\text{obs}}^{SS,\text{loosened isolation}}-N_{\text{other bkgs}}^{SS,\text{loosened isolation}}}.
\end{equation}
This correction factor takes the ratio between the observed data with other
backgrounds subtracted in the region equivalent to the signal region, but with the
charge requirement of the pair inverted, and the region equivalent to the sideband
with loosened isolation but with the charge requirement on the pair inverted.

The QCD estimate in the b-tag category is derived using an analogous method, 
 however, the sideband as used for the 
no b-tag category does not contain enough events to 
provide a background estimate. Therefore a sideband where
the highest \pT~tau passes the tight working point of the
tau ID discriminator, and the other tau passes the loose working
point of the discriminator, but not the tight working point, is used.
This sideband is slightly further away from the signal region than the
`tight--medium' sideband. %The bias from using this looser
%sideband has been assessed and has a less than 1$\sigma$ effect.
%FIXME THIS ISN'T VERY USEFUL INFO!

\subsubsection{\texorpdfstring{\emu channel}{e mu channel}}
\label{sec:mssm_bkgs_qcd_em}
In the \emu channel the QCD background
is estimated by inverting the charge requirement
of the pair, considering the same--sign region
with otherwise identical cuts to the signal region.
Other backgrounds present in this region are subtracted.
Like for the \etau and \mutau channels, the number of
QCD events with opposite--sign \emu pairs is not
necessarily equal to the number of QCD events
with same--sign \emu pairs and therefore the ratio
of opposite--sign to same--sign pairs is measured by inverting
the isolation requirements on the electron or muon. Both
leptons need to satisfy $I_{\text{rel}} < 0.4$ and at 
least one of them needs to fail the nominal isolation requirement.
The opposite--sign to same--sign ratio is parameterised in terms
of lepton kinematics and the separation in $\Delta R$ between the
two leptons. These ratios are then applied to the same--sign region
QCD estimate to give a QCD estimate in the signal region. 
Because the opposite--sign to same--sign ratios are measured
before applying the categorisation, and \ac{MC} studies suggest
that the opposite--sign to same--sign ratio is different in the b-tag
category than in the no b-tag category, an extra scale factor of 
1.45/2.20 is applied to the QCD estimate in the b-tag category.


\subsection{\texorpdfstring{\ttbar}{ttbar}}
\label{sec:mssm_bkgs_tt}
The \ttbar shape and normalisation are estimated from \ac{MC} 
samples and are checked against data in a control
region with a \ttbar purity of 91\%. This control region
is defined as $D_{\zeta} < -20$ GeV and \MET $>80$ GeV in 
the \emu channel, and it is shown in figure \ref{fig:mssm_corrs_toppt}.
Because the 6\%
\ttbar normalisation uncertainty covers the observed
discrepancy between data and \ac{MC}, no scale factor is applied.
In the \mutau, \etau and \tautau channels
the \ttbar contribution is split into
two components, one with real taus and 
one without, for fitting purposes. The component with real taus is composed
of \ttbar events in which the hadronic tau is matched
to a generator level tau. In the fully hadronic channel
both taus need to be generator matched to a hadronically
decaying tau.

\begin{figure}[h!]
\begin{center}
\includegraphics[width=0.5\textwidth]{./MSSM/Figures/mt_tot_ttcontrol_em_2016.png}
\end{center}
\caption{\mTtot~distribution in the \ttbar enriched control region.
The uncertainty band includes the statistical uncertainty as well as a 6\% \ttbar cross--section uncertainty.}
\label{fig:mssm_corrs_toppt}
\end{figure}


\subsection{Other backgrounds}
\label{sec:mssm_bkgs_other}
For all channels the di--boson plus 
single--top background is small.
Both normalisation and shape are estimated from \ac{MC}
samples. For the \etau, \mutau and \tautau channels
this background contribution is split into a component
where the hadronically decaying tau originates from a real
tau, and one where it does not, for purposes of the fit. This is done in a similar
way as for the \ttbar background.
In the \tautau and \emu channels the \Wjets background
is less important than in the \etau and \mutau channels, and both
its shape and normalisation are estimated from the
\ac{MC} samples. In the \emu channel the \PW\Pphoton background
is also added to the \Wjets background component. For plotting purposes the single-top plus
di-boson and \Wjets backgrounds are drawn together as the `electroweak' contribution for all channels, even though
they are considered as separate components in the fit.


\section{Systematic uncertainties}
\label{sec:mssm_uncs}
Just like for the analysis presented in chapter \ref{chap:hhh}
two types of systematic uncertainty are considered. Normalisation
uncertainties only affect the yield of a process while shape
uncertainties affect both the process normalisation and the shape
of the \mTtot~distribution. The uncertainties are taken into account 
in the final result as described in section \ref{sec:hhh_results_extraction}.

\subsection{Normalisation uncertainties}
\label{sec:mssm_uncs_norm}
\subsubsection*{Luminosity uncertainty}
The uncertainty on the luminosity measurement amounts to 6.2\% for
data collected during 2016 \cite{cms-pas-lum-15-001}, and it is
applied to all processes for which the normalisation is estimated 
using \ac{MC} samples.
\subsubsection*{Identification, isolation and trigger efficiencies}
The uncertainty on electron and muon identification, isolation, and
trigger efficiency amounts to 2\% \cite{CMS-PAS-HIG-16-037}. For hadronic taus the identification and
isolation uncertainty amounts to 6\% in the \etau and \mutau channels
and 12\% in the \tautau channel, with an additional 7\% uncertainty
on the tau trigger efficiency added in quadrature for that channel \cite{CMS-PAS-HIG-16-037}. This
uncertainty is split between a part that is correlated between the channels, 
a 5\% (10\%) uncertainty in the \etau and \mutau (\tautau) channels. The uncorrelated 
part amounts to 3\% (9.2\%) for the \etau and \mutau (\tautau) channels.
The uncertainties on lepton/hadronic tau identificiation, isolation and 
trigger efficiency are applied to all processes for which the normalisation
is estimated using \ac{MC} samples.
\subsubsection*{jet$\rightarrow\Pgt_{h}$ fake rate}
The uncertainty on the jet$\rightarrow\Pgt_{h}$ fake rate
is 20\% \cite{cms-tau-2015}. This uncertainty is applied
to those backgrounds where the normalisation is estimated from \ac{MC} and
where the hadronic taus are mimicked by jets, that is the \Zellell background
with a jet faking a hadronic tau, the \Wjets background in the fully hadronic
channel, and the \ttbar background without real hadronic taus.
\subsubsection*{$\Pe\rightarrow\Pgt_{h}$ and $\Pgm\rightarrow\Pgt_{h}$ fake rate}
The $\Pe\rightarrow\Pgt_{h}$ fake rate uncertainty ranges from 10-30\%,
depending on the anti-electron discriminator used \cite{cms-tau-2015}. This uncertainty is 
applied to the \Zellell background with the hadronic tau faked by an electron.
The $\Pgm\rightarrow\Pgt_{h}$ fake rate uncertainty ranges from 20-30\%, depending
on the anti-muon discriminator used\cite{CMS-PAS-HIG-16-037}. This uncertainty is applied to the \Zellell
background with the hadronic tau faked by a muon.
\subsubsection*{Jet energy scale uncertainty}
The uncertainties on the jet-energy corrections are applied
by shifting the jet energy up and down by the \pT-and $\eta$-dependendent
uncertainty, and evaluating the the change in process normalisation in
each category. Where the change in normalisation is non-negligible the
uncertainty is applied. It ranges from 1--10\% depending on process and category.
\subsubsection*{B-tag scale factors}
Uncertainties on the b-tagging efficiency and light jet mis-tagging
rates are given as a function of jet \pT-and $\eta$-for the 
medium working point of the \ac{CSV}v2 discriminator \cite{cms-btag-run2}.
The b-tagging scale factors are are varied within these uncertainties
and the overall change in normalisation for each process in each category is
evaluated. If the change in normalisation is non-negligible this
uncertainty is applied. The uncertainty varies from 1--5\%.
\subsubsection*{\MET~resolution and response}
Uncertainties on \MET~resolution and response are estimated
by varying the recoil correction parameters within their
uncertainty and evaluating the effect on process
normalisations. The uncertainty amounts to around 2\%.
\subsubsection*{Background normalisation}
\begin{itemize}
\setlength{\itemsep}{-\baselineskip}
\item \Ztautau: As the \Zmm control region is included in the simultaneous
fit with the signal region to correct the \Ztautau normalisation,
the Drell-Yan cross section is not applied as an uncertainty. In the no b-tag (b-tag) category a 3\% (5\%)
extrapolation uncertainty from \Zmm to \Ztautau is applied \cite{CMS-PAS-HIG-16-037}.
\item \Zellell: The uncertainty on the Drell-Yan cross-section
amounts to 4\% \cite{CMS-PAS-HIG-16-037}.
\item \ttbar: The uncertainty on the \ttbar production cross-section amounts
to 6\% \cite{CMS-PAS-HIG-16-037}.
\item di-boson and single-top: The uncertainty on the di-boson and
single-top production cross-sections amounts to 5\% \cite{CMS-PAS-HIG-16-037}.
\item \Wjets: In the \mutau and \etau channels the statistical uncertainties
on the observed data and subtracted backgrounds in the control regions are taken into account by
the inclusion of these control regions in the fit. The statistical
uncertainty on $R_{W}^{\text{OS/SS}}$ amounts to 2\% in the no b-tag
category in both the \etau and \mutau channels, and 11\% (14\%) in the b-tag
category of the \mutau (\etau) channel. The systematic uncertainty on
this ratio amounts to 8\% (10\%) in the no b-tag (b-tag) category of both channels.
The statistical uncertainty on the low \mT/high \mT~ratio is 2\% in the no b-tag
category of both channels and 14\% (17\%) in the b-tag category of the \mutau (\etau) channel.
The systematic uncertainty on this ratio amounts to 20\%. As the \Wjets background
in the \tautau and \emu channels is estimated from \ac{MC} a 4\% theoretical
production cross-section uncertainty is applied.
\item QCD: In the \emu channel the uncertainty on the QCD estimate
is taken as the uncertainty on the measured opposite--sign to same--sign
ratio, which is 23\% in the no b-tag category and 34 \% in the b-tag category. In the
\tautau channel the statistical uncertainty on the QCD estimate is derived from
the uncertainties on the observations and subtracted backgrounds in the
sidebands used to derive the estimate. This uncertainty amounts to 3\% in the no b-tag
category and 20\% in the b-tag category. An additional systematic uncertainty on the
extrapolation factor from the anti-isolated sideband into the signal region is found
to be 12\% in the no b-tag category and 14 \% in the b-tag category. For the \etau and
\mutau channels the statistical
uncertainties on the observation and subtracted \ac{MC} backgrounds in the control
region used to derive the QCD estimate is accounted for by the inclusion of these control
regions in the fit. The uncertainty on the opposite--sign to same--sign ratio, based on
the studies described in section \ref{sec:mssm_bkgs_etmt_qcdosss} is found to be 4\% (60\%) in the 
no b-tag (b-tag) category of the \mutau channel, and 12\% (60\%) in the no b-tag (b-tag) category
of the \etau channel. 
\end{itemize}
\subsubsection*{Signal theory uncertainties}
For interpretations of the results in MSSM benchmark scenarios, theory
uncertainties on the SM and MSSM Higgs cross section predictions are taken into account.
The uncertainties for the SM signal processes are described in more detail in reference \cite{YR4}.
Uncertainties due to different renormalisation scales
amount to 3.9\% for gluon fusion,
0.4\% for VBF, 2.8\% for ZH and 0.5\% for WH. The uncertainties
due to different choice of pdf and $\alpha_s$ amount to 3.2\% for gluon fusion, 2.1\% for vector boson fusion,
1.6\% for ZH and 1.9\% for WH.
For the MSSM Higgs cross sections used in the models, the uncertainties
due to different choices of factorisation and renormalisation scale
are computed, separately for each \mA-\tanb~point, following the prescription in refs\cite{pdf-lhc,alphas-uncs}.

\subsection{Shape uncertainties}
\label{sec:mssm_uncs_shape}
\subsubsection*{$\Pgt_{h}$ energy scale}
The energy of hadronic taus is varied up and down by 3\% \cite{CMS-PAS-HIG-16-037}, affecting
the shape of the total transverse mass distribution.
The uncertainty is applied to the signal samples and to
backgrounds containing real hadronic taus in the \etau, \mutau, and \tautau channels:
\Ztautau, and the components of the \ttbar and single-top+di-boson background with real 
hadronic taus.
\subsubsection*{Electron energy scale}
The energy of electrons is varied by 1\% in the barrel and by 2.5\% in the endcaps \cite{CMS-PAS-HIG-16-037}. This 
uncertainty is applied to the
signals and to the \Ztautau background in the \emu channel.
\subsubsection*{High-\pT~$\Pgt_{h}$ ID efficiency}
An additional $\Pgt_{h}$ identification uncertainty of $\frac{20\% \times p_{\text{T}}}{\text{TeV}}$
is applied to account for the extrapolation from the tau ID efficiency
measurement, which uses mostly taus close to the Z peak, to taus with higher transverse momenta \cite{CMS-PAS-HIG-16-037}. This
uncertainty is applied to signal and backgrounds with real taus in the \etau, \mutau and \tautau channels.
The shape difference between the nominal, up- and down- shapes for this uncertainty in the no b-tag category
of the \tautau channel can be seen for a high-mass signal sample in figure \ref{fig:mssm_highpttauid_shapes}a and
for the \Ztautau background in figure \ref{fig:mssm_highpttauid_shapes}b. Because the \Ztautau background
has much softer hadronic taus than the high-mass signal sample, the shape-altering effect of this uncertainty
is much smaller.
\begin{figure}[h!]
\begin{center}
\subfloat[gg$\phi$ signal at $m_{\phi} = 3.2$ TeV]{\includegraphics[width=0.5\textwidth]{./MSSM/Figures/highpT_tauID_comp_tt_nobtag_ggh3200.pdf}}
\subfloat[\Ztautau background]{\includegraphics[width=0.5\textwidth]{./MSSM/Figures/highpT_tauID_comp_tt_nobtag_ZTT.pdf}}
\end{center}
\caption{Shape-altering effect of the high-\pT~tau ID uncertainty in the no b-tag category of the
\tautau channel for (a) a high-mass signal sample and (b) the \Ztautau background. The effect is much
larger in the high-mass signal sample, which contains hadronically decaying taus with much higher
\pT~than the \Ztautau sample.}
\label{fig:mssm_highpttauid_shapes}
\end{figure}
\subsubsection*{Top quark \pT~reweighting}
To account for the uncertainty in the derivation of the top-quark \pT~reweighting, a shape
uncertainty is applied to the \ttbar background. This uncertainty amounts to a 100\% variation
of the correction, that is the difference between applying the correction twice and not applying the correction at all.
\subsubsection*{Drell-Yan shape reweighting}
An uncertainty on the Drell-Yan shape reweighting based on \Zmm events
is applied as 100\% of the correction to the \Ztautau background \cite{CMS-PAS-HIG-16-037}. 
\subsubsection*{Jet$\rightarrow\Pgt_{h}$ fake rate shape reweighting}
On the \Wjets background in the \etau and \mutau channels a shape 
uncertainty of $\frac{-20\% \times p_{\text{T}}}{\text{GeV}}$ is 
applied. This uncertainty is derived from variations in the data/MC ratio
for the jet$\rightarrow\Pgt_{h}$ fake rate as a function of jet \pT~in 
\Wjets events \cite{CMS-PAS-HIG-16-037}.

Overview tables of the systematic uncertainties per category of each
channel can be found in appendix \ref{appendix:uncerts}. The correlations
between the different channels and categories for these uncertainties
is also indicated.

\section{Signal extraction}
\label{sec:mssm_signalextraction}
The total transverse mass \mTtot~is used as discriminating variable in this analysis.
All bins of the distribution are used to perform a shape analysis. The statistical
methods, including the incorporation of nuisance parameters in the fit, were already
described in section \ref{sec:hhh_results_extraction}. Some additional
methods are used for this analysis, and they will be discussed in this section.

\subsection{Inclusion of control regions in the fit}
\label{sec:mssm_sigext_ctrl}
The control regions used for estimation of the 
\Wjets and QCD background contributions in the signal regions, discussed
in section \ref{sec:mssm_bkgs_etmt_ctrl}, are included in the fit. They
are added to the likelihood as single-bin counting experiments. In a given
category of one of the channels, the \Wjets
normalisation is fully correlated between the three control 
regions and the signal region. One common, freely floating, nuisance parameter
controls the \Wjets normalisation in all four regions. The QCD normalisation
is fully correlated between the opposite--sign and same--sign low-\mT~regions. The
same is true for the opposite--sign and same--sign high-\mT~regions. This means there are
two freely floating nuisance parameters that govern the QCD normalisation: one
that controls the QCD normalisation in the two low-\mT~regions, and one
that controls it in the two high-\mT~regions.

The \Zmm control regions mentioned in section \ref{sec:mssm_bkgs_ztt}
are also included in the likelihood as single-bin counting experiments.
The \Zmm control region in the no b-tag category is tied
to the \Ztautau normalisation in the no b-tag categories of all four
channels, again being controlled by a freely floating nuisance parameter. 
The same is true for the \Zmm control region in the b-tag category
and the \Ztautau normalisation in the b-tag categories of all four channels.

\subsection{Signal process profiling and 2D likelihood scans}
\label{sec:mssm_sigext_profile}
Because the analysis targets the two main MSSM production modes, gluon fusion and
b-associated production, limits are set on these two signal processes. As we
have seen in section \ref{sec:mssm_eventsel_categories}  we cannot fully separate the two signals by means
of the categorisation used. 
However, in the upper limits on cross--section times branching ratio for the gg$\phi$
process we should not make any assumptions about the presence or lack of the bb$\phi$
process and vice versa. Therefore when calculating the upper limits on the gg$\phi$ (bb$\phi$)
process, the contribution of the bb$\phi$ (gg$\phi$) process is profiled. This means it is floated
in the fit like a nuisance parameter.

Because the analysis targets these two signal processes, it is possible
to re-define the likelihood as a function of both the gg$\phi$ and bb$\phi$ signals:
\begin{equation}\label{mssm_2D_likelihood}
\mathcal{L}(\text{data}|\mu_{gg\phi},\mu_{bb\phi}, \theta) = \mathcal{L}(\text{data}|\mu_{gg\phi} \cdot s_{gg\phi}(\theta) + \mu_{bb\phi}\cdot s_{bb\phi}(\theta) + b(\theta)),
\end{equation}
where $s_{gg\phi}$ and $s_{bb\phi}$ are the signal expectations of the two
production processes and the signal strength modifiers $\mu_{gg\phi}$ and $\mu_{bb\phi}$ now represent the cross--section times branching ratio of the
two processes. Using this definition we can now perform a likelihood scan in two dimensions by 
constructing a two-dimensional grid in $\mu_{gg\phi}$ and $\mu_{bb\phi}$, both required to be greater than or equal to 0. 
The negative log-likelihood,
\begin{equation}\label{eqn:nll}
\text{NLL} = -\text{ln }\mathcal{L}(\text{data}|\mu_{gg\phi},\mu_{bb\phi},\theta),
\end{equation}
is then evaluated at each point. The point in the 2D grid where the NLL reaches
its lowest value gives the best fit value for $\mu_{gg\phi}$ and $\mu_{bb\phi}$.
68\% and 95\% confidence intervals are constructed as the values
$\mu_{gg\phi}^{N\%}$ and $\mu_{bb\phi}^{N\%}$ for which:
\begin{equation}\label{eqn:mssm_2D_deltaNLL}
\begin{split}
\Delta(\text{NLL})_{68\%} = \text{NLL}(\text{best fit}) - \text{NLL}(\mu_{gg\phi}^{68\%},\mu_{bb\phi}^{68\%}) = 0.5, ~\\
\Delta(\text{NLL})_{95\%} = \text{NLL}(\text{best fit}) - \text{NLL}(\mu_{gg\phi}^{95\%},\mu_{bb\phi}^{95\%}) = 1.92.
\end{split}
\end{equation}

\subsection{MSSMvsSM hypothesis testing}
\label{sec:mssm_sigext_mssmvssm}
The results of the search are interpreted in MSSM benchmark
scenarios, with the interpretation made in the \mA-\tanb~plane.
At each point in the \mA-\tanb~plane each MSSM scenario
predicts a production cross--section, and corresponding branching
ratio into $\Pgt\Pgt$, for each of the three neutral Higgs bosons. The
masses of the light h and the scalar H are also given as a function of \mA~and \tanb.
To interpret the results of this search in a particular scenario, the predicted
signal from the three neutral Higgs bosons is combined into a single signal template.
Examples of what such a combined signal profile might look like are given in figures \ref{fig:mssm_results_mttot_mt}--\ref{fig:mssm_results_mttot_em},
which give the \mTtot~distributions for all channels and categories with an example three-Higgs signal overlaid.
In this way the compatibility of the data with all three Higgs bosons, not just a single one, is tested.

Simply comparing a three-Higgs-signal-plus-background hypothesis with a background-only 
hypothesis to determine the exclusion power of the search in a particular
model is not strictly correct. Due to the existence of a Higgs boson with a 
mass of 125 GeV, all MSSM benchmark scenarios must include a light Higgs boson
with very similar properties to the boson discovered during Run 1 of the \ac{LHC}.
Therefore we should perform a test that distinguishes between three-Higgs-signal-plus-background, and
standard model Higgs-signal-plus-background.

This means the likelihood function defined in equation \ref{eqn:hhh_likelihood}
must be modified. Instead of testing for the compatibility of the data with
a signal + background expectation modified by the signal strenght modifier $\mu$ as
$\mu \cdot s(\theta) + b(\theta)$, a likelihood should be constructed as:
\begin{equation}\label{mssm_likelihood}
\mathcal{L}(\text{data}|\mu, \theta) = \mathcal{L}(\text{data}|\mu \cdot s_{\text{MSSM}}(\theta) + (1-\mu)\cdot s_{\text{SM}}(\theta) + b(\theta)),
\end{equation}
where $s_{\text{MSSM}}(\theta)$ corresponds to the MSSM signal expectation, that is the three-Higgs-signal, and $s_{\text{SM}}(\theta)$ 
corresponds to the SM signal expectation.
The signal strength modifier $\mu$ takes the role of distinguishing between
the two hypotheses in this case. The model must not allow for the coexistence 
of the SM and MSSM hypotheses; the MSSM hypothesis ($\mu=1$) has to be tested
against the SM hypothesis ($\mu=0$).

The modification of the likelihood function means it is no longer possible
to use the profile likelihood ratio of equation \ref{eqn:hhh_profilelikelihood}:
we need to test $\mu=1$ against $\mu=0$, and 
$\hat{\mu}$ in the denominator of equation \ref{eqn:hhh_profilelikelihood} is in 
general not 0. A solution to this problem can be found in test statistic used at the Tevatron \cite{LHCHComb2011},
\begin{equation}\label{eqn:mssm_tevatron_teststat}
q_{\mu} = -2\text{ln }\frac{\mathcal{L}(\text{data}|\mu,\hat{\theta_{\mu}})}{\mathcal{L}(\text{data}|0,\hat{\theta_0})} \text{ with } 0\leq\mu.
\end{equation}
In this test statistic we test a positive value of $\mu$ against $\mu=0$, thus we are able to define
the desired `MSSMvsSM' test statistic, using $\mu=1$, as,
\begin{equation}\label{eqn:mssm_mssmvssm_stat}
q_{\text{MSSMvsSM}} = -2\text{ln }\frac{\mathcal{L}(\text{data}|s_{\text{MSSM}}(\theta) + b(\theta))}{\mathcal{L}(\text{data}|s_{\text{SM}}(\theta)+b(\theta))}.
\end{equation}

The asymptotic approximation, discussed in section \ref{sec:hhh_results_extraction}, can not be applied
to the Tevatron test statistic, and therefore the probability density functions need to be generated 
with toy datasets. Using these pdfs we can then calculate $CL_s(\mu=1)$ as defined in equation \ref{eqn:hhh_cls}.
If $CL_s < 0.05$ the MSSM hypothesis is excluded at 95 \% CL. The generation of toy datasets and 
calculation of corresponding $CL_s$ value is performed for many points in a grid in the \mA-\tanb~plane,
with interpolation between grid points performed to obtain a smooth exclusion contour.
The pdfs for the SM and MSSM hypotheses, as well as the resulting $CL_s$ values,
are given for three \mA-\tanb~points in figure \ref{fig:mssm_mssmvssm_toys}. In figure \ref{fig:mssm_mssmvssm_toys}a
there is no separation between the MSSM and SM pdfs, and so we cannot exclude this point. The separation
between the pdfs in figure \ref{fig:mssm_mssmvssm_toys}b is much improved, and the observation is most compatible
with the SM hypothesis. This point is on the edge of being excluded. Finally the pdfs in figure \ref{fig:mssm_mssmvssm_toys}c
are very well separated, and the observation is compatible with the SM hypothesis, meaning this point of the parameter space is excluded.

\begin{figure}[h!]
\begin{center}
\subfloat[\mA=1.2 TeV, \tanb=16]{\includegraphics[width=0.5\textwidth]{./MSSM/Figures/plot_mA_1200_tanb_16.png}}
\subfloat[\mA=1.2 TeV, \tanb=50]{\includegraphics[width=0.5\textwidth]{./MSSM/Figures/plot_mA_1200_tanb_50.png}}~\\
\subfloat[\mA=700 GeV, \tanb=60]{\includegraphics[width=0.5\textwidth]{./MSSM/Figures/plot_mA_700_tanb_60.png}}
\end{center}
\caption{Pdfs, obtained by generating a large number of toy datasets, for the SM and MSSM hypothesis at three different points in the $m_{\text{h}}^{\text{mod+}}$
scenario, each with different levels of separation between the MSSM and SM pdfs.}
\label{fig:mssm_mssmvssm_toys}
\end{figure}

\section{Results}
\label{sec:mssm_results}
\subsection{Model--independent results}
\label{sec:mssm_results_modelindep}
The \mTtot~distributions in the no b-tag and b-tag
categories of all channels, after the fit to the observed
data has been performed, are shown in figures \ref{fig:mssm_results_mttot_mt}--\ref{fig:mssm_results_mttot_em}.
The signal of the 3 neutral Higgs bosons at \mA=1 TeV and \tanb=50 in the $m_{h}^{mod+}$ scenario is overlaid 
on the total transverse mass distributions. The signal peaks twice due to the presence of a light Higgs boson
with mass compatible with 125 GeV in this model, which is necessary to incorporate the Higgs boson
observed at 125 GeV.

\begin{figure}[h!]
\begin{center}
\subfloat[no b-tag]{\includegraphics[width=0.5\textwidth]{MSSM/Figures/CMS-PAS-HIG-16-037_Figure_005-a.pdf}}
\subfloat[b-tag]{\includegraphics[width=0.5\textwidth]{MSSM/Figures/CMS-PAS-HIG-16-037_Figure_005-b.pdf}}
\end{center}
\caption{Distributions of \mTtot~in the (a) no b-tag and (b) b-tag categories 
of the \mutau channel. The signal of the 3 neutral Higgs bosons at \mA=1 TeV 
and \tanb=50 in the $m_{h}^{mod+}$ scenario is overlaid. Note that, to incorporate
the observed Higgs boson at 125 GeV, the $m_{h}^{mod+}$ scenario includes a h boson
at around 125 GeV with a much larger $\sigma \times$BR than the 1 TeV H and A, and this is
the reason for the doubly--peaking shape of the overlaid signal \cite{CMS-PAS-HIG-16-037}.}
\label{fig:mssm_results_mttot_mt}
\end{figure}

\begin{figure}[h!]
\begin{center}
\subfloat[no b-tag]{\includegraphics[width=0.5\textwidth]{MSSM/Figures/CMS-PAS-HIG-16-037_Figure_006-a.pdf}}
\subfloat[b-tag]{\includegraphics[width=0.5\textwidth]{MSSM/Figures/CMS-PAS-HIG-16-037_Figure_006-b.pdf}}
\end{center}
\caption{Distributions of \mTtot~in the (a) no b-tag and (b) b-tag categories 
of the \etau channel. The signal of the 3 neutral Higgs bosons at \mA=1 TeV 
and \tanb=50 in the $m_{h}^{mod+}$ scenario is overlaid \cite{CMS-PAS-HIG-16-037}.}% Note that, to incorporate
%the observed Higgs boson at 125 GeV, the $m_{h}^{mod+}$ scenario includes a h boson
%at around 125 GeV with a much larger $\sigma \times$BR than the 1 TeV H and A, and this is
%the reason for the doubly--peaking shape of the overlaid signal \cite{CMS-PAS-HIG-16-037}.}
\label{fig:mssm_results_mttot_et}
\end{figure}

\begin{figure}[h!]
\begin{center}
\subfloat[no b-tag]{\includegraphics[width=0.5\textwidth]{MSSM/Figures/CMS-PAS-HIG-16-037_Figure_008-a.pdf}}
\subfloat[b-tag]{\includegraphics[width=0.5\textwidth]{MSSM/Figures/CMS-PAS-HIG-16-037_Figure_008-b.pdf}}
\end{center}
\caption{Distributions of \mTtot~in the (a) no b-tag and (b) b-tag categories 
of the \tautau channel. The signal of the 3 neutral Higgs bosons at \mA=1 TeV 
and \tanb=50 in the $m_{h}^{mod+}$ scenario is overlaid \cite{CMS-PAS-HIG-16-037}.}% Note that, to incorporate
%the observed Higgs boson at 125 GeV, the $m_{h}^{mod+}$ scenario includes a h boson
%at around 125 GeV with a much larger $\sigma \times$BR than the 1 TeV H and A, and this is
%the reason for the doubly--peaking shape of the overlaid signal \cite{CMS-PAS-HIG-16-037}.}
\label{fig:mssm_results_mttot_tt}
\end{figure}

\begin{figure}[h!]
\begin{center}
\subfloat[no b-tag]{\includegraphics[width=0.5\textwidth]{MSSM/Figures/CMS-PAS-HIG-16-037_Figure_007-a.pdf}}
\subfloat[b-tag]{\includegraphics[width=0.5\textwidth]{MSSM/Figures/CMS-PAS-HIG-16-037_Figure_007-b.pdf}}
\end{center}
\caption{Distributions of \mTtot~in the (a) no b-tag and (b) b-tag categories 
of the \tautau channel. The signal of the 3 neutral Higgs bosons at \mA=1 TeV 
and \tanb=50 in the $m_{h}^{mod+}$ scenario is overlaid \cite{CMS-PAS-HIG-16-037}.}% Note that, to incorporate
%the observed Higgs boson at 125 GeV, the $m_{h}^{mod+}$ scenario includes a h boson
%at around 125 GeV with a much larger $\sigma \times$BR than the 1 TeV H and A, and this is
%the reason for the doubly--peaking shape of the overlaid signal \cite{CMS-PAS-HIG-16-037}.}
\label{fig:mssm_results_mttot_em}
\end{figure}

None of these distributions show hints of a significant excess. The
95\% CL upper limits on $\sigma\times$BR are shown in figure \ref{fig:mssm_results_limits}a
for gluon fusion and in figure \ref{fig:mssm_results_limits}b for b--associated
production. All channels and categories are combined. 
\begin{figure}[h!]
\begin{center}
\subfloat[gg$\phi$]{\includegraphics[width=0.5\textwidth]{MSSM/Figures/CMS-PAS-HIG-16-037_Figure_009-a.pdf}}
\subfloat[bb$\phi$]{\includegraphics[width=0.5\textwidth]{MSSM/Figures/CMS-PAS-HIG-16-037_Figure_009-b.pdf}}
\end{center}
\caption{Upper limits at 95\% CL for (a) the gluon fusion production
process and (b) the b--associated production process. All four final states and 
all categories are combined for these limits \cite{CMS-PAS-HIG-16-037}.}
\label{fig:mssm_results_limits}
\end{figure}

A comparison of the expected limits
per channel is shown in figure \ref{fig:mssm_results_limits_breakdown}a for gg$\phi$ production
and in figure \ref{fig:mssm_results_limits_breakdown}b for bb$\phi$ production.
These figures show that the \mutau channel is the most sensitive at masses up to 
200 GeV, beyond which the \tautau channel becomes more sensitive. The \etau
channel is always slightly less sensitivie than the \mutau channel. The \emu channel
is the least sensitive over a large part of the parameter space, until it overtakes the 
\etau channel in sensitivity at 1.6 TeV and is slightly more sensitive than the
\mutau channel at 3.2 TeV. The \emu channel becomes more sensitive at higher masses as
a narrower binning of the \mTtot~distribution is possible than for the other channels.
The \mTtot~distribution for the high mass signals is therefore wide enough not to
fall entirely into the last bin used for the fit. This means at high mass this channel
benefits more from the shape discrimination between signal and backgrounds than
the other channels.

\begin{figure}[h!]
\begin{center}
\subfloat[gg$\phi$]{\includegraphics[width=0.5\textwidth]{MSSM/Figures/mssm_hig16037_limitcomp_ggh.pdf}}
\subfloat[bb$\phi$]{\includegraphics[width=0.5\textwidth]{MSSM/Figures/mssm_hig16037_limitcomp_bbh.pdf}}
\end{center}
\caption{Expected upper limits at 95\% CL for (a) gluon fusion and (b) b--associated production,
comparing the combination of all channels (green) with the \mutau (red), \etau (blue) \tautau (black)
and \emu (gold) channels. For masses below 200 GeV the \mutau channel is the most sensitive,
while the \tautau channel dominates for higher masses. The \etau channel is always
slightly less sensitive than the \mutau channel. The \emu channel is the least sensitive, 
until at around 1.6 TeV it starts to overtake the \etau channel, with the sensitivity at 3.2 TeV 
being similar or slightly better than the \mutau channel.}
\label{fig:mssm_results_limits_breakdown}
\end{figure}

\clearpage

\subsection{Sensitivity to the standard model Higgs boson}
\label{sec:mssm_results_125GeV}
As we set limits on MSSM Higgs boson production 
down to masses of 90 GeV it is important to consider possible sensitivity
to the 125 GeV SM Higgs boson. As the cross--section times branching ratio for gluon fusion
production of the 125 GeV Higgs boson in the standard model, with decay into $\Pgt\Pgt$,
is 3.05 pb \cite{YR4} and the expected limit on cross--section times
branching ratio for MSSM gg$\phi$ production at around 125 GeV is around 20 pb
the analysis is not yet sensitive to the SM Higgs boson. The
limits shown in figure \ref{fig:mssm_results_greenband},
which include the standard model Higgs boson as part of the backgrounds,
are almost exactly the same as those shown in figure \ref{fig:mssm_results_limits}, which
also indicates the current lack of sensitivity to the standard model Higgs boson.

\begin{figure}[h!]
\begin{center}
\subfloat[gg$\phi$]{\includegraphics[width=0.5\textwidth]{MSSM/Figures/CMS-PAS-HIG-16-037_Figure-aux_005-a.pdf}}
\subfloat[bb$\phi$]{\includegraphics[width=0.5\textwidth]{MSSM/Figures/CMS-PAS-HIG-16-037_Figure-aux_005-b.pdf}}
\end{center}
\caption{Upper limits at 95\% CL for (a) the gluon fusion production
process and (b) the b--associated production process. The SM, 125 GeV Higgs boson
is included as one of the backgrounds. All four final states and 
all categories are combined for these limits \cite{CMS-PAS-HIG-16-037-addit}.}
\label{fig:mssm_results_greenband}
\end{figure}


\subsection{2D likelihood scans}
\label{sec:mssm_results_2D}
The 2D likelihood scans, as described in section \ref{sec:mssm_sigext_profile},
are shown in figure \ref{fig:mssm_results_2D_scans}
for four mass points ranging from 100 GeV to 3.2 TeV. The black cross indicates the best-fit value, with the red diamond showing the expected
best-fit value in the presence of the 125 GeV SM Higgs boson. The dark and light purple contours
indicate the 68\% and 95\% confidence interval, respectively.  From the shape of the contours
we can see the correlation between the gg$\phi$ and bb$\phi$ processes, illustrating
the fact that we cannot fully disentangle the two processes.

\begin{figure}[h!]
\begin{center}
\subfloat[$m_{\phi} = 100$ GeV]{\includegraphics[width=0.5\textwidth]{./MSSM/Figures/CMS-PAS-HIG-16-037_Figure_011-a.pdf}}
\subfloat[$m_{\phi} = 700$ GeV]{\includegraphics[width=0.5\textwidth]{./MSSM/Figures/CMS-PAS-HIG-16-037_Figure_011-g.pdf}}~\\
\subfloat[$m_{\phi} = 1.6$ TeV]{\includegraphics[width=0.5\textwidth]{./MSSM/Figures/CMS-PAS-HIG-16-037_Figure_011-i.pdf}}
\subfloat[$m_{\phi} = 3.2$ TeV]{\includegraphics[width=0.5\textwidth]{./MSSM/Figures/CMS-PAS-HIG-16-037_Figure_011-l.pdf}}
\end{center}
\caption{2D likelihood scans showing the best-fit value (black cross) for cross--section times branching ratio 
of both the gluon fusion (x-axis) and b- associated production (y-axis) processes. The red diamond indicates the
expected best fit point for the presence of a 125 GeV SM Higgs boson, and the purple bands indicate the 68\% and
95\% confidence interval contours \cite{CMS-PAS-HIG-16-037}.}
\label{fig:mssm_results_2D_scans}
\end{figure}

\subsection{Interpretation in MSSM benchmark scenarios}
\label{sec:mssm_results_modeldep}
The results are interpreted in two MSSM benchmark scenarios: the $m_{\text{h}}^{\text{mod+}}$
scenario and the hMSSM scenario. Both scenarios are described in chapter \ref{chap:theory}.

Figure \ref{fig:mssm_mhmodp_2016}a shows the expected (dashed line and grey bands) and
observed (blue shaded area bounded by solid line) exclusion of the search presented in this chapter
in the \mA-\tanb~ plane of the $m_{\text{h}}^{\text{mod+}}$ scenario. The red shaded band
indicates the area that is excluded due to the lack of a light Higgs boson with a mass compatible
with 125 GeV. Compared with the most sensitive limits set during Run 1 of the \ac{LHC} these
limits at high \tanb~have been significantly improved, by around 300 GeV.

\begin{figure}[h!]
\begin{center}
\subfloat[$m_{h}^{\text{mod}+}$ scenario]{\includegraphics[width=0.65\textwidth]{./MSSM/Figures/CMS-PAS-HIG-16-037_Figure_012-a.pdf}}~\\
\subfloat[hMSSM scenario]{\includegraphics[width=0.65\textwidth]{./MSSM/Figures/CMS-PAS-HIG-16-037_Figure_012-b.pdf}}
\end{center}
\caption{Exclusion in (a) the $m_{h}^{\text{mod}+}$ scenario and (b) the hMSSM scenario 
obtained by the combination
of all channels in the \AHtotautau analysis. The blue shaded area bounded by the 
solid black line is the observed exclusion, with the dashed black line the
expected exclusion. The grey bands indicate the $\pm 1,2$ $\sigma$ 
expected exclusion. The red shaded area in the $m_{h}^{\text{mod}+}$ figure
is excluded by the lack of a light Higgs boson with mass compatible with 125 GeV \cite{CMS-PAS-HIG-16-037}.}
\label{fig:mssm_mhmodp_2016}
\end{figure}

Figure \ref{fig:mssm_mhmodp_2016}b shows the expected (dashed line and grey bands)
and observed (blue shaded area bounded by solid line) exclusion of the search
presented in this chapter in the \mA-\tanb~plane of the hMSSM scenario. 
We should note again that this model, although defined for \tanb~ values
upwards of 10, is only strictly valid for \tanb~below 10.
The small excluded area around \mA=200 GeV is genuine. It is caused by the 
production cross-section times branching ratio in the model, which is driven by
two effects \cite{CMS-PAS-HIG-16-007}. On the one hand the branching ratio into taus decreases with decreasing
\tanb, while on the other hand a negative top--bottom interference effect increases the
gluon fusion cross--section with decreasing \tanb. Therefore the
combined cross--section times branching ratio decreases with decreasing \tanb, until it
increases again from a low value of \tanb~down to \tanb=1. The feature is 
thus expected to grow with more data.

\chapter{Combination of MSSM analyses}
\label{sec:mssm_combination}
The results of the search for MSSM \AHtotautau on a dataset
corresponding to an integrated luminosity of 12.9 fb$^{-1}$, collected during the
first half of 2016 and thus referred to as `2016 analysis', can be combined 
with a previous search. This previous analysis, very similar to the 2016 analysis,
was performed on a dataset corresponding to an integrated luminosity of 2.3 fb$^{-1}$
collected during 2015 (`2015 analysis'), detailed in reference \cite{CMS-PAS-HIG-16-006}.
By combining the two analyses we should be able to set the most stringent
limits on the two MSSM production modes, and in the \mA-\tanb~plane of 
MSSM benchmark scenarios, available at the time of writing. It is also
a first step on the way to combining the \AHtotautau analysis with
searches for charged Higgs bosons or heavy neutral Higgs bosons in 
other final states. 

Because the 2015 analysis is very similar to the 2016 analysis it 
is not described in detail, however, the results will
be presented, and key differences with the more recent result highlighted
where needed.

\section{Results from 2015 analysis}
\label{sec:mssm_combination_2015}
The results from the search for \AHtotautau on a
dataset corresponding to an integrated luminosity of 2.3 fb$^{-1}$
collected at a centre-of-mass energy of 13 TeV during 2015 are presented
in this section. Like for the analysis presented in 
chapter \ref{chap:mssm} a shape analysis is performed, however a different
discriminating variable is used. The variable used is the transverse component of the di-$\Pgt$ mass
as reconstructed using the \texttt{SVFit} algorithm, 
denoted $m_{\text{T},\tau\tau}$. The transverse di-$\Pgt$ mass 
distributions for the signal region categories of the four channels are shown in figures \ref{fig:mssm_hig16006_mtsv_mt} -- \ref{fig:mssm_hig16006_mtsv_em}.
\begin{figure}[h!]
\begin{center}
\subfloat[No b-tag]{\includegraphics[width=0.5\textwidth]{./MSSM/Figures/htt_mt_8_hig16006_shapeshig16006_postfit_logy_logx.pdf}}
\subfloat[B-tag]{\includegraphics[width=0.5\textwidth]{./MSSM/Figures/htt_mt_9_hig16006_shapeshig16006_postfit_logy_logx.pdf}}
\end{center}
\caption{Transverse component of the di-$\Pgt$ mass for expected backgrounds and
observed events in the (a) no b-tag and (b) b-tag categories of the \mutau channel.
The signal overlaid corresponds to the signal of the three Higgs bosons at \mA=1 TeV and \tanb=50
in the $m_{\text{h}}^{\text{mod+}}$ scenario, and so it peaks once at low mass, for the light Higgs boson,
and once at higher mass, for the heavy H and A.}
\label{fig:mssm_hig16006_mtsv_mt}
\end{figure}

\begin{figure}[h!]
\begin{center}
\subfloat[No b-tag]{\includegraphics[width=0.5\textwidth]{./MSSM/Figures/htt_et_8_hig16006_shapeshig16006_postfit_logy_logx.pdf}}
\subfloat[B-tag]{\includegraphics[width=0.5\textwidth]{./MSSM/Figures/htt_et_9_hig16006_shapeshig16006_postfit_logy_logx.pdf}}
\end{center}
\caption{Transverse component of the di-$\Pgt$ mass for expected backgrounds and
observed events in the (a) no b-tag and (b) b-tag categories of the \etau channel.}
%The signal overlaid corresponds to the signal of the three Higgs bosons at \mA=1 TeV and \tanb=50
%in the $m_{\text{h}}^{\text{mod+}}$ scenario, and so it peaks once at low mass, for the light Higgs boson,
%and once at higher mass, for the heavy H and A.}
\label{fig:mssm_hig16006_mtsv_et}
\end{figure}

\begin{figure}[h!]
\begin{center}
\subfloat[No b-tag]{\includegraphics[width=0.5\textwidth]{./MSSM/Figures/htt_tt_8_hig16006_shapeshig16006_postfit_logy_logx.pdf}}
\subfloat[B-tag]{\includegraphics[width=0.5\textwidth]{./MSSM/Figures/htt_tt_9_hig16006_shapeshig16006_postfit_logy_logx.pdf}}
\end{center}
\caption{Transverse component of the di-$\Pgt$ mass for expected backgrounds and
observed events in the (a) no b-tag and (b) b-tag categories of the \tautau channel.}
%The signal overlaid corresponds to the signal of the three Higgs bosons at \mA=1 TeV and \tanb=50
%in the $m_{\text{h}}^{\text{mod+}}$ scenario, and so it peaks once at low mass, for the light Higgs boson,
%and once at higher mass, for the heavy H and A.}
\label{fig:mssm_hig16006_mtsv_tt}
\end{figure}

\begin{figure}[h!]
\begin{center}
\subfloat[No b-tag]{\includegraphics[width=0.5\textwidth]{./MSSM/Figures/htt_em_8_hig16006_shapeshig16006_postfit_logy_logx.pdf}}
\subfloat[B-tag]{\includegraphics[width=0.5\textwidth]{./MSSM/Figures/htt_em_9_hig16006_shapeshig16006_postfit_logy_logx.pdf}}
\end{center}
\caption{Transverse component of the di-$\Pgt$ mass for expected backgrounds and
observed events in the (a) no b-tag and (b) b-tag categories of the \emu channel.}
%The signal overlaid corresponds to the signal of the three Higgs bosons at \mA=1 TeV and \tanb=50
%in the $m_{\text{h}}^{\text{mod+}}$ scenario, and so it peaks once at low mass, for the light Higgs boson,
%and once at higher mass, for the heavy H and A.}
\label{fig:mssm_hig16006_mtsv_em}
\end{figure}

These distributions do not show any significant excesses. The upper limits
on cross--section times branching ratio for the gluon fusion and b-associated production
processes are shown in figure \ref{fig:mssm_results_hig16006_limits}a
and b, respectively. All channels and categories are combined. A comparison of
the expected limits per channel is given in figure \ref{fig:mssm_results_limits_breakdown_hig16006}.
These follow a similar pattern to those given in figure \ref{fig:mssm_results_limits_breakdown}, notice
however the \emu channel remains the least sensitive even at very high mass, as the
transverse component of the di-$\Pgt$ mass exhibits a sizeable \ttbar tail in the \emu channel. The total
transverse mass collapses events from the tail into lower regions of the distribution, thus reducing the backgrounds
at higher mass.

\begin{figure}[h!]
\begin{center}
\subfloat[gg$\phi$]{\includegraphics[width=0.5\textwidth]{MSSM/Figures/Figure_006-a.pdf}}
\subfloat[bb$\phi$]{\includegraphics[width=0.5\textwidth]{MSSM/Figures/Figure_006-b.pdf}}
\end{center}
\caption{Upper limits at 95\% CL for (a) the gluon fusion production
process and (b) the b--associated production process. All four final states and 
all categories are combined for these limits \cite{CMS-PAS-HIG-16-006}.}
\label{fig:mssm_results_hig16006_limits}
\end{figure}

\begin{figure}[h!]
\begin{center}
\subfloat[gg$\phi$]{\includegraphics[width=0.5\textwidth]{MSSM/Figures/mssm_hig16006_limitcomp_ggH.pdf}}
\subfloat[bb$\phi$]{\includegraphics[width=0.5\textwidth]{MSSM/Figures/mssm_hig16006_limitcomp_bbH.pdf}}
\end{center}
\caption{Expected upper limits at 95\% CL for (a) gluon fusion and (b) b--associated production,
comparing the combination of all channels (green) with the \mutau (red), \etau (blue) \tautau (black)
and \emu (gold) channels. For masses below 200 GeV the \mutau channel is the most sensitive,
while the \tautau channel dominates for higher masses. The \etau channel is always
slightly less sensitive than the \mutau channel. The \emu channel is the least sensitive.}
\label{fig:mssm_results_limits_breakdown_hig16006}
\end{figure}
\clearpage


\section{Combination procedure}
\label{sec:mssm_combination_procedure}
The idea of combining the two analyses is to combine the 
individual likelihoods to improve the analysis sensitivity and
physics reach. To do this correctly the correlations between
nuisance parameters in the 2015 and 2016 analyses need to be taken into account.
The correct way to establish the correlations is to perform
dedicated studies into them. In many cases such studies would be 
unfeasible to perform, or it would not be clear how to approach the 
problem in the first place.

Even without performing these studies, a choice of correlation scheme can be motivated
by carefully considering the differences between the two analyses and the conditions
under which they were performed.
This is what has been done for this combination.
%Strictly, studies into the underlying correlations between
%nuisance parameters should be performed. This is however not always
%feasible and in some cases
The correlation scheme between nuisance parameters relating to the 2015 and 2016
analyses, as used for this combination, is given in table \ref{tab:mssm_combination_correlations}. 
The third column of this table gives the motivation for the chosen correlation.

\begin{table}[htp]
\begin{center}
\caption{Correlations between nuisance parameters in 2015 and 2016 analysis}
{\footnotesize
\begin{tabular}{p{3cm}p{2cm}p{10cm}}
\toprule
\textbf{Nuisance} & \textbf{Correlation} & \textbf{Motivation}\\
\midrule
Luminosity & Partially \mbox{correlated} & Recommended by the CMS collaborators who perform the luminosity measurement, after dedicated studies.\\
\midrule
Jet energy scale & Fully \mbox{correlated} & Recommended by the CMS collaborators who derive jet energy corrections, after dedicated studies.\\
\midrule
B-tagging & Uncorrelated & Scale factors used in 2015 and 2016 analysis measured via different methods.\\
\midrule
Mis-tagging & Partially \mbox{correlated} & Recommended by the CMS collaborators responsible for deriving b-tagging scale factors, after dedicated studies.\\
\midrule
Muon ID/isolation/trigger & Uncorrelated & The 2015 and 2016 analyses use different muon ID and isolation working points, and the \ac{L1} trigger was replaced completely between 2015 and 2016.\\
\midrule
Electron ID/isolation/trigger & Uncorrelated & The \ac{L1} trigger was completely replaced between 2015 and 2016.\\
\midrule
Tau ID/isolation/trigger& Uncorrelated & \scriptsize{On top of the replacement of the \ac{L1} trigger, a tau ID scale factor was applied in the 2016 analysis but not in the 2015 analysis, despite an observed downward correction of 8\% on the yield of backgrounds with real hadronic taus.}\\
\midrule
High-\pT~tau ID \mbox{efficiency} & Correlated & This should not be affected by the lack of tau ID scale factor and covers the same underlying issue in both analyses.\\
\midrule
Tau energy scale & Correlated & There is no reason to expect a different best-fit energy scale between 2015 and 2016.\\
\midrule
Electron energy scale & Correlated & There is no reason to expect a different best-fit energy scale between 2015 and 2016.\\
\midrule
Drell-Yan shape & Correlated & \scriptsize{The same \ac{MC} samples, exhibiting the observed discrepancies, are used. In addition the weights as derived for the 2015 analysis and the 2016 analysis are very similar.}\\
\midrule
W jet$\rightarrow\Pgt_{h}$ fake rate shape & Uncorrelated & \scriptsize{Correction not applied in the 2016 dataset while it was applied for the 2015 analysis. The uncertainties have a different functional form, possibly due to different jet$\rightarrow\Pgt_{h}$ fake rates as a result of the use of a different tau isolation working point.}\\
\midrule
MET uncertainties & Correlated & The same MVA \MET training was used both for 2015 and 2016.\\
\midrule
\mbox{Top quark} \pT~reweighting & Correlated & The same corrections were applied to the 2015 and 2016 analyses.\\
\midrule
Drell-Yan $\sigma$& Correlated & Theoretical uncertainty so should be correlated.\\
\midrule
Di--boson $\sigma$ & Correlated & Theoretical uncertainty so should be correlated.\\
\midrule
\ttbar $\sigma$ & Correlated & Theoretical uncertainty so should be correlated.\\
\midrule
\Wjets $\sigma$ & Correlated & Theoretical uncertainty so should be correlated.\\
\midrule
SM signal theory uncertainties & Correlated & Theoretical uncertainties so should be correlated.\\
\midrule
\Ztautau \mbox{acceptance} & N/A & \scriptsize{Uncertainty only exists in 2015 analysis as no fits to the \Zmm control region were included yet.}\\
\midrule
\Ztautau \mbox{extrapolation}& N/A & \scriptsize{Uncertainty only exists in 2016 analysis as fits to the \Zmm control regions were included and this uncertainty should cover only the extrapolation from \Zmm to \Ztautau, not the full acceptance.}\\
\midrule
%e$\rightarrow\Pgt_{h}$ fake rate & Correlated & Expect to behave the same in 2015 and 2016 datasets.\\
%$\mu\rightarrow\Pgt_{h}$ fake rate & Uncorrelated & Measurement had not been made at time of 2015 analysis and so no scale factors were applied yet\\
%jet$\rightarrow\Pgt_{h}$ fake rate & Correlated & ....\\
%QCD extrapolation uncertainty (\emu channel) & Uncorrelated & Use of slightly different anti-iso sideband in 2015 and 2016 analyses\\
%QCD normalisation uncertainty (\tautau channel) & Uncorrelated & Use of different anti-isolated sideband\\
%Parameters tying control regions to signal region normalisations & Uncorrelated & 2016 control region fits should not influence the 2015 SR and vice versa\\
%Statistical uncertainty on W opposite--sign to same--sign ratio (\etau and \mutau)& Uncorrelated & Statistical uncertainties should not be correlated\\
%Systematic uncertainty on W opposite--sign to same--sign ratio (\etau and \mutau)& Uncorrelated & Uncertainty based on high \mT~ region with anti-isolated $\Pgt_{h}$, but because of the tau \pT~ cuts being different in 2015 and 2016 analyses we could expect a different event composition in this region. In addition this uncertainty was already uncorrelated between channels and categories in each individual analysis, and we can't claim to know the correlation between the two years any more accurately than that\\
%Statistical uncertainty on W low/high\mT~ ratio (\etau and \mutau)& Uncorrelated & Statistical uncertainties should not be correlated\\
%Systematic uncertainty on W low/high\mT~ ratio (\etau and \mutau) & Uncorrelated & This was already uncorrelated between channels and categories in each individual analysis, we can't claim to know the correlation between the two years any more accurately\\
%QCD OS/SS ratio statistical uncertainty (\etau and \mutau) & Uncorrelated & Statistical uncertainties should not be correlated\\
%QCD OS/SS ratio systematic uncertainty (\etau and \mutau) & Uncorrelated & The ratios were measured in different anti-isolated sidebands, and with different $\Pgt_{h}$ \pT~ selections, in the two analyses. Same comments about the correlations between channels and categories within the individual analyses as for the W systematic uncertainties apply.\\
%bin--by--bin uncertainties for templates with low numbers of events & Uncorrelated &Because the same initial \ac{MC} simulation was used these should in theory be correlated, in practice this is unfeasible to achieve\\
\end{tabular}}
\label{tab:mssm_combination_correlations}
\end{center}
\clearpage
\end{table}
\begin{table}[pt!]
\begin{center}
{\footnotesize
\begin{tabular}{p{3cm}p{2cm}p{10cm}}
\midrule
e$\rightarrow\Pgt_{h}$ fake rate & Correlated & Expect to behave the same in 2015 and 2016 datasets.\\
\midrule
$\mu\rightarrow\Pgt_{h}$ fake rate & Uncorrelated & Measurement had not been made at time of 2015 analysis and so no scale factors were applied yet, while this had changed for the 2016 analysis.\\
\midrule
jet$\rightarrow\Pgt_{h}$ fake rate & Correlated & Uncertainty on the jet$\rightarrow\Pgt_{h}$ fake rate should not be affected by changes in hadronic tau isolation working point.\\
\midrule
QCD extrapolation uncertainty (\emu channel) & Uncorrelated & Use of slightly different anti-isolated sideband in 2015 and 2016 analyses.\\
\midrule
QCD normalisation uncertainty (\tautau channel) & Uncorrelated & Use of different anti-isolated sideband.\\
\midrule
Parameters tying control regions to signal region normalisations & Uncorrelated & 2016 control region fits should not influence the 2015 SR and vice versa.\\
\midrule
Statistical \mbox{uncertainty} on W opposite--sign to same--sign ratio (\etau and \mutau)& Uncorrelated & Statistical uncertainties should not be correlated.\\
\midrule
Systematic \mbox{uncertainty} on W opposite--sign to same--sign ratio (\etau and \mutau)& Uncorrelated & \scriptsize{Uncertainty based on high \mT~ region with anti-isolated $\Pgt_{h}$, but because of the tau \pT~ cuts being different in 2015 and 2016 analyses we could expect a different event composition in this region. In addition this uncertainty was already uncorrelated between channels and categories in each individual analysis, and we can't claim to know the correlation between the two years any more accurately than that.}\\
\midrule
Statistical \mbox{uncertainty} on W low/high-\mT~\mbox{ratio} (\etau and \mutau)& Uncorrelated & Statistical uncertainties should not be correlated.\\
\midrule
Systematic \mbox{uncertainty} on W low/high-\mT~\mbox{ratio} (\etau and \mutau) & Uncorrelated & \scriptsize{This was already uncorrelated between channels and categories in each individual analysis, we can't claim to know the correlation between the two years any more accurately.}\\
\midrule
QCD OS/SS \mbox{ratio} statistical \mbox{uncertainty} (\etau and \mutau) & Uncorrelated & Statistical uncertainties should not be correlated.\\
\midrule
QCD OS/SS \mbox{ratio} systematic \mbox{uncertainty} (\etau and \mutau) & Uncorrelated & \scriptsize{The ratios were measured in different anti-isolated sidebands, and with different $\Pgt_{h}$ \pT~ selections, in the two analyses. Same comments about the correlations between channels and categories within the individual analyses as for the W systematic uncertainties apply.}\\
\midrule
bin--by--bin \mbox{uncertainties} for templates with low numbers of events & Uncorrelated &Because the same initial \ac{MC} simulation was used these should in theory be correlated, in practice this is unfeasible.\\
\bottomrule
\end{tabular}}
\end{center}
\end{table}
\clearpage



\section{Results}
\label{sec:mssm_combination_results}
The expected and observed upper limits on cross--section times branching ratio,
for the combination of the 2015 and 2016 analyses, are shown in figure
\ref{fig:mssm_results_combination_limits}. The limits behave as expected
from the individual 2015 and 2016 limits. 

\begin{figure}[h!]
\begin{center}
\subfloat[gg$\phi$]{\includegraphics[width=0.5\textwidth]{MSSM/Figures/mssm_combination_090117_ggH_cmb.png}}
\subfloat[bb$\phi$]{\includegraphics[width=0.5\textwidth]{MSSM/Figures/mssm_combination_090117_bbH_cmb.png}}
\end{center}
\caption{Upper limits at 95\% CL for (a) the gluon fusion production
process and (b) the b--associated production process. These results
combine all categories and all four final states from both the 2015
and the 2016 analysis.}
\label{fig:mssm_results_combination_limits}
\end{figure}

To get a better picture of how the sensitivity of the individual analyses compares
with the combination, we should consider the comparison
of expected limits in figure \ref{fig:mssm_results_combination_limits_comp}. The
gluon fusion limits in figure \ref{fig:mssm_results_combination_limits_comp}a show that at
higher masses the sensitivity is only very slightly improved with respect to the 
analysis of the 12.9 fb$^{-1}$ dataset. However, for masses between 200 and 800 GeV
there is a larger improvement in sensitivity from combining the two analyses.
This behaviour is expected since the analysis on the 2016 dataset used a slightly different
selection that made the analysis more optimal at masses above 800 GeV,
and slightly less optimal below this mass, than the analysis on the 2015 dataset. This
effect is less visible on the b--associated production limits shown in figure \ref{fig:mssm_results_combination_limits_comp}b.

\begin{figure}[h!]
\begin{center}
\subfloat[gg$\phi$]{\includegraphics[width=0.5\textwidth]{MSSM/Figures/mssm_combination_limitcomp_ggH.pdf}}
\subfloat[bb$\phi$]{\includegraphics[width=0.5\textwidth]{MSSM/Figures/mssm_combination_limitcomp_bbH.pdf}}
\end{center}
\caption{Comparison of the expected upper limits at 95\% CL for (a) the gluon fusion production
process and (b) the b--associated production process. The green curve shows the results
from combining the 2015 and 2016 analyses (integrated luminosity of 15.2 fb$^{-1}$),
with the red line indicating the limits from the 2016 analysis only (integrated luminosity of 12.9 fb$^{-1}$)
and the blue line indicating the limits from the 2015 analysis only (integrated luminosity of 2.3 fb$^{-1}$).}
\label{fig:mssm_results_combination_limits_comp}
\end{figure}

Interpretations in MSSM benchmark scenarios to be added.




